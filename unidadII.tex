\documentclass[10pt]{beamer}
\mode<presentation>
\usepackage{amssymb,textcomp}
%\usepackage{beamerthemesplit}
\usepackage{beamerthemeJuanLesPins}
\usepackage{verbatim}
\usepackage{algorithm2e}
\usefonttheme{serif}
\title{M\'etodos Iterativos para la Resoluci\'on Num\'erica de Sistemas de Ecuacionesl Lineales.}
\author{Jos\'e Luis Ram\'irez B.}
\date{\today}
\begin{document}
\frame{\titlepage}
%\section[Introducci\'on]{}
\frame{\tableofcontents}
\section{Introducci\'on}
\begin{frame}{Problemas de los m\'etodos directos para la resoluci\'on de (SL).}
  \begin{itemize}
   \item<1-> El m\'etodo de Gauss y sus variantes se conocen con el nombre de m\'etodos
   directos: se ejecutan un n\'umero finito de pasos y dan a lugar a una soluci\'on
   que ser\'ia exacta si no fuese por los errores de redondeo.
   \item <2->Cuando el tama\~no de la matriz $A$ es grande ($n >> 100$), la propagaci\'on del error de redondeo es tambi\'en grande, y los
  resultados obtenidos pueden diferir de los exactos.
  \end{itemize}
\end{frame}
  %%%%%%%
  \frame{
    \frametitle{Problemas de los m\'etodos directos para la resoluci\'on de (SL).}
    \begin{itemize}
      \item Muchas de las matrices que aparecen en (SL) poseen la mayor\'ia de sus elementos nulos. Estas matrices reciben el nombre de matrices dispersas o sparse.
      \begin{enumerate}
      \item<2-> Si los elementos no nulos est\'an distribuidos alrededor de la diagonal principal, son de aplicaci\'on todav\'ia los m\'etodos directos que conservan la estructura diagonal, como $LU$.
      \item<3-> Si no ocurre lo anterior, al aplicar m\'etodos directos se produce un fen\'omeno de llenado. Entonces, si no se realiza una adaptaci\'on de los m\'etodos directos los resultados no van a ser, en general, buenos.
      \end{enumerate}
    \end{itemize}
  }
  %%%%%%
  \section{M\'etodos Iterativos}
  \begin{frame}{M\'etodos Iterativo}
  \begin{itemize}
  \item<1-> Un m\'etodo iterativo que da resoluci\'on al sistema $Ax = b$ es aquel que genera, a partir de un vector inicial $x^{(0)}$, una sucesi\'on de vectores $x^{(1)},x^{(2)}, \ldots$.
  \item<2-> El m\'etodo se dir\'a que es consistente con el sistema $Ax = b$, si el l\'imite de dicha sucesi\'on, en caso de existir, es
  soluci\'on del sistema.
  \item<3-> Se dir\'a que el m\'etodo es convergente si la sucesi\'on generada por cualquier vector inicial $x^{(0)}$ es convergente a la soluci\'on del sistema.
  \item<4-> El vector $r^{(k)} = b-Ax^{(k)}$ es el vector residual obtenido en la $k$-\'esima iteraci\'on.
  \end{itemize}
  \end{frame}
  %%%%%%
  \begin{frame}{M\'etodos Iterativo}
  Si un m\'etodo es convergente es consistente, sin embargo, el rec\'iproco no es cierto.
  \uncover<2->{
  \begin{block}{Ejemplo:}
  El m\'etodo $x^{(n+1)} = 2x^{(n)} - A^{-1}b$ es consistente con el sistema $Ax = b$ pero no es convergente. En efecto:
  \end{block}}
  \end{frame}
  %%%%%%
  \begin{frame}{M\'etodos Iterativo}
  \begin{eqnarray}
  x^{(n+1)}-x & = & 2x^{(n)}-A^{-1}b-x = 2x^{(n)}-2x-A^{-1}b+x \nonumber\\
  & = & 2(x^{(n)}-x)-(A^{-1}b-x)\nonumber
  \end{eqnarray}
  y como $A^{-1}b = x$, se tiene que:
  \begin{block}{}
  $$
  x^{(n+1)}-x = 2(x^{(n)}-x)
  $$
  \end{block}
  \uncover<2->{
  Si existe \textcolor{red}{$\displaystyle\lim_{n \to \infty} x^{(n)} = x^*$}, se tiene que:
  \begin{block}{}
  $$
  x^* - x=2(x^* - x) \Rightarrow x^* - x =0 \Rightarrow x^* = x
  $$
  \end{block}
  es decir, el l\'imite es soluci\'on del sistema $Ax = b$, por lo que el m\'etodo es consistente.}
  \end{frame}
  %%%%%%
  \begin{frame}{M\'etodos Iterativo}
  Sin embargo, de $x^{(n+1)} - x = 2(x^{(n)} - x)$ se obtiene que:
  \begin{block}{}
  $$
  \|x^{(n+1)}-x\| = 2\|x^{(n)}-x\|
  $$
  \end{block}
  es decir, el vector $x^{(n+1)}$ dista el doble de lo que distaba $x^{(n)}$, por lo que el m\'etodo no puede ser convergente.
  \end{frame}  
  %%%%
  \section{Refinamiento Iterativo}
  \begin{frame}{Refinamiento Iterativo}
    \begin{itemize}
      \item Al resolver un sistema de ecuaciones $Ax = b$ utilizando un m\'etodo num\'erico
      se obtiene una aproximaci\'on $\tilde x$ de la verdadera soluci\'on del sitema.
      \item<2-> La exactitud de dicha soluci\'on depende de errores inherentes a los c\'alculos realizados.
      \item<3-> Sea $x$ la soluci\'on exacta del sistema y $\tilde x$ es la aproximaci\'on, por lo tanto cuando se sustituye $\tilde x$ en el sistema se obtiene:
      \begin{block}{}
      $$
      A\tilde x \approx b
      $$
      \end{block}
      esto significa que al realizar la resta $b - A\tilde x \neq 0$      
    \end{itemize}
  \end{frame}
  %%%%%
  \begin{frame}{Refinamiento Iterativo}
    \begin{itemize}      
      \item Definiendo a esta diferencia $r$ (residuo), as\'i $r = b - A\tilde x$.
      \item<2-> La soluci\'on deseada es
      de la forma $\tilde x + z$ tal que al sustituir en el sistema de ecuaciones se obtenga
      $$
      A(\tilde x + z) = b
      $$
      y desarrollando se obtiene
      \begin{eqnarray}
        \uncover<3->{\nonumber A(\tilde x + z) & = & b}\\
        \uncover<4->{\nonumber A\tilde x + Az & = & b}\\
        \uncover<5->{\nonumber Az & = & b - A\tilde x}\\
        \uncover<6->{\nonumber Az & = & r}
      \end{eqnarray}
      \item<7-> Una vez que obtenida $z$ se puede
      crear una mejor aproximación $\tilde x + z$ de la soluci\'on.
    \end{itemize}
  \end{frame}
  %%%%%%
  \begin{frame}
    \frametitle{Ejemplo:}
    Al resolver el sistema Ax = b donde
    $$
    A = \left[\begin{array}{ccc}
      60 & 30 & 20\\
      30 & 20 & 15\\
      20 & 15 & 12
    \end{array}\right] \quad \text{ y } \quad b = \left[\begin{array}{c}
      110\\
      65\\
      47
    \end{array}\right]
    $$
    suponiendo que una soluci\'on aproximada es $b = \left[\begin{array}{c}
      0.9\\
      0.8\\
      1.2
    \end{array}\right]$
  \end{frame}
  %%%%%%
  \begin{frame}
    \frametitle{Ejemplo:}
    \begin{itemize}
    \item Aplicando un paso de refinamiento iterativo tomando $tol = 10^{-5}$, se tendr\'ia que:
    $$
    x^{(0)} = \left[\begin{array}{c}
      0.9\\
      0.8\\
      1.2
    \end{array}\right]
    $$
    \item<2-> Calculando el residuo $r^{(0)} = b - Ax^{(0)} = \left[\begin{array}{c}
      8\\
      4\\
      2.6
    \end{array}\right]$
    \item<3->Verificando criterio de parada $\|r^{(0)}\|_{\infty} = 8 > tol$
  \end{itemize}
  \end{frame}
  %%%%%%
  \begin{frame}
    \frametitle{Ejemplo:}
    \begin{itemize}
      \item Obteniendo $z$ resolviendo el sistema $Az = r$ se obtiene $z =\left[\begin{array}{c}
        0.1\\
        0.2\\
        -0.2
      \end{array}\right]$
      \item<2->Generando la nueva aproximaci\'on
      $$
      x^{(1)} = x^{(0)} + z = \left[\begin{array}{c}
        0.9\\
        0.8\\
        1.2
      \end{array}\right] + \left[\begin{array}{c}
        0.1\\
        0.2\\
        -0.2
      \end{array}\right] = \left[\begin{array}{c}
        1\\
        1\\
        1
      \end{array}\right]
      $$
      \item<3-> Calculando el residuo $r^{(1)} = b - Ax^{(1)} = \left[\begin{array}{c}
        -0.14210854715202\\
        0\\
        0
      \end{array}\right]\times 10^{-13}$
    \end{itemize}
  \end{frame}
  %%%%%%
  \begin{frame}
    \frametitle{Ejemplo:}
    \begin{itemize}
      \item Verificando criterio de parada $\|r^{(1)}\|_{\infty} = 0.14210854715202\times 10^{-13} < tol$
      \item<2->Seg\'un el criterio de parada la mejor aproximaci\'on es $x=\left[\begin{array}{c}
        1\\
        1\\
        1
      \end{array}\right]$
    \end{itemize}
  \end{frame}
  %%%%%%}
  \begin{frame}
    \frametitle{Refinamiento Iterativo}
    \begin{itemize}
      \item Si suponemos que la soluci\'on aproximada al sistema lineal $A x = b$ se determina usando aritm\'etica de $t$ d\'igitos, se puede demostrar que el vector residual $r$ para la aproximaci\'on $\tilde x$ tiene la propiedad
      $$
       \|r\| = 10^{-t}\|A\|\|\tilde x\|
      $$
      \item<2-> De esta ecuaci\'on aproximada, se puede obtener una estimaci\'on del n\'umero de condici\'on efectivo para la aritm\'etica de $t$ d\'igitos, sin la necesidad de invertir la matriz $A$.
    \end{itemize}
  \end{frame}
  %%%%%%
  \begin{frame}
    \frametitle{Refinamiento Iterativo}
    \begin{itemize}
      \item La aproximaci\'on del n\'umero de condici\'on $\kappa(A)$ a $t$ d\'igitos viene de considerar el sistema lineal $A z = r$.
      \item<2-> De hecho $\tilde z$, la soluci\'on aproximada de $A z = r$, satisface que      
      $$
       \tilde z \approx A^{-1}r = A^{-1}(b-A\tilde x) = A^{-1}b - A^{-1}A\tilde x = x - \tilde x
      $$
      \item<3-> as\'i que $\tilde z$ es una estimaci\'on del error cometido al aproximar la soluci\'on del sistema original.      
      \begin{eqnarray}
       \nonumber \|\tilde z\| & \approx & \|x-\tilde x\|=\|A^{-1}r\| \leq \|A^{-1}\|\|r\|\\
       \nonumber & \approx & \|A^{-1}\|\left(10^{-t}\|A\|\|\tilde x\|\right) = 10^{-t}\|\tilde x\|\kappa(A)
      \end{eqnarray}
      \item<4-> Esto proporciona una aproximaci\'on para el n\'umero de condici\'on involucrado en la soluci\'on del sistema $A x = b$ usando $t$ d\'igitos:
      $$
       \kappa(A) \approx 10^t\frac{\|\tilde z\|}{\|\tilde x\|}
      $$      
    \end{itemize}
  \end{frame}      
  %%%%%%
  \begin{frame}
    \frametitle{Ejemplo:}
    \begin{itemize}
      \item<1->El sistema lineal $A x = b$ dado por
      $$
      \left(\begin{array}{ccc}
          3.333 & 15920 & -10.333\\
          2.222 & 16.71 & 9.612\\
          1.5611 & 5.1791 & 1.6852
        \end{array}\right)\left(\begin{array}{c}
        x_1\\
        x_2\\
        x_3
        \end{array}\right)=\left(\begin{array}{c}
        15913\\
        28.544\\
        8.4254
        \end{array}\right)
      $$      
      tiene la soluci\'on exacta $x = (1, 1, 1)^t$
      \item<2-> Usando eliminaci\'on Gaussiana y aritm\'etica de redondeo de 5 d\'igitos a
      $$
      \left(\begin{array}{ccc|c}
             3.333 & 15920 & -10.333 & 15913\\
             0 & -10596 & 16.501 &-10580\\
             0 & 0 & -5.079 & -4.7
            \end{array}\right)
      $$
      
      La soluci\'on aproximada a este sistema es
      
      $$
      \tilde x^{(0)} = (1.2001; 0.99991; 0.92538)^t
      $$
      \end{itemize}
\end{frame}
  %%%%%%
  \begin{frame}
    \frametitle{Ejemplo:}
    \begin{itemize}
      \item El vector residual correspondiente a $\tilde x$ calculado con doble precisi\'on (y luego redondeado a cinco d\'igitos) es 
      \small{
      \begin{eqnarray}
       \nonumber r^{(0)} & = & b - A \tilde x^{(0)} =\\
       \nonumber   & = & \left(\begin{array}{c}
            15913\\
            28.544\\
            8.4254
            \end{array}\right) - \left(\begin{array}{ccc}
             3.333 & 15920 & -10.333\\
             2.222 & 16.71 & 9.612\\
             1.5611 & 5.1791 & 1.6852
            \end{array}\right)\left(\begin{array}{c}
            1.2001\\
            0.99991\\
            0.92538
            \end{array}\right)\\
      \nonumber & = & \left(\begin{array}{c}
            -0.0051818\\
            0.27413\\
            -0.18616
            \end{array}\right)
      \end{eqnarray}}
      
      \item<2-> as\'i que 
       $$
       \|r^{(0)}\|_{\infty} = 0.27413
       $$
      \end{itemize}
    \end{frame}
  %%%%%%
  \begin{frame}
    \frametitle{Ejemplo:}
    \begin{itemize}
      \item<1-> La estimaci\'on del n\'umero de condici\'on se obtiene resolviendo primero el sistema $A z^{(0)} = r$:      
      $$
      \left(\begin{array}{ccc}
             3.333 & 15920 & -10.333\\
             2.222 & 16.71 & 9.612\\
             1.5611 & 5.1791 & 1.6852
            \end{array}\right)\left(\begin{array}{c}
            z_1\\
            z_2\\
            z_3
            \end{array}\right)=\left(\begin{array}{c}
          -0.0051818\\
          0.27413\\
          -0.18616
            \end{array}\right)
      $$
      \item <2-> La soluci\'on $z^{(0)} = (-0.20008; 8.9989 \times 10^{-5}; 0.074607)^t$ Usando la estimaci\'on del n\'umero de condici\'on      
       $$
       \kappa(A) \approx 10^5\frac{\|\tilde z^{(0)}\|_{\infty}}{\|\tilde x^{(0)}\|_{\infty}} = \frac{10^5(0.20008)}{1.2001} = 16672
       $$
      \end{itemize}
    \end{frame}  
  %%%%%%
  \begin{frame}
    \frametitle{Ejemplo:}
    \begin{itemize}
      \item Calculado $\tilde z^{(0)}$ se puede generar la nueva aproximaci\'on $\tilde x^{(1)}$
      $$
      \tilde x^{(1)} = \tilde x^{(0)} + \tilde z^{(0)} = (1.0000; 1.0000; 0.99999)^t
      $$
      \item<2-> y el error real en esta aproximaci\'on es
      $$
      \|x - \tilde x^{(1)}\|\infty = 1.0 \times 10^{-5}
      $$
      \item<3-> calculando $r^{(1)} = b - A \tilde x^{(1)}$, y resolviendo el sistema $A z^{(1)} = r^{(1)}$, se obteniene
      $$
      \tilde z^{(1)} = (-2.7003 \times 10^{-8}; 1.2973 \times 10^{-8}; 9.9817 \times 10^{-6})^t
      $$
      \item<4-> Puesto que $\|\tilde z^{(1)}\| \leq 10^{-5}$, se concluye que
      $$
      \tilde x^{(2)}= \tilde x^{(1)} + \tilde z^{(1)} = (1.0000; 1.0000; 1.0000)^t
      $$      
      es suficientemente preciso.
    \end{itemize}
  \end{frame}
  %%%%%%
  \begin{frame}
    \frametitle{Ejemplo:}
    \begin{itemize}
      \item<1-> Se ha usado la estimaci\'on $\tilde z \approx x - \tilde x$, donde $\tilde z$ es la soluci\'on aproximada al sistema $A z = r$.
      \item<2-> A partir de este resultado, se genera la nueva aproximaci\'on $\tilde x + \tilde z$.
      \item<3-> Este proceso puede ser repetido para refinar la soluci\'on sucesivamente hasta alcanzar convergencia.
    \end{itemize}
  \end{frame}
  %%%%%  
  \begin{frame}    
    \frametitle{Algoritmo}
    \small
\begin{algorithm}[H]
 \SetKwInOut{Input}{input}
 \SetKwInOut{Output}{output}
 \caption{Algoritmo de Refinamiento Iterativo.}
 \Input{$A \in \mathbb{R}^{n \times n}$, $b \in \mathbb{R}^n$, N\'umero m\'aximo de iteraciones $N$, tolerancia $TOL$.}
 \Output{Soluci\'on aproximada $x\in\mathbb{R}^n$.}
 \BlankLine
 Resolver $Ax=b$\\
 \For{$k\leftarrow 1$ \KwTo $N$}
 {
  $r=b-Ax$\\
  Resolver $Ay=r$ (usando eliminaci\'on Gaussiana en el mismo orden que en el paso 1).\\
  Calcular $K(A) = 10^t\displaystyle\frac{\|y\|}{\|x\|}$ (solo se calcula la primera vez).\\
  $x=x+y$\\
  \If{$\|y\|<TOL$}
  {
    salida $x$\\
    parar
  }
 }
\end{algorithm}
  \end{frame}
  %%%%%%
  \section{M\'etodos Iterativos de Punto Fijo}
  \begin{frame}{Esquemas de Punto Fijo}
    Supongase un (SL) $Ax = b$, se busca una matriz $T \in \mathcal{M}_n$ y un vector $c \in \mathbb{R}^{n}$, de forma que la matriz $I - T$ sea inversible y que la \'unica soluci\'on del sistema lineal
\begin{block}{}
$$
 \underbrace{x=Tx+c}_{(I-T)x=c}
 $$
\end{block}
es la soluci\'on  de $Ax=b$.
\end{frame}    
%%%%%%
\begin{frame}{Esquemas de Punto Fijo}
  Considerando $x^{(0)} \in \mathbb{R}^{n}$ un vector arbitrario, se construye una sucesi\'on de vectores $\{x\}_{k=0}^{\infty}$ dada por
  \begin{block}{}
   $$
   x^{(k+1)} = Tx^{(k)}+c; \qquad k \in \mathbb{N}\cup \{0\}
   $$
  \end{block}
  y se pretende que la sucesi\'on $\{x\}_k$ converja a la soluci\'on del sistema lineal.
  \end{frame}
  %%%%%%  
  \begin{frame}{Esquemas de Punto Fijo}
    \begin{block}{Definici\'on}
    El m\'etodo iterativo $x^{(k+1)} = Tx^{(k)} + c$ es convergente si existe un vector $x \in \mathbb{R}^n$ tal que:
    $$
    \lim_{k\rightarrow\infty}x^{(k)} = x
    $$
    para cualquier vector inicial $x^{(0)} \in \mathbb{R}^n$. En ese caso,
    $$
    x = Tx + c
    $$
    \end{block}
    \end{frame}
    %%%%%%
    \begin{frame}{Criterio de Convergencia}
    El error en cada iteraci\'on se puede medir, por tanto como:
    \uncover<2->{
    \begin{block}{}
    $$
    e^{(k)} = x^{(k)} - x
    $$
    \end{block}}
    \uncover<3->{Se tiene que:}
    \begin{eqnarray}
    \uncover<4->{e^{(k)} = x^{(k)} - x & = & (Tx^{(k-1)} + c) - (Tx + c) = T(x^{(k-1)} - x) =\nonumber\\}
    \uncover<5->{&&Te^{(k-1)} =  \cdots = T^ke^{(0)}\nonumber\\}
    \uncover<6->{&&\Rightarrow e^{(k)} = T^ke^{(0)}\nonumber}
    \end{eqnarray}
    \uncover<7->{De ese modo, el error en las iteraciones depende de las potencias sucesivas de la matriz $T$, lo que dar\'a el criterio para la convergencia del M\'etodo Iterativo.}
    \end{frame}
    %%%%%%
    \begin{frame}{Criterio de Convergencia}
    \begin{itemize}
    \item<1-> Se mostr\'o por inducci\'on que
    $$
    e^{(k)} = T^ke^{(0)}
    $$
    \item<2-> Entonces $e^{(k)} \to 0$ para todo $e^{(0)}$ si y s\'olo si $ T^k \to 0$.
    \begin{block}{}
    $$
     \lim_{k \to \infty}\|e^{(k)} \| \to 0 \quad sii \quad \lim_{k \to \infty}\|T^k\| \to 0
     $$
    \end{block}
    \item<3-> Si $T$ es una matriz diagonalizable, exiten matrices $P$ y $\Lambda$, tales que $T=P\Lambda P^{-1}$, donde $\Lambda$ es una matriz diagonal con los autovalores de $T$ en la diagonal.
    \begin{eqnarray}
     T & = & P \Lambda P^{-1}\nonumber\\
     T^{k} & = & P \Lambda P^{-1}P \Lambda P^{-1}\cdots P \Lambda P^{-1}\nonumber\\
     T^{k} & = & P \Lambda^{k} P^{-1}\nonumber
    \end{eqnarray}
    \end{itemize}
    \end{frame}
    %%%%%%
    \begin{frame}{Criterio de Convergencia}
    \begin{itemize}
     \item<1-> Donde
     $$
     \Lambda^{k} = \left(\begin{array}{cccc}
                          \lambda_1^{k} & & & \\
                          & \lambda_2^{k} & & \\
                          & & \ddots & \\
                          & & & \lambda_n^{k}
                         \end{array}\right)
     $$
     \item<2-> De esta manera
     \begin{block}{}
     \begin{eqnarray}
        \lim_{k \to \infty}\|T^{k}\| \to 0 & \Rightarrow & \lim_{k \to \infty}\|P \Lambda^{k} P^{-1}\| \to 0 \Rightarrow \lim_{k \to \infty}\|\Lambda^{k}\| \to 0\nonumber\\
         & \Rightarrow & \lim_{k \to \infty}|\lambda_i^{k}| \to 0 \Rightarrow |\lambda_i|<1\quad i=1,2,\ldots,n\nonumber
     \end{eqnarray}
     \end{block}
    \end{itemize}
    \end{frame}
    %%%%%%
    \begin{frame}{Criterio de Convergencia}
    \begin{itemize}
     \item Si la magnitud de todos los autovalores de la matriz de iteraci\'on $T$ son menores que 1, entonces el esquema iterativo es convergente.
     \uncover<2->{\begin{block}{Teorema}
                   Un esquema iterativo definido por $x^{(k+1)} = Tx^{(k)}+c$ es convergente si y s\'olo si $\rho(T) < 1$.
                  \end{block}}
     \item<3-> Es m\'as sencillo calcular la norma de una matriz que el c\'alculo de sus autovalores
     $$
     \left.\begin{array}{lcl}
            Tx & = & \lambda x \\
            \|Tx\| & = & |\lambda|\|x\|\\
            \|Tx\| & \leq & \|T\|\|x\|
           \end{array}\right\} \Rightarrow |\lambda|\|x\| \leq \|T\|\|x\| \Rightarrow |\lambda| \leq \|T\|
     $$
     \uncover<4->{Por lo que una condici\'on suficiente para la convergencia es: \textcolor{red}{$\|T\| < 1$}}
    \end{itemize}
    \end{frame}
    %%%%%%
    \begin{frame}{Tasa de Convergencia}
    \begin{itemize}
    \item<1-> En el dise\~no de un m\'etodo iterativo, adem\'as de asegurar la consistencia y convergencia del m\'etodo, es importante el concepto de velocidad de convergencia.
    \item<2-> Supongase que se tiene una aproximaci\'on de la soluci\'on, $x^{(k)}$ con $q$ cifras significativas,
    $$
    \|e^{(k)}\| \leq \frac{1}{2}10^{-q}\|x\|
    $$
    \item<3-> y se desea saber cu\'antas iteraciones m\'as se tienen que hacer para obtener $m$ cifras correctas m\'as. Interesa, por lo tanto, saber cu\'antas iteraciones $N$ hay que hacer para que,
    $$
    \|e^{(k+N)}\| \leq \frac{1}{2}10^{-(q+m)}\|x\|
    $$
    es decir
    \begin{block}{}
     $$
    \|e^{(k+N)}\| \leq 10^{-m}\|e^{(k)}\|
    $$
    \end{block}
    \end{itemize}
    \end{frame}
    %%%%%
    \begin{frame}{Tasa de Convergencia}
    \begin{itemize}
     \item<1-> Esto permitir\'a dar informaci\'on sobre la velocidad con que la sucesi\'on de aproximaciones $x^{(k)}$ se acerca a la soluci\'on.
     \item<2-> Se sabe que $e^{(k+N)} = T^{N}e^{(k)}$, que tomando norma es
     $$
     \|e^{(k+N)}\| \leq \|T^{N}\|\|e^{(k)}\|
     $$
     \item<3-> De manera que para conseguir las $m$ cifras significativas es suficiente exigir que $\|T^{N}\| \leq 10^{-m}$ o, equivalentemente
     $$
     -\log_{10}\|T^{N}\| \geq m
     $$
     \item<4-> As\'i, para conseguir $m$ cifras significativas es suficiente hacer $N$ iteraciones con
     $$
     N \geq \frac{m}{R}
     $$
    \end{itemize}
    \end{frame}
    %%%%%%%
    \begin{frame}{Tasa de Convergencia}
    \begin{itemize}
     \item Donde $R$ se define como:
     $$
     R = -\frac{1}{N}\log_{10}\|T^{N}\| = -\log_{10}\left(\|T^{N}\|^{1/N}\right)
     $$
     \item El par\'ametro $R$ se puede interpretar como la velocidad de convergencia: cuanto mayor es $R$ menos iteraciones hay que hacer para incrementar la precisi\'on en la aproximaci\'on.
     \item Se puede comprobar que
     $$
     \lim_{N \to \infty}\|T^{N}\|^{1/N} = \rho(T)
     $$
     por lo tanto, la velocidad asint\'otica de convergencia del m\'etodo es
     $$
     R_{\infty} = \log_{10}\left(\frac{1}{\rho(T)}\right)
     $$
     \end{itemize}
    \end{frame}
    %%%%%%
    \begin{frame}{Tasa de Convergencia}
    \begin{itemize}
     \item<1-> El radio espectral hace el papel del factor asint\'otico de convergencia del m\'etodo.
     \item<2-> Para $\rho(T) \simeq 0$, la velocidad de convergencia $R_{\infty}$ es muy grande y el m\'etodo tiene una convergencia muy r\'apida.
     \item<3-> Para $\rho(T) = 1 - \varepsilon$, con $\varepsilon \simeq 0$ positivo, la velocidad de convergencia es positiva pero peque\~na y el m\'etodo converge pero lo hace lentamente.
     \item<4-> En cambio, si $\rho(T) > 1$ la velocidad de convergencia es negativa por lo que la convergencia no est\'a asegurada.
    \end{itemize}
    \end{frame}
    %%%%%   
    \begin{frame}{Construcci\'on de los M\'etodos}
      Considerando la descomposici\'on
      $$
      A = M-N, \quad M \mbox{ regular }
      $$
      \uncover<2->{
      El sistema $Ax=b$ se puede escribir en la forma
      $$
      Mx = Nx+b,\mbox{ es decir, } x = M^{-1}Nx + M^{-1}b
      $$
      lo que sugiere el esquema iterativo}
      \uncover<3->{
      \begin{block}{}
      $$
      x^{(k+1)}=M^{-1}Nx^{(k)} + M^{-1}b
      $$
      \end{block}
      donde $T=M^{-1}N$ y $c=M^{-1}b=(I-T)A^{-1}b$, por lo que el m\'etodo as\'i construido es consistente con el sistema.}
      \end{frame}
      %%%%%
      \begin{frame}{Construcci\'on de los M\'etodos}
      \begin{itemize}
       \item El esquema anterior tamb\'en es esquivalente a
      \begin{block}{}
       $$
       x^{(k+1)} = x^{(k)} + M^{-1}(b-Ax^{(k)})
       $$
      \end{block}
      donde $r^{(k)} = b-Ax^{(k)}$ es el residual de la $k$-\'esima iteraci\'on y $M$ es un precondicionador del sistema.
      \item<2-> En lo sucesivo se considera la descomposici\'on aditiva de la matriz $A$ como
      \begin{center}
      \includegraphics[scale=0.5]{descomA.png}
      \end{center}
      donde $L_A$ es la parte triangular inferior de la matriz sin incluir la diagonal, $D_A$ es la diagonal de $A$, y $U_A$ es la parte triangular superior sin incluir la diagonal.
      \end{itemize}
      \end{frame}
      %%%%%%
      \subsection{M\'etodo de Jacobi}      
      \begin{frame}{M\'etodo de Jacobi}
      \begin{itemize}
       \item Se basa en una descomposici\'on de la matriz $A$ del tipo $A = M - N$ , con $M = D$, parte diagonal de la matriz $A$, y $N = L + U$
       \item<2-> Obliga a resolver un sistema diagonal en cada paso.
       \item <3-> De esta manera queda que $A=D-L-U=D-(L+U)$, por lo tanto el sistema se transforma en:       
       $$
       Ax=b \Leftrightarrow (D-(L+U))x=b \Leftrightarrow Dx=(L+U)x + b
       $$
       \end{itemize}
      \end{frame}
      %%%%%%
      \begin{frame}{M\'etodo de Jacobi}
      \begin{itemize}
        \item<1-> As\'i, se calcula la soluci\'on del sistema, como el l\'imite de la sucesi\'on $\{x^{(k)}\}_{k \in N}$ donde se define el t\'ermino $(k+1)$-\'esimo como:
       
       $$
       x^{(k+1)}= D^{-1}(L+U)x^{(k)} + D^{-1}b
       $$
       
       siendo as\'i su matriz de iteraci\'on:
       
       $$
       T=D^{-1}(L+U)
       $$
       
       Entonces el sistema ha sido escrito como un proceso iterativo de la forma
       $$
       x^{(n+1)}= Tx^{(n)}+c .
       $$
      \end{itemize}
      \end{frame}
      %%%%%%       
      \begin{frame}{M\'etodo de Jacobi}
      \begin{itemize}
       \item<1-> Dado un sistema de ecuaciones:
       $$
       \left\{\begin{array}{c}
               a_{1,1}x_1 + a_{1,2}x_2 + \cdots + a_{1,i}x_i + \cdots + a_{1,n}x_n = b_1\\
         a_{2,1}x_1 + a_{2,2}x_2 + \cdots + a_{2,i}x_i + \cdots + a_{2,n}x_n = b_2\\
                   \vdots\\
         a_{i,1}x_1 + a_{i,2}x_2 + \cdots + a_{i,i}x_i + \cdots + a_{i,n}x_n = b_i\\
                   \vdots\\
         a_{n,1}x_1 + a_{n,2}x_2 + \cdots + a_{n,i}x_i + \cdots + a_{n,n}x_n = b_n
              \end{array}
       \right.
       $$
      \end{itemize}
    \end{frame}
    %%%%%%
    \begin{frame}{M\'etodo de Jacobi}
    \begin{itemize}
      \item<1->Si $a_{i,i}\neq 0$ para todo $i=1,\ldots,n$, entonces despejando $x_i$ de la $i$-\'esima ecuaci\'on, obtenemos el sistema equivalente       
       $$
       x_i = -\sum_{j=1,j\neq i}^n\left(\frac{a_{i,j}}{a_{i,i}}\right)x_j + \frac{b_i}{a_{i,i}}, \qquad i=1,\ldots,n
       $$       
       \item<2->Conocido $x^{(k-1)}$, se calcula la aproximaci\'on $x^{(k)}$, mediante la f\'ormula:       
       $$
       x_i^{(k)} = \displaystyle\frac{b_i-\displaystyle\sum_{j=1,j\neq i}^na_{i,j}x_j^{(k-1)}}{a_{i,i}}, \qquad i=1,\ldots,n
       $$       
       que es llamada f\'ormula escalar de iteraci\'on del m\'etodo de Jacobi.
      \end{itemize}
      \end{frame}     
      %%%%%%
      \begin{frame}{Ejemplo:}      
      Resolver el sistema de ecuaciones
      $$
\left(\begin{array}{ccc}
       4 & 1 & 0\\
       1 & 4 & 1\\
       0 & 1 & 4
      \end{array}\right)\left(\begin{array}{c}
      x_1\\
      x_2\\
      x_3
      \end{array}\right)=\left(\begin{array}{c}
      -3\\
      10\\
      1
      \end{array}\right)
      $$
      cuya soluci\'on es el vector

$$
\left(\begin{array}{c}
      x_1\\
      x_2\\
      x_3
      \end{array}\right)=\left(\begin{array}{c}
      -1.5\\
      3\\
      -0.5
      \end{array}\right)
$$
      y tomando como vector inicial $x^{(0)} = \left(\begin{array}{c}
      -1\\
      4\\
      -1
      \end{array}\right)$
      \end{frame}
      %%%%%%
      \begin{frame}{Ejemplo:}
        Procedamos a realizar la descomposici\'on de la matriz del sistema:
        \small{
        $$
        M=D=\left(\begin{array}{ccc}
               4 & 0 & 0\\
               0 & 4 & 0\\
               0 & 0 & 4
              \end{array}\right) \quad L=\left(\begin{array}{ccc}
               0 & 0 & 0\\
               -1 & 0 & 0\\
               0 & -1 & 0
              \end{array}\right) \quad U=\left(\begin{array}{ccc}
               0 & -1 & 0\\
               0 & 0 & -1\\
               0 & 0 & 0
              \end{array}\right)
        $$}           
        con        
        $$
        N=L+U
        $$
        Para comprobar si estamos convergiendo a la soluci\'on tomaremos elvalor de la norma del residuo $r^{(k)} = b - Ax^{(k)}$.
        $$        
        r^{(0)}	=b-Ax^{(0)} = \left(\begin{array}{c}
        -3\\
        -4\\
        1
        \end{array}\right) \Rightarrow \|r^{(0)}\|_\infty = 4
        $$
      \end{frame}   
      %%%%%%
      \begin{frame}{Ejemplo:}     
        El primero paso del esquema vendr\'ia dado por:
        $$
        Dx^{(1)} = N x^{(0)} + b = (L + U )x^{(0)} + b
        $$
        $$
\left(\begin{array}{ccc}
       4 & 0 & 0\\
       0 & 4 & 0\\
       0 & 0 & 4
      \end{array}\right)x^{(1)}=\left(\begin{array}{ccc}
       0 & -1 & 0\\
       -1 & 0 & -1\\
       0 & -1 & 0
      \end{array}\right)\left(\begin{array}{c}
      -1\\
      4\\
      -1
      \end{array}\right)+\left(\begin{array}{c}
      -3\\
      10\\
      1
      \end{array}\right)
      $$
      Y al resolver el sistema diagonal:
$$
x^{(1)} = \left(\begin{array}{c}
      -1.75\\
      3\\
      -0.75
      \end{array}\right)
$$
Con vector residual
$$
r^{(1)} = \left(\begin{array}{c}
      1\\
      0.5\\
      1
      \end{array}\right) \Rightarrow \|r^{(1)}\|_\infty = 1
$$
      \end{frame}
      %%%%%%
      \begin{frame}{Ejemplo:}
        Por tanto, el residuo ha disminuido. Si seguimos iterando:
        \begin{eqnarray}
          \nonumber x^{(2)} = \left(\begin{array}{c}
               -1.5\\
               3.125\\
               -0.5
               \end{array}\right) \quad \|r^{(2)}\|_\infty=0.5 \\
          \nonumber x^{(3)} = \left(\begin{array}{c}
               -1.5312\\
               3.0000\\
               -0.50312
               \end{array}\right) \quad \|r^{(3)}\|_\infty=0.125\\
               \nonumber x^{(4)} = \left(\begin{array}{c}
                -1.5\\
                3.0156\\
                -0.5
                \end{array}\right) \quad \|r^{(4)}\|_\infty=0.0625 \\
           \nonumber x^{(5)} = \left(\begin{array}{c}
                -1.5039\\
                3.0000\\
                -0.5039
                \end{array}\right) \quad \|r^{(5)}\|_\infty=1.5625e-02
         \end{eqnarray}
      \end{frame}        

      \end{document}
