\documentclass[12pt,letterpaper]{article}
\usepackage[utf8]{inputenc}
\usepackage[spanish]{babel}
\usepackage{amsmath}
\usepackage{amsfonts}
\usepackage{amssymb}
\usepackage{graphicx}
\usepackage[left=2cm,right=2cm,top=0.5cm,bottom=0cm]{geometry}

\begin{document}
\pagestyle{empty}
\begin{flushleft}
Departamento de Matem\'aticas.\\
FACYT-UC.\\

\mbox{}\hfill Secci\'on:\rule{1cm}{0.4pt}\\
Nombre y Apellido: \rule{5cm}{0.4pt} \hfill
C.I: \rule{3cm}{0.4pt}
\end{flushleft}
\begin{center}
{\Large Cuarto Parcial (C\'alculo 1)}
\end{center}

\begin{enumerate}
\item Calcular los límites
\begin{enumerate}
\item $\lim_{x\to 0}  \frac{sen(3x)}{2-2cos(x)}$ (2 pts)
\item $\lim_{x\to 0} \frac{ln(x+1)}{x}$ (2 pts)
\item $\lim_{x\to -1} \frac{x+1}{x^3+1}$ (2 pts)
\end{enumerate}
\item  Hallar la f\'ormula de la recta tangente a $f(x)=sen(x)$
en el punto $(\frac{\pi}{4},sen(\frac{\pi}{4}))$. Graficar el $sen()$ y la recta tangente resultante. (3 pts)
\item Calcular las siguientes derivadas. 
\begin{enumerate}
\item $(\sqrt[\ln(x)]{-x})'$. (3 pts)
\item $[\tan\left(\arccos(\frac{1}{1+x^2})\right)]'$. (3 pts)
\item $[sen(\sqrt{x^2})+\frac{1}{\ln(x+2)}]'$. (3 pts)
\end{enumerate}
\item Calcular la derivada usando fórmulas y demuestre al simplificar que 
$$
\left(\frac{1+cos(x)}{1-sen(x)}\right)'=\frac{cos(x)-sen(x)}{(1-sen(x))^2}.\ \ \mbox{(2 pts)} $$
\end{enumerate}

\rule{2\linewidth}{0.4pt}

\begin{flushleft}
  Departamento de Matem\'aticas.\\
  FACYT-UC.\\
  
  \mbox{}\hfill Secci\'on:\rule{1cm}{0.4pt}\\
  Nombre y Apellido: \rule{5cm}{0.4pt} \hfill
  C.I: \rule{3cm}{0.4pt}
  \end{flushleft}
  \begin{center}
  {\Large Cuarto Parcial (C\'alculo 1)}
  \end{center}
  
  \begin{enumerate}
  \item Calcular los límites
  \begin{enumerate}
  \item $\lim_{x\to 0}  \frac{sen(3x)}{2-2cos(x)}$ (2 pts)
  \item $\lim_{x\to 0} \frac{ln(x+1)}{x}$ (2 pts)
  \item $\lim_{x\to -1} \frac{x+1}{x^3+1}$ (2 pts)
  \end{enumerate}
  \item  Hallar la f\'ormula de la recta tangente a $f(x)=sen(x)$
  en el punto $(\frac{\pi}{4},sen(\frac{\pi}{4}))$. Graficar el $sen()$ y la recta tangente resultante. (3 pts)
  \item Calcular las siguientes derivadas. 
  \begin{enumerate}
  \item $(\sqrt[\ln(x)]{-x})'$. (3 pts)
  \item $[\tan\left(\arccos(\frac{1}{1+x^2})\right)]'$. (3 pts)
  \item $[sen(\sqrt{x^2})+\frac{1}{\ln(x+2)}]'$. (3 pts)
  \end{enumerate}
  \item Calcular la derivada usando fórmulas y demuestre al simplificar que 
  $$
  \left(\frac{1+cos(x)}{1-sen(x)}\right)'=\frac{cos(x)-sen(x)}{(1-sen(x))^2}.\ \ \mbox{(2 pts)} $$
  \end{enumerate}
  

\end{document}