\documentclass[12pt,letterpaper]{article}
\usepackage[spanish]{babel}
\usepackage[english]{algorithm2e}

%\usepackage[dvips]{graphicx}
\usepackage{amsthm,amssymb,amsfonts,amsmath}
\usepackage[left=1.0cm,right=1.0cm,bottom=1.5cm,top=1.5cm]{geometry}
\usepackage{graphicx,color}
\usepackage{multicol}
%\usepackage[cp1252]{inputenc}
%\usepackage{pifont}
\newcommand{\sen}{\ensuremath{sen}}
\newcommand{\ds}{\displaystile}
\theoremstyle{plain}
\newtheorem{teo}{Teorema}%[section]
\newtheorem{defi}{Definici\'on}
\newtheorem{ej}[defi]{Ejemplo}
\newcommand{\I}{\mathbf{I}}
\newcommand{\II}{\mathbf{II}}

\pretolerance=10000

\tolerance=12000
%%%%%%%%%%%%
%%%%%%%%%%%%%%%%%%%%%%%%%%%
%Estos comandos controlan el interlineado.
\newlength{\defbaselineskip}
\setlength{\defbaselineskip}{\baselineskip}
\newcommand{\setlinespacing}[1]%
           {\setlength{\baselineskip}{#1 \defbaselineskip}}

%%%%%%%%%%%%%%%%%%%%%%%%%%%%%%%%%%%%%%%%%%%%
%Para que TeX no sea tan exigente al momento de saltar la l\'inea
\sloppy
\linespread{1.3} % Por defecto tiene 1, para espacio y medio (1,5 cm) se coloca 1.3 y para doble espacio 1.6.
\setlength{\parindent}{0cm}  %Sin sangr\'{\i}a entre p\'arrafos.
\setlength{\parskip}{.6cm plus .1cm minus .1cm} %Separaci\'{o}n entre p\'{a}rrafos.
%%%%%%%%%%%%%%%%%%%%%%%%%%%%%%%%%%%%%%%%%%%%%%%%
\begin{document}

\thispagestyle{empty}

%\textcolor{blue}

\begin{center}

\noindent\begin{minipage}{15.5cm}
\textbf{UNIVERSIDAD DE CARABOBO.\\
\noindent FACULTAD DE CIENCIAS Y TECNOLOG\'IA.\\
\noindent DEPARTAMENTO DE MATEM\'ATICAS.\\
\noindent Materia: M\'etodos Num\'ericos II\\
\noindent Prof. Jos\'e Luis Ram\'irez B.\\
}

\end{minipage}
\begin{minipage}[ht]{2.5cm}
\includegraphics[width=2.5cm]{LogoUC.jpg}
\end{minipage}


\begin{center} {\it\bf\large Gu\'ia de Ejercicios I}
\end{center}
%\today\ \\
\end{center}

\begin{enumerate}
 \item ``Resuelva'' los siguientes sistemas de ecuaciones lineales usando Eliminaci\'on Gaussiana con sustituci\'on hacia atr\'as con una aritm\'etica redondeo correcto a dos d\'igitos. No reordene las ecuaciones.(La soluci\'on exacta para cada sistema es $x_1=1\quad,x_2=-1,\quad x_3=3$).
\begin{multicols}{2}
\begin{enumerate}
 \item $\left\{\begin{array}{lcc}
         4x_1-x_2+x_3&=&8,\\
         2x_1+5x_2+2x_3&=&3,\\
         x_1+2x_2+4x_3&=&11
        \end{array}\right.$

 \item $\left\{\begin{array}{ccc}
         4x_1+x_2+2x_3&=&9,\\
         2x_1+4x_2-x_3&=&-5,\\
         x_1+x_2-3x_3&=&-9
        \end{array}\right.$


 \item $\left\{\begin{array}{ccc}
         x_1+2x_2+4x_3&=&11,\\
         4x_1-x_2+x_3&=&8,\\
         2x_1+5x_2+2x_3&=&3
        \end{array}\right.$

 \item $\left\{\begin{array}{ccc}
         2x_1+4x_2-x_3&=&-5,\\
         x_1+x_2-3x_3&=&-9,\\
         4x_1+x_2+2x_3&=&9
        \end{array}\right.$
\end{enumerate}
\end{multicols}

\item Use el algoritmo de Eliminaci\'on Gaussiana y, si es posible, aritm\'etica de precisi\'on simple en una computadora, para resolver los siguientes sistemas lineales:
\begin{multicols}{2}
\begin{enumerate}
 \item $\left\{\begin{array}{l}
         \frac{1}{4}x_1+\frac{1}{5}x_2+\frac{1}{6}x_3 = 9,\\
         \frac{1}{3}x_1+\frac{1}{4}x_2+\frac{1}{5}x_3 = 8,\\
         \frac{1}{2}x_1+x_2+2x_2 = 8.\\
        \end{array}\right.$

\item $\left\{\begin{array}{ccc}
        3.333x_1+15920x_2-10.333x_3&=&15913\\
        2.222x_1+16.71x_2+9.612x_3&=&28.544\\
        1.5611x_1+5.1791x_2+1.6852x_3&=&8.4254
       \end{array}\right.$

\item $\left\{\begin{array}{ccc}
        x_1+\frac{1}{2}x_2+\frac{1}{3}x_3+\frac{1}{4}x_4&=&\frac{1}{6}\\
        \frac{1}{2}x_1+\frac{1}{3}x_2+\frac{1}{4}x_3+\frac{1}{5}x_4&=&\frac{1}{7}\\
        \frac{1}{3}x_1+\frac{1}{4}x_2+\frac{1}{5}x_3+\frac{1}{6}x_4&=&\frac{1}{8}\\
        \frac{1}{4}x_1+\frac{1}{5}x_2+\frac{1}{6}x_3+\frac{1}{7}x_4&=&\frac{1}{9}\\
       \end{array}\right.$
\end{enumerate}
\end{multicols}
\item Dado el sistema lineal

$$
\left\{\begin{array}{ccc}
 x_1-x_2+\alpha x_3&=&-2\\
 -x_1+2x_2-\alpha x_3&=&3\\
 \alpha x_1+x_2+x_3&=&2\\
\end{array}\right.
$$

\begin{enumerate}
 \item Obtenga el (los) valor(es) de $\alpha$ para el(los) cual(es) el sistema no tiene soluci\'on.
 \item Encuentre el (los) valor(es) de $\alpha$ para el(los) cual(es) el sistema tiene un infinito n\'umero de soluciones.
 \item Considerando que existe una soluci\'on \'unica para un valor dado $\alpha$, encuentre la soluci\'on.
\end{enumerate}

\item Factorice las siguientes matrices en la descomposici\'on $LU$ usando el Algoritmo de Factorizaci\'on con $l_{ii}=1$ para toda $i$:
\begin{multicols}{2}
\begin{enumerate}
 \item $\left[ \begin{array}{ccc}
         2&-1&1\\
         3&3&9\\
         3&3&5
        \end{array}\right]
$

\item $\left[ \begin{array}{ccc}
         2&-1.5&3\\
         -1&0&2\\
         4&-4.5&5
        \end{array}\right]
$

\item $\left[ \begin{array}{ccc}
         1.012&-2.132&3.104\\
         -2.132&4.096&-7.013\\
         3.104&-7.013&0.014
        \end{array}\right]
$

\item $\left[ \begin{array}{cccc}
         2&0&0&0\\
         1&1.5&0&0\\
         0&-3&0.5&0\\
         2&-2&1&1
        \end{array}\right]
$
\end{enumerate}
\end{multicols}

\item Sean $\{1,-2,3\},\{1,-1\},\{-1\}$ los multiplicadores de la Eliminaci\'on Gaussiana para obtener el sistema triangular
    $$
    \left\{\begin{array}{rrr}
    x_1+2x_2-x_3+x_4 & = & 1\\
    2x_2+x_3+ 3x_4 & = & -2\\
    x_3-x_4 & = & 0\\
    2x_4 &=&7\\
    \end{array}\right.
    $$
    Calcule el sistema original.

\item Calcule los n\'umeros de productos y sumas para factorizar una matriz y para resolver el sistema triangular.

\item Resuelva el sistema $Ax=b$ con
$$
A=\left( \begin{array}{cccc}
         4 & -2 & 8 & 4\\
        -2 & 10 & -4 & 5\\
         8 & -4 & 17 & 9\\
         4 & -5 & 9 & 15\\
        \end{array}\right), \quad
b=\left(\begin{array}{c}
0\\
3\\
-1\\
-11\\
\end{array}\right)
$$

\item Considere el sistema de orden $n$ definido por la igualdad
$$
A=\left( \begin{array}{ccccc}
         2 & -1 &  &  &\\
        -1 & 2 & -1 &  &\\
           & -1 & 2 & -1 &\\
           &   & \ddots & \ddots & \ddots\\
           &   &  & -1 & 2\\
        \end{array}\right)\left(\begin{array}{c}
x_1\\
x_2\\
x_3\\
\vdots\\
x_n\\
\end{array}\right) = \left(\begin{array}{c}
0\\
0\\
0\\
\vdots\\
1\\
\end{array}\right)
$$

Muestre que el vector $x=\frac{1}{n+1}(1,2,\ldots,n)$ es la \'unica soluci\'on del sistema anterior.

\item Sea
$$
A=\left( \begin{array}{ccc}
         0 & -1 & 0 \\
         0 & 0 & 1 \\
         2 & 3 & 1 \\
        \end{array}\right)
$$
  \begin{itemize}
   \item Hallar una factorizaci\'on $A = QR$. Una vez hallada, explicar qu\'e relaci\'on hay (en este ejemplo) entre los factores $QR$ y los de una factorizaci\'on $LU$ con pivoteo.
  \item Probar que el producto $B = RQ$ tiene siempre los mismos autovalores que $A = QR$. (Naturalmente, si $Q$ es ortogonal).
  \end{itemize}

\item Triangularizar la matriz de Hilbert de orden 4. Usando operaciones con fracciones en forma exacta en (a) y usando aritm\'etica de punto decimal flotante con tres d\'igitos con redondeo en (b):
\begin{multicols}{2}
\begin{enumerate}
 \item $$
H=\left( \begin{array}{cccc}
         1   & 1/2 & 1/3 & 1/4\\
         1/2 & 1/3 & 1/4 & 1/5\\
         1/3 & 1/4 & 1/5 & 1/6\\
	 1/4 & 1/5 & 1/6 & 1/7\\
        \end{array}\right)
$$
\item $$
H=\left( \begin{array}{cccc}
         1.000   & 0.500 & 0.333 & 0.250\\
         0.500 & 0.333 & 0.250 & 0.200\\
         0.333 & 0.250 & 0.200 & 0.167\\
	 0.250 & 0.200 & 0.167 & 0.143\\
        \end{array}\right)
$$
 \end{enumerate}
\end{multicols}
\item Calcular la inversa $A^{-1}$ de $A=
\left( \begin{array}{ccc}
         2   & 1 & 2 \\
         1 & 2 & 3\\
         4 & 1 & 2\\
        \end{array}\right)
$ de las dos maneras siguientes:
\begin{enumerate}
\item Resolviendo el sistema matricial $AX = I$ por pivoteo parcial.
\item Calculando la factorizaci\'on $LU$ de $A$ y aplicando la identidad $A^{-1} = U ^{-1}L^{-1}$

\end{enumerate}
\item Determine los valores de $\alpha$ para los que la matriz
$$
A=\left( \begin{array}{ccc}
         2   & \alpha & 0\\
         \alpha & 2 & \alpha\\
         0 & \alpha & 2\\
        \end{array}\right)
$$
admite factorizaci\'on de Cholesky y calc\'ulela en su caso.

\item Sea $A$ una matriz no singular de $\mathbb{R}^{n\times n}$ escrita en forma de bloques
$$
A=\left( \begin{array}{cc}
         A_{11}   & A_{12} \\
         A_{21} & A_{22} \\
        \end{array}\right)
$$
donde $A_{11}$ es una matriz de tama\~no $m\times m$ y $A_{22}$ es de tama\~no $(n-m)\times(n-m)$.
\begin{enumerate}
\item Verificar la f\'ormula
$$
\left( \begin{array}{cc}
         I   & 0 \\
         -A_{21}A_{11}^{-1} & I \\
        \end{array}\right)\left( \begin{array}{cc}
         A_{11}   & A_{12} \\
         A_{21} & A_{22} \\
        \end{array}\right)=\left( \begin{array}{cc}
         A_{11}   & A_{12} \\
         0 & A_{22} - A_{21}A_{11}^{-1}A_{12} \\
        \end{array}\right)
$$
para la eliminaci\'on del bloque $A_{21}$ (la matriz $A_{22} - A_{21}A_{11}^{-1}A_{12}$ es conocida como complemento de Schur de $A_{11}$ en $A$).
\item Si $A_{21}$ se elimina fila a fila en $m$ pasos de eliminaci\'on Gaussiana, mostrar que el bloque $(2,2)$ de la matriz resultante de aplicar el proceso de eliminaci\'on Gaussiana es igual a $A_{22} - A_{21}A_{11}^{-1}A_{12}$.

\end{enumerate}

\item Calcular la factorizaci\'on $LU$ de la matriz
$$
A=\left( \begin{array}{cccc}
         10 & 6 & -2 & 1\\
         10 & 10 & -5 & 0\\
         -2 & 2 & -2 & 1\\
	      1 & 3 & -2 & 3\\
        \end{array}\right)
$$
donde los elementos de la diagonal de $L$ son uno. Use la factorizaci\'on para hallar la soluci\'on del sistema $Ax=b$ donde $b^t=(-2,0,2,1)$.

\item Calcular la factorizaci\'on de Cholesky de la matriz
$$
\left( \begin{array}{cccccc}
         1 & 1 &   &   &   & \\
         1 & 2 & 1 &   &   & \\
           & 1 & 3 & 1 &   & \\
	       &   & 1 & 4 & 1 & \\
	       &   &   & 1 & 5 & 1\\
	       &   &   &   & 1 & \lambda\\
        \end{array}\right)
$$
Deducir de la factorizaci\'on el valor de $\lambda$ que hace a la matriz singular.

\item Analizar que sucede cuando se aplica Eliminaci\'on Gaussiana con Pivoteo Parcial a la siguiente matriz
$$
A=\left( \begin{array}{ccccc}
          1 &  0  &  0 &  0 & 1 \\
         -1 &  1  &  0 &  0 & 1 \\
         -1 & -1  &  1 &  0 & 1 \\
	     -1 & -1  & -1 &  1 & 1 \\
	     -1 & -1  & -1 & -1 & 1 \\
        \end{array}\right)
$$    

\item M\'etodo de Gauss-Jordan: Este m\'etodo se puede describir como sigue. Use la $i$-\'esima
ecuaci\'on para eliminar no solo los $x_i$ en las ecuaciones $E_{i+1},E_{i+2},\ldots,E_n$, como se hizo en el
m\'etodo de eliminaci\'on de Gauss, sino tambi\'en de $E_1,E_2,\ldots,E_{i-1}$. Reduciendo $[A,b]$ a:
$$
\left( \begin{array}{cccc|c}
          a_{1,1}^{(1)} &  0  &  \cdots &  0 & a_{1,n+1}^{(1)} \\
          0 &  a_{2,2}^{(2)}  &  \ddots &  \vdots & a_{2,n+1}^{(2)} \\
         \vdots & \ddots  &  \ddots &  0 & \vdots \\
	     0 & \cdots  & 0 &  a_{n,n}^{(n)} & a_{n,n+1}^{(n)} \\
        \end{array}\right)
$$    
la soluci\'on se obtiene tomando $x_i=\frac{a_{i,n+1}^{(i)}}{a_{i,i}^({(i)})}$ para $i=1,2,\ldots,n$. Este procedimiento evita la necesidad de la sustituci\'on hacia atr\'as en la Eliminaci\'on Gaussiana. Construya un algoritmo para el procedimiento de Gauss-Jordan siguiendo el patr\'on del algoritmo de Eliminaci\'on de Gauss sin pivoteo.

\item Realice el conteo de operaciones del m\'etodo de Gauss-Jordan.

\item Utilice el m\'etodo de Gauss-Jordan para resolver los ejercicios (1), (2) y (7).

\item Si $A$ es sim\'etrica y definida positiva:
  \begin{enumerate}
  \item Escribir el algoritmo de la factorizaci\'on $LDL^T$.
  \item Hacer el conteo de operaciones de $(a)$.
  \end{enumerate}

\item \begin{enumerate}
      \item Demuestre que el producto de dos matrices triangulares inferiores $n\times n$, es una matriz triangular inferior.
      \item Demuestre que el producto de dos matrices triangulares superiores $n\times n$, es una matriz triangular superior.
      \item Demuestre que la inversa de una matriz triangular inferior no singular $n\times n$ es triangular inferior.
      \end{enumerate}

\item  Suponga que $m$ sistemas lineales $Ax=b$, $p=1,\ldots,m$, han de resolverse, cada uno con la matriz de coeficientes $A$ de $n\times n$.
  \begin{enumerate}
    \item Hallar el n\'umero de operaciones que requiere la eliminaci\'on gaussiana con sustituci\'on hacia atr\'as aplicada a la matriz ampliada $[A : b^1 b^2 \ldots b^m ]$.
   \item Hallar el n\'umero de operaciones que requiere el m\'etodo de Gauss-Jordan aplicado a la matriz ampliada $[A : b^1 b^2 \ldots b^m ]$.

  \end{enumerate}

\item Una matriz $Q \in \mathbb{R}^{n \times n}$ se denomina ortogonal si $QQ^t=Q^tQ=I$, es decir, $Q^{-1}=Q^t$. Considere el sistema lineal $Ax=b$, con $A \in \mathbb{R}^{n \times m}$, $x \in \mathbb{R}^m$,  y $b \in \mathbb{R}^n$. Suponga que $A$ se puede factorizar como $QR$, es decir, $A=QR$, donde $Q$ es una matriz ortogonal y $R$ es una matriz triangular superior.
\begin{enumerate}
 \item Explique como se puede usar la factorizaci\'on $QR$ de $A$, para resolver el Sistemas de Ecuaciones Normales (SEN) asociado al sistema general $Ax=b$.
 \item Considere el sistema $Ax=b$
$$
A=\left(\begin{array}{cc}
      5 & 6 \\
      4 & 8 \\
      4 & 8  
      \end{array}\right), \qquad  b=\left(\begin{array}{c}
      22 \\
      24 \\
      24  
      \end{array}\right)
$$
\item Use la factorizaci\'on $QR$ de $A$, para hallar la soluci\'on del problema dado. Escriba las matrices, el sistema y la soluci\'on.
\end{enumerate}

\item La torre de Pisa se inclina m\'as a medida que pasa el tiempo. He aqu\'i las mediciones de la inclinaci\'on de la torre entre los a\~nos 75 y 87. La inclinaci\'on se da como la distancia entre el punto donde estar\'ia la torre en posici\'on vertical y el punto en el que realmente se encuentra. Las distancias se dan en d\'ecimas de mil\'imetros por encima de 2.9m.

\begin{tabular}{|l|c|c|c|c|c|c|c|c|c|c|c|c|c|}\hline
 A\~no & 75 & 76 & 77 & 78 & 79 & 80 & 81 & 82 & 83 & 84 & 85 & 86 & 87\\\hline
 Inclinaci\'on & 642 & 644 & 656 & 667 & 673 & 688 & 696 & 698 & 713 & 717 & 725 & 742 & 757\\\hline
\end{tabular}

\begin{enumerate}
 \item Muestre la matriz $A$ y el vector $b$ que resulta de ajustar los datos mediante una recta.
\item Determine el rango de la matriz $A$ y el rango de la matriz aumentasda $(A|b)$. Tiene soluci\'on el sistema?
\item Calcule el n\'umero de condici\'on de $A$ y $A^tA$. Que relaci\'on observa entre estos valores?
\item Construya los elementos del sistema de ecuaciones normales para la matriz $A$ dada y el vector $b$.
\item Resuelva el sistema anterior utilizando factorizaci\'on $QR$ y factorizaci\'on  de Cholesky.
\end{enumerate}

\item Usando primero el m\'etodo de Householder, y luego el m\'etodo de Givens, encuentre la factorizaci\'on $QR$ de la matriz:
$$
A=\left(\begin{array}{cc}
      1 & -8 \\
      2 & -1 \\
      2 & 14  
      \end{array}\right)
$$

\item Dado el siguiente sistema, muestre que la matriz asociada es positiva definida. Resuelva por medio de la factorizaci\'on de Cholesky y muestre la factorizaci\'on $ \tilde L D \tilde L^t$
$$
A=\left(\begin{array}{cccc}
      4 & -2 & 8 & 4 \\
      -2 & 10 & -4 & -5 \\
      8 & -4 & 17 & 9\\
      4 & -5 & 9 & 15
      \end{array}\right), \qquad b=\left(\begin{array}{c}
      0 \\
      3\\
      -1 \\
      -11 
      \end{array}\right)
$$

\item Resolver, usando el m\'etodo de eliminaci\'on gaussiana cl\'asica y sin pivoteo parcial, el siguiente sistema
$$
\left(\begin{array}{ccc}
      0.0001 & 1 & 1  \\
      3 & 1 & 1  \\
      1 & 2 & 3 
      \end{array}\right)\left(\begin{array}{c}
      x_1 \\
      x_2\\
      x_3 
      \end{array}\right)=\left(\begin{array}{c}
      2.0001 \\
      3\\
      3 
      \end{array}\right)
$$
compare los resultados obtenidos.

\item Resolver, usando el m\'etodo de eliminaci\'on gaussiana con pivoteo total y con factorizaci\'on $QR$ (Gram-Schmidt, Householder, Givens) el siguiente sistema
$$
\left(\begin{array}{ccc}
      0 & 1 & 1  \\
      2 & 2 & 3  \\
      4 & 1 & 1 
      \end{array}\right)\left(\begin{array}{c}
      x_1 \\
      x_2\\
      x_3 
      \end{array}\right)=\left(\begin{array}{c}
      2 \\
      6\\
      3 
      \end{array}\right)
$$

\item Estudiar el condicionamiento del sistema $Ax=b$ con $A=\left(\begin{array}{cc}
                                                                    1 & 1+\varepsilon\\
								    1-\varepsilon & 1
                                                                   \end{array}
\right)$ siendo $\varepsilon>0$ en la norma $\|\cdot\|_{\infty}$.

\item Repetir el ejercicio anterior con la norma $\|\cdot\|_1$.

\item Dado el sistema lineal
$$
P=\left\{\begin{array}{lll}
          0.1x + y & = & b_1\\
	  0.1x + 1.5y & = & b_2
         \end{array}\right.
$$
\begin{enumerate}
\item Calcule el n\'umero de condici\'on en norma infinito.
\item Sustituya $x$ por $x/\alpha$ y calcule el n\'umero de condici\'on de la nueva matriz. ?`Qu\'e relaci\'on tiene con
la original? ?`Cu\'al $\alpha$ minimiza el n\'umero de condici\'on?
\end{enumerate}

\item Dado el sistema lineal
$$
\left(\begin{array}{cc}
       \varepsilon & 1\\
      1 & 1
      \end{array}
\right)\left(\begin{array}{c}
              x_1\\
	      x_2
             \end{array}
\right)=\left(\begin{array}{c}
               1\\
	      2
              \end{array}
\right)
$$
con $\varepsilon > 0$, peque\~no pero no nulo para el computador, ?`qu\'e resultado num\'erico se obtiene en el computador si se aplica el m\'etodo de Gauss sin pivoteo?, ?`dicho resultado es la soluci\'on correcta del sistema lineal? Razone su respuesta.

\item Considere la siguiente matriz $A=\left(\begin{array}{cc}
                                              d & a\\
						a & d
                                             \end{array}
\right)$, donde $0<a<d$.
\begin{enumerate}
 \item Encuentre el factor $L$ de Cholesky.
\item Consiga el vector $x$ tal que $Ax = b$ cuando $b = (3, 2)^t$ , $d = 9$ y $a = 6$.
\end{enumerate}

\item Sea $A \in \mathbb{R}^{n\times n}$ una matriz triangular inferior con unos en la diagonal, elementos $a_{i,1} = \alpha_i , i = 2,\ldots,n$ y ceros en el resto de las posiciones. Por ejemplo, si $n = 4$ entonces
$$
A=\left(\begin{array}{cccc}
       1 & 0 & 0 & 0\\
      \alpha_2 & 1 & 0 & 0\\
      \alpha_3 & 0 & 1 & 0\\
      \alpha_4 & 0 & 0 & 1\\
      \end{array}\right)
$$
\begin{enumerate}
 \item Calcule el n\'umero de condici\'on en la norma infinito de $A$, $\kappa_\infty(A)$ para cualquier $n$.
\item Sean $b^t = (1, 1,\ldots, 1)$ y $b^t = (1.00005, 1,\ldots, 1)$. Suponga que $Ax = b$ y $A\tilde{x} = \tilde{b}$. Use $\kappa_\infty(A)$ para obtener una cota del error relativo en $\tilde{x}$, como aproximaci\'on de $x$, cuando $\max|\alpha_i| = 0,01$ y cuando $\max|\alpha_i| = 100$.
\end{enumerate}

\item Considere las matrices
$$
P=\left(\begin{array}{ccc}
      0 & 1 & 0  \\
      1 & 0 & 0  \\
      0 & 0 & 1 
      \end{array}\right), \qquad D=\left(\begin{array}{ccc}
      3 & 0 & 0  \\
      0 & 2 & 0  \\
      0 & 0 & -5 
      \end{array}\right), \qquad A=\left(\begin{array}{ccc}
      2 & -1 & 2  \\
      4 & -3 & 1  \\
      3 & -4 & 1 
      \end{array}\right)
$$
Calcular los productos $PA$, $AP$ , $DA$ y $AD$. ?`Qu\'e efecto produce premultiplicar y postmultiplicar una matriz $A$ por una matriz de permutación?, ?`Y por una matriz diagonal?

\item Si $A \in \mathbb{R}^{n\times n}$ es una matriz Hessenberg superior, diga qu\'e hace el siguiente seudoc\'odigo y calcule la cantidad total de operaciones de punto flotante (sumas, restas, multiplicaciones y divisiones) que \'este efect\'ua.

%\incmargin{1em}
%\restylealgo{boxed}
%\linesnumbered
\begin{algorithm}[H]
%\SetLine
 \SetKwInOut{Input}{input}
 \Input{$A \in \mathbb{R}^{n \times n}$}
 \BlankLine
 \For{$k\leftarrow 1$ \KwTo $n-1$}
 {
      $\alpha \leftarrow \displaystyle\frac{a_{k+1,k}}{a_{k,k}}$\\
      \For{$j \leftarrow k$ \KwTo $n$}
      {
	$a_{k+1,j} \leftarrow a_{k+1,j}-\alpha*a_{k,j}$\\
      }	
 }
\end{algorithm}
%\decmargin{1em}

\item Si se tiene una descomposici\'on $LU$ de una matriz $A \in \mathbb{R}^{n\times n}$, donde $L \in \mathbb{R}^{n\times n}$ es una matriz triangular inferior y $U \in \mathbb{R}^{n\times n}$ es una matriz triangular superior, proponga un seudoc\'odigo para resolver el sistema lineal $x^tA = b^t$.

\item Sea $A=\left(\begin{array}{ccccc}
                    90 & 1 & 2 & 3 & 4\\
1 & 90 & 2 & 3 & 4\\
 1 & 2 & 90 & 3 & 4\\
 1 & 2 & 3 & 90 & 4\\
 1 & 2 & 3 & 4 & 90\\
                   \end{array}
\right)$, $b=\left(\begin{array}{c}
                     1 \\
 2 \\
 3 \\
 4\\
 5\\
                   \end{array}
\right)$, $c= \left(\begin{array}{ccccc}
                    1 & 2 & 3 & 4 & 5
                   \end{array}
\right)$, $d=\left(\begin{array}{c}
                     11 \\
 -3 \\
 8 \\
 7\\
 9\\
                   \end{array}
\right)$

Resuelva los sistemas $Ax = b$ , $y^tA = c$ y $Ax = d$ usando una sola factorizaci\'on $LU$.

\end{enumerate}
\end{document}
