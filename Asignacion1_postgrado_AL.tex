\documentclass[12pt]{article}
\usepackage[spanish]{babel}
\usepackage{amsfonts,amssymb,amsthm,amsmath,color,graphicx,bm,fancyhdr}
\usepackage[left=1cm,right=1cm,bottom=1cm,top=1cm]{geometry}
\pretolerance=2000 \tolerance=3000 \linespread{1.0}
\pagestyle{fancy}
\newcommand{\er}{\ensuremath{I\!\!R}}
\newcommand{\sen}{\ensure{sen}}

\pagestyle{empty}
\begin{document}

%\textcolor{blue}

\begin{center}

\begin{figure}[!h]
\includegraphics[scale=0.2]{LogoUC.jpg}
  \label{fig:hoo}
   \hspace{14cm}\includegraphics[scale=0.6]{LogoFaCyT.jpg}
\end{figure}
 \vspace{-2,5cm}
Universidad de Carabobo \\ Facultad Experimental de Ciencias y
Tecnolog\'{\i}a \\ Departamento de Matem\'aticas\\ Asignatura:
M\'etodos Num\'ericos II|\\
\vspace{1cm} {\it\bf\large Tarea Computacional I}\\
\end{center}

\begin{enumerate}

\item Implementaci\'on de la Factorizaci\'on $LU$

Implementar cada una de los siguientes algoritmos del \'Algebra Lineal en MATLAB empleando solo 
comandos b\'asicos (no emplear las funciones intr\'insecas que posee MATLAB). Debe usar las 
cabeceras de funci\'on dada para cada una.

\begin{enumerate}
   \item Factorizaci\'on $LU$ sin pivoteo, sin embargo el algoritmo debe verificar por pivote cero 
y finalizar emitiendo el respectivo mensaje de error. El c\'odigo debe retornar las matrices $L$ y 
$U$ como una sola matriz con la idea de ahorrar espacio en memoria.

\begin{tt}
function LU = LUFactorization(A)\\
\% Uso: LU = LUFactorization(A)\\
\% Computa la factorizaci\'on LU de la matriz A\\
\% y retorna la matriz factorizada
\end{tt}

\item Sustituci\'on Progresiva. Esta rutina debe tomar como dato de entrada la matriz resultante de 
la Factorizaci\'on $LU$ de la parte ({\it a})

\begin{tt}
   function y = ForwardSubstitution(LU, b)\\
\% Uso: y = ForwardSubstitution(LU, b)\\
\% Lleva a cabo substituci\'on progresiva usando la parte\\ 
\% triangular inferior unitaria de la matriz LU.
\end{tt}

\item Sustituci\'on Regresiva. Esta rutina debe tomar como dato de entrada la matriz resultante de 
la Factorizaci\'on $LU$ de la parte ({\it a})

\begin{tt}
function x = BackwardSubstitution(LU, y)
\% Uso: y = ForwardSubstitution(LU, b)\\
\% Lleva a cabo substituci\'on regresiva usando la parte\\ 
\% triangular superior de la matriz LU.
\end{tt}

\item Use el archivo {\tt Verify.m} para probar sus c\'odigos. Seleccione un entero $n$ entre 100 y 
200, ejecute {\tt VerifyLU(n)}  e indique para $n$ el error obtenido.

\end{enumerate}

\item Tridiagonal $LU$

Considere la siguiente matriz tridiagonal
$$
A_t = \left(\begin{array}{ccccccc}
               b_1 & c_1 & 0 & & \cdots & 0 & 0 \\
               a_2 & b_2 & c_2 & & & 0 & 0 \\
               0 & a_3 & b_2 & c_2 & & 0 & 0 \\
                &  & \ddots &\ddots &\ddots &  &  \\
                 & &  & \ddots &\ddots &\ddots &   \\
                0 & 0 & 0 & & a_{n-1} & b_{n-1} & c_{n-1} \\
                0 & 0 & 0 & &  & a_{n} & b_{n} 
            \end{array}\right)_{n \times n}
$$
Tales sistemas pueden encontrarse en distintas aplicaciones como por ejemplo Interpolaci\'on por 
Splines. Asumiendo que no hay necesidad de pivoteo, entonces los factores de $A_t=LU$ tienen la 
forma
$$
L = \left(\begin{array}{cccccc}
             1 & 0 &  & \cdots & 0 & 0 \\
             l_2 & 1 & 	0 &  & 0 & 0\\
             0 & l_3 & 1 &  & & 0\\
             & & \ddots & \ddots & & \\
             0 & 0  & & & 1 & 0\\
             0 & 0 & & & l_n & 1             
          \end{array}\right) \qquad
U = \left(\begin{array}{cccccc}
             d_1 & u_1 &  & \cdots & 0 & 0 \\
             0 & d_2 & 	u_2 &  &  & 0\\
             0 & 0 & d_3 & u_3 & & \\
             & &  & \ddots & \ddots &  \\
             0 & 0  & & & d_{n-1} & u_{n-1}\\
             0 & 0 & & & 0 & d_n             
          \end{array}\right)
$$
\begin{enumerate}
   \item Derive la relaci\'on de recurrencia para $l_i$, $d_i$ y $u_i$ en t\'erminos de $a_i$, 
$b_i$ y $c_i$. Explique su trabajo.


\item Implemente la rutina que permita resolver $A_tx=b$ usando la descomposici\'on $LU$ derivada 
en su trabajo. La rutina debe tomar como argumentos 4 arreglos unidimensionales, tres 
correspondientes a las diagonales de $A_t$ y el otro correspondiente al vector de t\'erminos 
independientes.
\begin{tt}
   function x = TriDiagonalSolve(a, b, c, k)\\
\% Uso: x = TriDiagonalSolve(a, b, c, k)\\
\% Resuelve el sistema Ax = k, donde\\
\% A es una matriz tridiagonal. Los vectores columna\\
\% a, b, c describen las entradas  de lasdiagonales de\\ 
\% la matriz. a(1) y c(n) son ignorados.
\end{tt}

\item ?`C\'ual es la complejidad computacional de su algoritmo?, Explique.

\item Use el archivo {\tt VerifyTridiagonalLU.m} para probar sus c\'odigos. Seleccione un entero 
$n$ entre 1000 y 2000, ejecute {\tt VerifyTridiagonalLU(n)}  e indique para $n$ el error obtenido.
\end{enumerate}

\item Sistemas Lineales para Ecuaciones en Diferencia. 

Sistemas de ecuaciones lineales frecuentemente se generan ecuaciones diferenciales en 
computaci\'on. Considere el siguiente problema de ecuaciones diferenciales:
\begin{eqnarray}
   u'' = f(t)
   u(0) = 0
   u(1) = 0
\end{eqnarray}

con 
$$
f(t) = 16\pi\cos(8\pi t^2)-256\pi^2t^2\sin(8\pi t^2), \qquad t \in [0,1]
$$

La soluci\'on exacta a este problema est\'a dada por $u(t) = \sin(8\pi t^2)$. Aproximando la 
soluci\'on exacta $u$ a este sistema en un n\'umero discreto de puntos, se puede construir una 
aproximaci\'on $\{v_i\}_{i=0}^N$ tal que $v_0=u_0=0$ cuando $t_0=0$ y $v_N=u_N=0$ cuando $t_N=1$. 
Aplicando una aproximaci\'on en Diferencia Finitas dada por:
$$
u_i'' \approx \dfrac{v_{i-1}-2v_i+v_{i+1}}{h^2}
$$
donde $h$ representa el tama\~no de paso ($h=1/N$). Se define el tiempo en cada punto como $t_i = 
i/N$. Con esta aproximaci\'on se genera un sistema tridiagonal $(N-1)\times(N-1)$ de la forma:
$$
\left(\begin{array}{cccccc}
         -2 & 1 & 0 & \cdots & 0 & 0\\
         1 & -2 & 1 & \cdots & 0 & 0\\
         0 & 1 & -2 & \cdots & 0 & 0\\
         \vdots & \vdots & \vdots & \ddots & \vdots& \vdots \\
         0 & 0 & 0 & \cdots & -2 & 1\\
         0 & 0 & 0 & \cdots & 1 & -2
      \end{array}\right)\left(\begin{array}{c}
                                 v_1\\
                                 v_2\\
                                 v_3\\
                                 \vdots\\
                                 v_{N-2}\\
                                 v_{N-1}
                              \end{array}\right) = h^2\left(\begin{array}{c}
                                 f(t_1)\\
                                 f(t_2)\\
                                 f(t_3)\\
                                 \vdots\\
                                 f(t_{N-2})\\
                                 f(t_{N-1})
                              \end{array}\right)-\left(\begin{array}{c}
                                 v_0\\
                                 0\\
                                 0\\
                                 \vdots\\
                                 0\\
                                 v_{N}
                              \end{array}\right)
$$
donde $v_0$ y $v_N$ son definidos en el problema. Sea $n=N-1$ el \'umero de inc\'ognitas en el 
sistema lineal
\begin{enumerate}
   \item Resolver este sistema usando la descomposici\'on $LU$ y Sustituci\'on progresiva y 
regresiva con $n=10, 100, 200, 400,800$. Determine el cantidad de tiempo computacional requerido 
para cada $n$ (Use {\tt tic} y {\tt toc} para este prop\'osito y {\tt format long} para ver la 
cantidad de d\'igitos suficientes). Compare los resultados en cada paso con la soluci\'on exacta en 
la misma gr\'afica tanto para $n=10$ y $n=400$. Explique que observa.

\item  Resolver este sistema usando la descomposici\'on $LU$ para sistemas tridiagonales con $n=10, 
100, 200, 400, 800$. Determine el cantidad de tiempo computacional requerido 
para cada $n$. Compare los resultados en cada paso con la soluci\'on exacta en 
la misma gr\'afica tanto para $n=10$ y $n=400$.

\item Compare los resultados obtenidos en los dos items anteriores. ?`C\'ual algoritmo provee una 
soluci\'on del sistema m\'as r\'apido?. Muestre una tabla comparativa de los tiempos de ejecuci\'on 
para cada valor de $n$. Los resultados coinciden con la complejidad computacional de cada 
algoritmo? 
\end{enumerate}
\item Sea $A$ la matriz definida por los elementos $a_{i,j}$ tales que
$$
a_{i,j} = \frac{1}{i+j-1}
$$

\begin{enumerate}
 \item Elabore en matlab una funci\'on que tenga como par\'ametro de entrada un valor $n$ y como par\'ametro de salida la matriz $A$.
 \item En un \textit{script} defina un ciclo con el contador $k = 2 : 25$, y genere la matriz de orden $k \times k$ seg\'un la rutina
anterior y calcule su factorizaci\'on $QR$. De ser el resultado totalmente preciso, se deber\'ia tener que $QQ^t =I_k$. Para medir
la precisi\'on del algoritmo, calcule el valor $err(k)= \|I_k - QQ^t \|$. Al salir del ciclo debe tener todos los valores del error
almacenados en el vector $err$.
\item Aplique el item anterior obteniendo la factorizaci\'on $QR$ mediante Gram-Schmidt, Householder y Givens. Obteniendo de esta manera
tres vectores de errores (uno por cada m\'etodo).
\item Refleje gr\'aficamente los resultados obtenidos, para mayor claridad agregue una leyenda al gr\'afico e interprete los resultados
obtenidos.
\end{enumerate}

\end{enumerate}

\end{document}

