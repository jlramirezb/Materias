\documentclass[12pt,letterpaper]{article}
\usepackage{amsmath}
\usepackage{amsfonts}
\usepackage{amssymb}
%\usepackage[spanish]{babel}
\usepackage[utf8]{inputenc}
\usepackage{geometry}
\geometry{tmargin=1.5cm, lmargin=1.5cm, rmargin=1.5cm, bmargin=1.5cm}
\usepackage{pgfplots}
\pgfplotsset{compat=1.18}
\usepackage{tikz}
\usetikzlibrary{intersections, backgrounds, patterns, calc, decorations.pathreplacing}
\usepgfplotslibrary{fillbetween}
\usepackage{multicol}


\begin{document}
\begin{center}
    {\Large \bf Práctica III. Cálculo III}
\end{center}

\begin{enumerate}
  \item Si $V_1 = (1,-1)$, $V_2 = (2,-1)$, $V_3=(-3,2)$ y $W_1=(1,0)$, $W_2=(0,-1)$, $W_3=(1,1)$. ¿Existe una transformación lineal $T: \mathbb{R}^2\to\mathbb{R}^2$, tal que $T(V_i) = W_i$ para $i=1,2,3$?
  \item Determinar cuáles de las siguientes aplicaciones son lineales:
  \begin{enumerate}
    \item $f : \mathbb{R}^3\to\mathbb{R}^2$ definida por $f((x; y; z)) = (x - y; y + 2z)$.
    \item $f : \mathbb{R}^3\to\mathbb{R}^2$ definida por $f((x; y; z)) = (x - y^2; y + 2z)$.
  \end{enumerate}
  \item Sean $\mathcal{P}_2$ el espacio vectorial real de los polinomios de grado menor o igual a dos con coeficientes reales y la transformación $F : \mathbb{R}^3\to\mathcal{P}_2$ definida por:\\
  $F(a,b,c) = (a + b)v_1 - cv_2$; donde $v_1 = x^2+1; v_2= 3x-1 \in \mathcal{P}_2 \quad\forall a,b,c \in \mathbb{R}$.\\
  Determinar si $F$ es lineal.
  \item Sea $T: \mathbb{R}^3\to\mathbb{R}^3$, una transformación lineal , tal que: $T (1,1,1) = (1,0,2)$; $T( 1,0,1) = (0,1,1)$; $T ( 0,1,1) = ( 1,0,1)$. Encontrar $T(x,y,z)$
  \item Se considera $f : \mathbb{R}^2\to\mathbb{R}^3$ aplicación lineal tal que $f((1,-1)) = (-1,-2,-3)$ y $f((-3, 2)) = (0, 5, 3)$. Determinar, si es posible, $f((x, y))$ donde $(x, y) \in \mathbb{R}^2$.
  \item Sea la transformación $S : \mathcal{P}_2\to\mathcal{R}^2$, definida por:
  $$
  S(ax^2 + bx + c) = (a + b,c)
  $$
  Determinar:
  \begin{enumerate}
    \item Si $S$ es una transformación lineal
    \item El núcleo de la transformación $S$
    \item El recorrido de la transformación $S$
    \item Verificar $dim(\mathcal{P}_2)= dim(Nu(S))+dim(Img(S))$
  \end{enumerate}
  \item Para la transformación lineal $S : \mathbb{R}^2\to\mathcal{M}_2$ definida por:
  $$
  S(x,y,z) = \begin{pmatrix}
    x-2y & y+z\\
    y+z & x-y+z
  \end{pmatrix}
  $$
  donde $\mathcal{M}_2$ es el espacio vectorial real de las matrices simétricas de orden dos con elementos reales, obtener:
  \begin{enumerate}
    \item El núcleo $N(S)$ de la transformación, su dimensión y una de sus bases.
    \item El recorrido $S(\mathbb{R}^3)$ de la transformación, su dimensión y una de sus bases.
    \item Demostrar que: $dim\mathbb{R}^3 = dim N(S) + dim S(\mathbb{R}^3)$.
  \end{enumerate}
  \item Para la transformación lineal $T : \mathbb{R}^3\to\mathbb{R}^3$ definida por:
  $$
  T (x, y, z) = (3x + y,6x- z,2y + z)
  $$
  Obtener:
  \begin{enumerate}
    \item El núcleo de $T$ y su dimensión.
    \item El recorrido de $T$ y su dimensión.
  \end{enumerate}
  \item Sea $T: \mathbb{R}^2\to\mathbb{R}$, una transformación lineal definida por
  $$
  T (x,y,z) = 2x -3y + z
  $$
  \begin{enumerate}
    \item Encontrar $[T]_{\beta,\alpha}$ donde $\beta = \{(1,0,0),(1,1,0),(1,1,1)\}$ y $\alpha =\{2\}$
    \item  Encontrar kernel ($T$), Imagen ($T$), Nulidad($T$) y Rango ($T$).
  \end{enumerate}
  \item Sea $T: \mathbb{R}^4\to\mathbb{R}^3$ una transformación lineal definida por.
  \begin{align*}
    T(1,1,1,1) = (7,2,3)\\
    T(1,1,1,0) =(6,1,7)\\
    T(1,1,0,0) = (4,1,5)\\
    T(1,0,0,0) = (1,0,1)
  \end{align*}
  Hallar $T( x,y,z,w)$.
  \item Sean $f : \mathbb{R}^3\to\mathbb{R}^4$, $f(x_1, x_2, x_3) = (x_1+x_2, x_1+x_3, 0, 0)$ y $g : \mathbb{R}^4\to\mathbb{R}^2$, $g(x_1, x_2, x_3, x_4) = (x_1 - x_2, 2x_1 - x_2)$. Calcular el núcleo y la imagen de $f$, de $g$ y de $g \circ  f$. Decidir si son inyectivas, sobreyectivas o biyectivas.
  \item Dada $f : V \to V$, calcular $M_{BB'}(f)$ en cada uno de los siguientes casos:
  \begin{enumerate}
    \item $V = \mathbb{R}^3$; $f(x_1, x_2, x_3) = (3x_1 - 2x_2 + x_3, 5x_1 + x_2 - x_3, x_1 + 3x_2 + 4x_3)$
    \begin{enumerate}
      \item $B = B'$ la base canónica de $\mathbb{R}^3$.
      \item $B = \{(1, 2, 1); (-1, 1, 3); (2, 1, 1)\}$ y $B' = \{(1, 1, 0); (1, 2, 3); (2, 3, 4)\}$.
    \end{enumerate}
    \item $V = \mathbb{R}^{2\times2}$, $f(A) = A^t$, $B = B'$ la base canónica de $\mathbb{R}^{2\times2}$.
  \end{enumerate}
  \item Sea $B = \{v_1, v_2, v_3\}$ una base de $\mathbb{R}^3$ y $B' = \{w_1, w_2, w_3\}$ una base de $\mathbb{R}^4$. Sea $f : \mathbb{R}^3\to\mathbb{R}^4$ la transformación lineal tal que
  $$
  M_{BB'}(f) = \begin{pmatrix}
    1 & -2 & 1\\
    -1 & 1 & -1\\
    2 & 1 & 4\\
    3 & -2 & 5
  \end{pmatrix}
  $$
  \begin{enumerate}
    \item Hallar $f(3v_1 + 2v_2 - v_3)$ ¿Cuáles son sus coordenadas en la base $B'$?
    \item Hallar una base de $Nu(f)$ y una base de $Im(f)$.
    \item Describir el conjunto $f^{-1}(w_1 - 3w_3 - w_4)$.
  \end{enumerate}
  \item Sean $T: \mathbb{R}^4\to\mathbb{R}^3$ y $S: \mathbb{R}^3\to\mathbb{R}^2$ dos transformaciones lineales definidas por:
  $$
  T\begin{pmatrix}
    x\\
    y\\
    z\\
    w
  \end{pmatrix}=\begin{pmatrix}
    x+2y\\
    x-z\\
    w+2z
    \end{pmatrix}\qquad\text{y}\qquad S\begin{pmatrix}
    x\\
    y\\
    z
  \end{pmatrix}=\begin{pmatrix}
    2x+y\\
    3y+4z
  \end{pmatrix}
  $$
  encontrar la transformación lineal $S \circ T$.
  \item Interpretar geométricamente las siguientes aplicaciones lineales $f : \mathbb{R}^2 \to \mathbb{R}^2$
    \begin{enumerate}
      \item $f (x, y) = (x, 0)$
      \item $f (x, y) = (0, y)$
      \item $f (x, y) = (x, -y)$
      \item $f (x, y) = ( \frac{1}{2}(x + y), \frac{1}{2}(x + y))$
      \item $f (x, y) = (x\cdot cos(t) - y\cdot sen(t) , x\cdot sen(t) + y\cdot cos(t))$
    \end{enumerate}
  \item Considere la siguiente matriz
    $$
    A=\begin{pmatrix}
      1 & 3 & 4 & -3\\
      0 & 1 & 3 & -2\\
      3 & 7 & 6 & \alpha
  \end{pmatrix}
  $$
  y la siguiente aplicaciónla aplicación lineal $T : \mathbb{R}^4\to\mathbb{R}^3$, definida por $T(v) = Av$.
  \begin{enumerate}
    \item ¿Para qué valores de $\alpha$ el vector $u = (1, -1, 7)^t$ pertenece a la imagen de $T$?
    \item ¿Para qué valores de $\alpha$ el vector $v = (2, 1, -5, 0)^t$ pertenece al núcleo de $T$?
  \end{enumerate}
  \item Considere los vectores
  $$
  b_1 = (1, 0, 1)^t,\quad b_2 = (-1, 1, 2)^t,\quad b_3 = (0, 1, 5)^t,\quad u = (1, 2, 3)^t .
  $$
  Demostrar que $B = \{b_1 , b_2 , b_3\}$ es una base de $\mathbb{R}^3$. Hallar las coordenadas de $u$ con respecto a $B$.
  
  Cierta transformación lineal $T : \mathbb{R}^3\to\mathbb{R}^3$ verifica que $T(b_1) = e_1$, $T(b_2) = e_2$, $T(b_3) = e_3$, siendo $B_0 = \{e_1 , e_2 , e_3 \}$ la base canónica. Escribir la correspondiente matriz $A_{T,B_0}$ y hallar el núcleo y la imagen de $T$.
  \item Sean la base de $\mathbb{R}^2$, $B = \{v_1 , v_2\}$, donde $v_1 = (1, 1)^t$ y $v_2 = (-1, 0)^t$ , y la transformación lineal dada por $T ((x, y)^t ) = (4x - 2y, 2x + y)^t$, expresada respecto a la base canónica. Encontrar la matriz de $T$ relativa a la base dada.
  \item Determinar las ecuaciones del giro en el plano con centro $(3, 4)$ y ángulo $45^\circ$. Calcular su inversa.
\end{enumerate}
\end{document}