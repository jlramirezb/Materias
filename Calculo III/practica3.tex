\documentclass[12pt,letterpaper]{article}
\usepackage{amsmath}
\usepackage{amsfonts}
\usepackage{amssymb}
%\usepackage[spanish]{babel}
\usepackage[utf8]{inputenc}
\usepackage{geometry}
\geometry{tmargin=1.5cm, lmargin=1.5cm, rmargin=1.5cm, bmargin=1.5cm}
\usepackage{pgfplots}
\pgfplotsset{compat=1.18}
\usepackage{tikz}
\usetikzlibrary{intersections, backgrounds, patterns, calc, decorations.pathreplacing}
\usepgfplotslibrary{fillbetween}
\usepackage{multicol}


\begin{document}
\begin{center}
    {\Large \bf Práctica III. Cálculo III}
\end{center}

\begin{enumerate}
  \item Si $V_1 = (1,-1)$, $V_2 = (2,-1)$, $V_3=(-3,2)$ y $W_1=(1,0)$, $W_2=(0,-1)$, $W_3=(1,1)$. ¿Existe una transformación lineal $T: \mathbb{R}^2\to\mathbb{R}^2$, tal que $T(V_i) = W_i$ para $i=1,2,3$?
  \item Determinar cuáles de las siguientes aplicaciones son lineales:
  \begin{enumerate}
    \item $f : \mathbb{R}^3\to\mathbb{R}^2$ definida por $f((x; y; z)) = (x - y; y + 2z)$.
    \item $f : \mathbb{R}^3\to\mathbb{R}^2$ definida por $f((x; y; z)) = (x - y^2; y + 2z)$.
  \end{enumerate}
  \item Sean $\mathcal{P}_2$ el espacio vectorial real de los polinomios de grado menor o igual a dos con coeficientes reales y la transformación $F : \mathbb{R}^3\to\mathcal{P}_2$ definida por:\\
  $F(a,b,c) = (a + b)v_1 - cv_2$; donde $v_1 = x^2+1; v_2= 3x-1 \in \mathcal{P}_2 \quad\forall a,b,c \in \mathbb{R}$.\\
  Determinar si $F$ es lineal.
  \item Sea $T: \mathbb{R}^3\to\mathbb{R}^3$, una transformación lineal , tal que: $T (1,1,1) = (1,0,2)$; $T( 1,0,1) = (0,1,1)$; $T ( 0,1,1) = ( 1,0,1)$. Encontrar $T(x,y,z)$
  \item Se considera $f : \mathbb{R}^2\to\mathbb{R}^3$ aplicación lineal tal que $f((1,-1)) = (-1,-2,-3)$ y $f((-3, 2)) = (0, 5, 3)$. Determinar, si es posible, $f((x, y))$ donde $(x, y) \in \mathbb{R}^2$.
  \item Sea la transformación $S : \mathcal{P}_2\to\mathcal{R}^2$, definida por:
  $$
  S(ax^2 + bx + c) = (a + b,c)
  $$
  Determinar:
  \begin{enumerate}
    \item Si $S$ es una transformación lineal
    \item El núcleo de la transformación $S$
    \item El recorrido de la transformación $S$
    \item Verificar $dim(\mathcal{P}_2)= dim(Nu(S))+dim(Img(S))$
  \end{enumerate}
  \item Para la transformación lineal $S : \mathbb{R}^2\to\mathcal{M}_2$ definida por:
  $$
  S(x,y,z) = \begin{pmatrix}
    x-2y & y+z\\
    y+z & x-y+z
  \end{pmatrix}
  $$
  donde $\mathcal{M}_2$ es el espacio vectorial real de las matrices simétricas de orden dos con elementos reales, obtener:
  \begin{enumerate}
    \item El núcleo $N(S)$ de la transformación, su dimensión y una de sus bases.
    \item El recorrido $S(\mathbb{R}^3)$ de la transformación, su dimensión y una de sus bases.
    \item Demostrar que: $dim\mathbb{R}^3 = dim N(S) + dim S(\mathbb{R}^3)$.
  \end{enumerate}
  \item Para la transformación lineal $T : \mathbb{R}^3\to\mathbb{R}^3$ definida por:
  $$
  T (x, y, z) = (3x + y,6x- z,2y + z)
  $$
  Obtener:
  \begin{enumerate}
    \item El núcleo de $T$ y su dimensión.
    \item El recorrido de $T$ y su dimensión.
  \end{enumerate}
  \item Sea $T: \mathbb{R}^2\to\mathbb{R}$, una transformación lineal definida por
  $$
  T (x,y,z) = 2x -3y + z
  $$
  \begin{enumerate}
    \item Encontrar $[T]_{\beta,\alpha}$ donde $\beta = \{(1,0,0),(1,1,0),(1,1,1)\}$ y $\alpha =\{2\}$
    \item  Encontrar kernel ($T$), Imagen ($T$), Nulidad($T$) y Rango ($T$).
  \end{enumerate}
  \item Sea $T: \mathbb{R}^4\to\mathbb{R}^3$ una transformación lineal definida por.
  \begin{align*}
    T(1,1,1,1) = (7,2,3)\\
    T(1,1,1,0) =(6,1,7)\\
    T(1,1,0,0) = (4,1,5)\\
    T(1,0,0,0) = (1,0,1)
  \end{align*}
  Hallar $T( x,y,z,w)$.
\end{enumerate}
\end{document}