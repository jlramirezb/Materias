\documentclass[12pt]{article}
\usepackage{amsmath}
\usepackage{amsfonts}
\usepackage{amssymb}
\usepackage[spanish]{babel}
\usepackage[utf8]{inputenc}
\usepackage{geometry}
\geometry{tmargin=1cm, lmargin=1cm, rmargin=1cm, bmargin=1cm}
\usepackage{pgfplots}
\pgfplotsset{compat=1.18}
\usepackage{tikz}
\usetikzlibrary{intersections, backgrounds, patterns, calc, decorations.pathreplacing}
\usepgfplotslibrary{fillbetween}


\begin{document}
\begin{center}
    {\Large \bf Práctica I. Cálculo II}
\end{center}

\begin{enumerate}
  \item Una compañía de muebles fabrica butacas, mecedoras y sillas, y cada una de ellas de tres modelos: E(económico), M (medio) y L (lujo). Cada mes produce 20 modelos E, 15 modelos M y 10 modelos L de butacas; 12 modelos E, 8 modelos M y 5 modelos L de mecedoras; y 18 modelos E, 20 modelos M y 12 modelos L de sillas. Representa esta información en una matriz.
  \item El inventario de una librería de la carrera de Ciencias Económicas es:

Libros de: Derecho 80 , Contabilidad 160 , matemática 120 y Administración 240

Apuntes de : Derecho 40, Contabilidad 120 , Matemática 80 y Administración 160
\begin{enumerate}
  \item Represente mediante una matriz A , el inventario de esa librería.
  \item Expresar como el producto de un escalar por una matriz en forma conveniente, de manera que los elementos de esta última sean números de un dígito.
  \item Si la biblioteca de la Universidad tiene el 10\% de ese material ¿Cuál es la matriz en ese caso?
\end{enumerate}
\item Don Antonio tiene dos estaciones de servicio, una en el centro y otra en el sur de la ciudad. Durante el primer fin de semana de mayo, las estaciones registraron las ventas de combustibles representadas por la siguiente información:
\[
A =
\begin{array}{r@{\hspace{0.5em}}r r r} % r para las etiquetas de fila, r r r para las columnas de datos
\multicolumn{1}{c}{} & \text{diesel} & \text{super} & \text{super plus} \\ % Encabezados de columna
\text{centro} & 1200 & 750 & 650 \\
\text{sur} & 1100 & 850 & 600
\end{array}
\qquad
B =
\begin{array}{r@{\hspace{0.5em}}r r r} % r para las etiquetas de fila, r r r para las columnas de datos
\multicolumn{1}{c}{} & \text{diesel} & \text{super} & \text{super plus} \\ % Encabezados de columna
\text{centro} & 1260 & 860 & 520 \\
\text{sur} & 1160 & 750 & 750
\end{array}
\]
\begin{enumerate}
  \item Halle la matriz que represente el total de ventas realizado en los dos días.
  \item Si el lunes las ventas son siempre el 10\% más de la del día anterior ¿Cuál resulta ser la nueva matriz?
\end{enumerate}

\item Un constructor hace una urbanización con tres tipos de viviendas: S(sencillas), N(normales) y L(lujo). Cada vivienda de tipo S tiene 1 ventana grande, 7 medianas y1 pequeña. Cada vivienda de tipo N tiene 2 ventanas grandes, 9 medianas y 2 pequeñas. Y cada vivienda de tipo L tiene 4 ventanas grandes, 10 medianas y 3 pequeñas.

Cada ventana grande tiene 4 cristales y 8 bisagras; cada ventana mediana tiene 2 cristales y 4 bisagras; y cada ventana pequeña tiene 1 cristal y2 bisagras.
\begin{enumerate}
  \item Escribir una matriz que describa el número y tamaño de ventanas en cada tipo de vivienda y otra matriz que exprese el numero de cristales y el número de bisagras de cada tipo de ventana.
  \item Calcular una matriz, a partir de las anteriores, que exprese el número de cristales y bisagras necesarios en cada tipo de vivienda.
\end{enumerate}
\item Una empresa nacional tiene cuatro distribuidoras, una en cada región (norte, centro, sur y
Cuyo). Las ventas de tres de sus productos por región, expresadas en millones de dólares,
fueron:\\
\begin{tabular}{ll}
Año 2004 & Año 2005\\
Región 1, producto 1: 2.6 & Región 1, producto 1: 3.6\\
Región 1, producto 2: 3.2 & Región 1, producto 2: 4.5\\
Región 1, producto 3: 2.4 & Región 1, producto 3: 2.9\\
Región 2, producto 1: 4.8 & Región 2, producto 1: 2.5\\
Región 2, producto 2: 4.4 & Región 2, producto 2: 5.0\\
Región 2, producto 3: 3.6 & Región 2, producto 3: 3.0\\
Región 3, producto 1: 1.8 & Región 3, producto 1: 3.0\\
Región 3, producto 2: 2.5 & Región 3, producto 2: 3.5\\
Región 3, producto 3: 3.8 & Región 3, producto 3: 4.6\\
Región 4, producto 1: 0.9 & Región 4, producto 1: 2.5\\
Región 4, producto 2: 2.8 & Región 4, producto 2: 3.8\\
Región 4, producto 3: 2.5 & Región 4, producto 3: 4.0
\end{tabular}
\begin{enumerate}
\item Organizar los datos anteriores de modo que la información se presente en forma más clara.
\item Si llamamos A a la matriz de ventas del año 2004 y B a la del año 2005
\begin{enumerate}
  \item Dar el significado de los elementos a23 y b21 .
  \item Calcular las ventas totales de los dos años de cada producto y cada región.
  \item Calcular e interpretar $A - B$
  \item La gerencia de la empresa había proyectado para el año 2006 un 30\% de incremento en las ventas de los productos en todas las regiones respecto al año 2004. Calcular la diferencia entre los niveles de venta proyectados y los niveles
de venta reales del año 2005.
\end{enumerate}
\end{enumerate}
\item El número de horas que ha trabajado durante los últimos meses en cada actividad está dada por la matriz:
\begin{center}
\begin{tabular}{|l|c|c|c|c|}
\cline{2-5}
\multicolumn{1}{c|}{} & Marzo & Abril & Mayo & Junio \\
\hline
Clases particulares & 20 & 15 & 40 & 5 \\
\hline
Trabajos con computadora & 15 & 10 & 12 & 0 \\
\hline
Ciber & 30 & 20 & 16 & 10 \\
\hline
\end{tabular}
\end{center}
Calcúlese la matriz de precios totales e interprete estas matrices.
\end{enumerate}
\end{document}
