\documentclass[12pt]{article}
\usepackage{amsmath}
\usepackage{amsfonts}
\usepackage{amssymb}
\usepackage[spanish]{babel}
\usepackage[utf8]{inputenc}
\usepackage{geometry}
\geometry{tmargin=1cm, lmargin=1cm, rmargin=1cm, bmargin=1cm}
\usepackage{pgfplots}
\pgfplotsset{compat=1.18}
\usepackage{tikz}
\usetikzlibrary{intersections, backgrounds, patterns, calc, decorations.pathreplacing}
\usepgfplotslibrary{fillbetween}


\begin{document}
\begin{center}
    {\Large \bf Tarea. Sección 34. Cálculo II}
\end{center}

\begin{enumerate}
  \item Hallar el \'area sombreada entre las curvas $f(x)$ y $g(x)$

\begin{center}
\begin{tikzpicture}[
  domain=0:4,
  declare function={f(\x) = sqrt(max(\x,0)); g(\x) = 3/2 - \x/2;}, % Use max(\x,0) for safety if domain starts < 0
  scale=1 % Consider setting width/height in axis options instead
]
\begin{axis}[
  axis lines=middle,
  xmin=-1, xmax=4.5,
  ymin=-1.5, ymax=4.5,
  xlabel=$x$, ylabel=$y$,
  xtick={-1,0,1,2,3,4},
  ytick={-1,0,1,2,3,4},
  %xticklabels={-1,0,1,2,3,4},
  %yticklabels={-1,0,1,2,3,4},
  axis line style=latex-latex,
  ticklabel style={font=\scriptsize},
  label style={font=\small},
  legend style={at={(0.95,0.95)},anchor=north east,font=\scriptsize}
]
\addplot [thick, orange] {f(x)} node [above left] {$f(x) = \sqrt{x}$};
\addplot [thick, blue] {g(x)} node [below left] {$g(x) = \frac{3}{2} - \frac{x}{2}$};

\pgfmathsetmacro\intersectionX{1};
\pgfmathsetmacro\intersectionY{1}; % Valor calculado manualmente para la intersección
% --- Shading ---

% Añadir estos dos bloques \addplot para el sombreado correcto:
% Parte 1: Área bajo f(x) desde x=0 hasta x=intersectionX
\addplot [pattern=north west lines, pattern color=gray!90, fill opacity=1, draw=none, domain=0:\intersectionX] {f(x)} \closedcycle;


% Parte 2: Área bajo g(x) desde x=intersectionX hasta x=3
\addplot [pattern=north west lines, pattern color=gray!90, fill opacity=1, draw=none, domain=\intersectionX:3] {g(x)} \closedcycle;

% Opcional: Marcar el punto de intersección
%\fill [red] (axis cs:\intersectionX, \intersectionY) circle (2pt) node[above right, font=\tiny] {$(1,1)$};

\end{axis}
\end{tikzpicture}
\end{center}
\begin{itemize}
  \item Integrando con respecto a la variable $x$
  \item Integrando con respecto a la variable $y$
\end{itemize}
\item Calcular el área de ls región limitada por las siguientes curvas empleando el método de Simpson con $h=0.1$, Realice un dibujo del área definida por las curvas,
$$
\begin{cases}
y=x^2+1\\
y=0\\
x=1\\
x=2  
\end{cases}
$$
\item Utilizando sumas de Riemann, determinar el área bajo la curva $f(x)=3x^2-x+7$ en el intervalo $[0,1]$
\item Resuelva las integrales impropias
que se proponen a continuación y diga si divergen o convergen:
\begin{align*}
a) &\;\; \int_{2c}^{4c}\displaystyle\frac{dx}{\sqrt{x^2-4c^2}}&\quad
 b) &\;\;\int_1^\infty (1-x)e^{-x}dx
\end{align*}
\item Determinar la longitud del arco de la curva $f(x) = \displaystyle\int_{1}^{x} \sqrt{t + 1 + \frac{1}{t}} \, dt$, con $1 \le x \le 4$.

\end{enumerate}
\end{document}
