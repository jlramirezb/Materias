\documentclass[12pt,letterpaper]{article}
\usepackage{amsmath}
\usepackage{amsfonts}
\usepackage{amssymb}
%\usepackage[spanish]{babel}
\usepackage[utf8]{inputenc}
\usepackage{geometry}
\geometry{tmargin=1.5cm, lmargin=1.5cm, rmargin=1.5cm, bmargin=1.5cm}
\usepackage{pgfplots}
\pgfplotsset{compat=1.18}
\usepackage{tikz}
\usetikzlibrary{intersections, backgrounds, patterns, calc, decorations.pathreplacing}
\usepgfplotslibrary{fillbetween}
\usepackage{multicol}


\begin{document}
\begin{center}
    {\Large \bf Práctica IV. Cálculo III}
\end{center}

\begin{enumerate}
  \item Calcular el polinomio característico de cada una de las siguientes matrices:
  \begin{multicols}{3}
    \begin{enumerate}
    \item $\begin{pmatrix}
      3 & -4\\
      2 & -3
    \end{pmatrix}$
    \item $\begin{pmatrix}
      1 & 3\\
      5 & -2
    \end{pmatrix}$
    \item $\begin{pmatrix}
      2 & 0 & -1\\
      0 & -1 & 2\\
      1 & 0 & 1
    \end{pmatrix}$
    \end{enumerate}      
  \end{multicols}
  \item Calcular los autovalores y los autovectores de cada una de las siguientes matrices y los subespacios asociados a cada uno de ellos.
  \begin{multicols}{3}
    \begin{enumerate}
    \item $\begin{pmatrix}
      -7 & 5\\
      -10 & 8
    \end{pmatrix}$
    \item $\begin{pmatrix}
      4 & 2\\
      3 & 3
    \end{pmatrix}$
    \item $\begin{pmatrix}
      2 & -1\\
      -4 & 2
    \end{pmatrix}$
    \item $\begin{pmatrix}
      4 & 1\\
      0 & 4
    \end{pmatrix}$
    \item $\begin{pmatrix}
      3 & -5\\
      1 & -1
    \end{pmatrix}$
    \item $\begin{pmatrix}
      1 & 3\\
      2 & 2
      \end{pmatrix}$
    \item $\begin{pmatrix}
      2 & 3 & -1\\
      -1 & 1 & 4\\
      1 & 2 & -1
    \end{pmatrix}$
    \item $\begin{pmatrix}
      1 & 2 & -1\\
      0 & -5 & -4\\
      0 & 8 & 7
    \end{pmatrix}$
    \end{enumerate}              
  \end{multicols}
  \item Se considera la matriz
  $$
  A=\begin{pmatrix}
    5 & -2 & -3\\
    2 & 0 & -2\\
    3 & -2 & 1
  \end{pmatrix}
  $$
  \begin{enumerate}
    \item Calcular el espectro de $A$.
    \item Estudiar si $A$ es diagonalizable.
  \end{enumerate}
  \item Se considera la matriz
  $$
  A=\begin{pmatrix}
    1 & -1 & -1\\
    -1 & 1 & -1\\
    2 & 2 & \alpha
  \end{pmatrix}
  $$
  \begin{enumerate}
    \item Probar que $v = (1,-1, 0)$ es un autovector de $A$ asociado al autovalor $\lambda = 2$.
    \item Calcular el valor de $\alpha$ para el que 2 es el único autovalor de $A$.
    \item Para el valor de $\alpha$ calculado en el apartado b), calcular la multiplicidad geométrica de 2. ¿Es $A$ diagonalizable?
  \end{enumerate}
  \item Construye una matriz $M \in \mathcal{M}_2$ tal que $v_1 = (2, 3)$ sea autovector con autovalor 2 y $v_2 =
(1, 2)$ sea autovector con autovalor $-1$.
  \item Se considera la matriz
  $$
  B=\begin{pmatrix}
  1 & 2\\
  -2 & 1
  \end{pmatrix} \in \mathcal{M}_{2\times2}(\mathbb{R}) 
  $$
  \begin{enumerate}
    \item Estudiar si $B$ es diagonalizable.
    \item Hallar una diagonalización ortogonal de $M = B^tB$.
    \item Calcular una raíz cuadrada de $M = B^tB$.
  \end{enumerate}
  \item Se considera la matriz
  $$
  C=\begin{pmatrix}
  1 & -1 & 2\\
  -1 & 1 & -2\\
  2 & -2 & 4
  \end{pmatrix}
  $$
  \begin{enumerate}
    \item Probar que $C$ tiene rango 1 y razonar que $\sigma(C) = \{6, 0, 0\}$.
    \item Encontrar un vector unitario $u \in \mathbb{R}^3$ tal que $C= 6uu^t$.  
  \end{enumerate}
  \item En cada uno de los siguientes ítems decidir si la matriz $A$ es diagonalizable y en caso afirmativo hallar matrices $P$ y $D$ tales que $A = PDP^{-1}$ y $D$ es diagonal.
  \begin{multicols}{3}
    \begin{enumerate}
    \item $\begin{pmatrix}
      3 & 5\\
      3 & 1
    \end{pmatrix}$
    \item $\begin{pmatrix}
      1 & -1 & 4\\
      3 & 2 & -1\\
      2 & 1 & -1 
    \end{pmatrix}$
    \item $\begin{pmatrix}
      1 & -2 & 4\\
      3 & -4 & 6\\
      5 & -5 & 9 
    \end{pmatrix}$
    \item $\begin{pmatrix}
      1 & -1  & 1\\
      -1 & 1 & 1\\
      -1 & -1 & 3
    \end{pmatrix}$
    \item $\begin{pmatrix}
      4 & 6 & 6\\
      1 & 3 & 2\\
      -1 & -5 & -2
      \end{pmatrix}$
    \end{enumerate}
  \end{multicols}  
  \item Sea $A=\begin{pmatrix}
    -3 & 2\\
    -4 & 3
  \end{pmatrix}$. Calcular $A^{14}$
  \item Sea $A=\begin{pmatrix}
    0 & 2 & 2\\
    2 & 3 & 4\\
    2 & 4 & 3  
  \end{pmatrix}$. Calcular $A^{10}$
  \item \begin{enumerate}
    \item Sea $f : \mathbb{R}^3 \rightarrow \mathbb{R}^3$ la transformación lineal tal que $M_{EE}(f) = \begin{pmatrix}
      3 & -1 & 1\\
      0 & -5 & 0\\
      0 & 7 & 2
    \end{pmatrix}$. Hallar una base $B$ de $\mathbb{R}^3$ tal que $M_{BB}(f) = \begin{pmatrix}
      3 & 0 & 0\\
      0 & -5 & 0\\
      0 & 0 & 2
    \end{pmatrix}$
    \item Sea $f : \mathbb{R}^3 \rightarrow \mathbb{R}^3$ la transformación lineal tal que $M_{EE}(f) = \begin{pmatrix}
      -2 & -2 & 4\\
      0 & 4 & 0\\
      -6 & -2 & 8
    \end{pmatrix}$. Hallar una base $B$ de $\mathbb{R}^3$ tal que $M_{BB}(f) = \begin{pmatrix}
      4 & 0 & 0\\
      0 & 4 & 0\\
      0 & 0 & 2
    \end{pmatrix}$
  \end{enumerate}
  \item Probar que existe una única transformación lineal $A : \mathbb{R}^2 \rightarrow \mathbb{R}^2$ con autovectores
$(3, 1)$ de autovalor $-2$ y $(-2;-2)$ de autovalor 3. Dar el polinomio característico de $A$ y hallar la matriz de $A$ asociada a la base canónica.
\end{enumerate}
\end{document}