\documentclass[12pt]{article}
\usepackage{amsmath}
\usepackage{amsfonts}
\usepackage{amssymb}
%\usepackage[spanish]{babel}
\usepackage[utf8]{inputenc}
\usepackage{geometry}
\geometry{tmargin=1cm, lmargin=1cm, rmargin=1cm, bmargin=1cm}
\usepackage{pgfplots}
\pgfplotsset{compat=1.18}
\usepackage{tikz}
\usetikzlibrary{intersections, backgrounds, patterns, calc, decorations.pathreplacing}
\usepgfplotslibrary{fillbetween}
\usepackage{multicol}


\begin{document}
\begin{center}
    {\Large \bf Práctica II. Cálculo III}
\end{center}

\begin{enumerate}
  \item Se considera el conjunto de todos los polinomios de grado menor o igual a $n$, en variable real $x$, $\mathcal{P}_n = \{p(x)=a_nx^n+a_{n-1}x^{n-1}+\cdots+a_1x+a_0\quad /\quad a_i \in \mathbb{R} \}$. Demostrar que $\mathcal{P}_n$ es un espacio vectorial real con las operaciones:
  
  $
  \text{Suma de polinomios:}\left.\begin{array}{l}
    p(x) = a_nx^n+a_{n-1}x^{n-1}+\cdots+a_1x+a_0\\
    q(x) = b_nx^n+b_{n-1}x^{n-1}+\cdots+b_1x+b_0
  \end{array}\right\}
  $
  $$
  p(x)+q(x) = (a_n+b_n)x^n+(a_{n-1}+b_{n-1})x^{n-1}+ \cdots + (a_1+b_1)x+(a_0+b_0) \in \mathcal{P}_n
  $$
  Producto de un polinomio por un escalar:
  $$
  \lambda p(x) = (\lambda a_n)x^n + (\lambda a_{n-1})x^{n-1} + \cdots + (\lambda a_1)x + (\lambda a_0) \quad \lambda \in \mathbb{R}
  $$
  \item Demuestra que $\mathbb{C} = \{x + yi : x, y \in \mathbb{R}\}$, junto con la suma usual de números complejos y la multiplicación escalar $\alpha (x + yi) = \alpha x + \alpha yi, \alpha \in \mathbb{R}$, es un espacio vectorial sobre $\mathbb{R}$.
  \item Considere el conjunto $\mathbb{R}\times\mathbb{R}$ y las operaciones 
  \begin{align*}
    (x_1,x_2)\oplus (y_1,y_2) & = (x_1y_1,x_2y_2)\\
    \alpha (x_1,x_2) & = (\alpha x_1,\alpha x_2)    
  \end{align*}
  donde $x_i, y_i, \alpha \in \mathbb{R}$. Explica por qué $\mathbb{R} \times \mathbb{R}$, junto con estas operaciones, no es un espacio vectorial real.
  \item Sea $V = \{(x_1, x_2) : x_1, x_2 \in \mathbb{C}\}$. Para $(x_1, x_2)$, $(y_1, y_2) \in V$ y $\alpha \in \mathbb{C}$, definamos
  \begin{align*}
    (x_1, x_2) \oplus (y_1, y_2) & = (x_1 + y_1+1, x_2 + y_2+1)\\
    \alpha \odot (x_1, x_2) & = (\alpha x_1 + \alpha - 1, \alpha x_2+\alpha - 1)
  \end{align*}
  Demuestra que $V$ es un espacio vectorial sobre $\mathbb{C}$. ¿Cuál es el vector cero?, ¿Cuál es el inverso aditivo?
  \item Sea $S = \mathbb{R}^+$. Demuestra que $S$ es un espacio vectorial con las operaciones
  \begin{align*}
    v \oplus w = vw\\
    \alpha \odot v = v^{\alpha}
  \end{align*}
  ¿Qué elemento de $S$ es la identidad aditiva? ¿Qué significado tiene $-x$ en este contexto?
  \item Sea $\mathbb{R}$ el campo de los números reales. ¿Cuáles de los siguientes conjuntos son subespacios de $\mathbb{R}^3$?. Justifica tu respuesta.
  \begin{enumerate}
    \item $W_1 = \left\{(x_1,2x_2,3x_3) : x_1,x_2,x_3 \in \mathbb{R}\right\}$
    \item $W_2 = \left\{(x_1,x_2,x_3) : x_1,x_2,x_3 \in \mathbb{Q}\right\}$
    \item $W_3 = \left\{(x_1,x_1,x_1) : x_1 \in \mathbb{R}\right\}$
    \item $W_4 = \left\{(x_1,x_2,x_3) : x_1,x_2,x_3 \in \mathbb{R}\quad \wedge\quad x_1^2+x_2^2+x_3^2\geq1 \right\}$
  \end{enumerate}
  \item Determina si los conjuntos $S_i$ son subespacios del espacio vectorial $V_i$. Justifica tu respuesta detalladamente.
  \begin{enumerate}
    \item $S_1=\left\{(x_1,x_2) \in \mathbb{R}^2 : x_1\leq x_2\right\}, V_1=\mathbb{R}^2$
    \item $S_2=\left\{(x_1,\ldots,x_n) \in \mathbb{R}^n : x_1^2+\cdots+x_n^2=1\right\}, V_2=\mathbb{R}^n$
    \item $S_3=\left\{(x_1,x_2,x_3,x_4) \in \mathbb{R}^4 : x_1+x_2=x_3-x_4\quad\wedge\quad x_1+2x_2+3x_3+4x_4=0\right\}, V_3=\mathbb{R}^4$
  \end{enumerate}
  \item Sea $V=E_{2\times2}(\mathbb{R})$ el espacio vectorial de todas las matrices cuadradas de orden 2 sobre el cuerpo $\mathbb{R}$. Estudiar si los siguientes subconjuntos son subespacios vectoriales de $V$:
  \begin{multicols}{2}
    \begin{enumerate}
      \item $W=\left\{A \in E_{2\times2}(\mathbb{R}) / |A|=0\right\}$
      \item $U=\left\{A \in E_{2\times2}(\mathbb{R}) / A=A^2 \right\}$
    \end{enumerate}    
  \end{multicols}
  \item Establecer si los siguientes conjuntos son o no subespacios del respectivo espacio vectorial indicado justificando la respuesta (probar las propiedades de subespacio en caso que sea, o bien dar un contraejemplo que muestre la propiedad que falla en caso que no lo sea). En los casos que sea subespacio encontrar una base del mismo.
  %\begin{multicols}{2}
    \begin{enumerate}
      \item $R = \left\{(x_1,x_2) \in \mathbb{R}^2\quad /\quad x_1 = 3x_2\right\}$
      \item $W = \left\{(x_1,x_2,x_3) \in \mathbb{R}^3\quad /\quad 2x_1 - 3x_2 = 0\right\}$
      \item $U = \left\{(x_1,x_2) \in \mathbb{R}^2 / x_1\cdot x_2 = 9\right\}$
      \item $N=\left\{\begin{pmatrix}
      a_{11} & a_{12}\\
      a_{21} & a_{22}
    \end{pmatrix} \in \mathbb{R}^{2\times2}:a_{11}+a_{12}+a_{22}=0 \wedge a_{21}=5a_{12}\right\}$
    \end{enumerate}            
  %\end{multicols}
  \item Determinar si el vector $v=\begin{pmatrix}
  3\\
  9\\
  -4\\
  -2
  \end{pmatrix}$ pertenece al subespacio generado por $u_1=\begin{pmatrix}
  1\\
  -2\\
  0\\
  3\end{pmatrix}$, $u_2 = \begin{pmatrix}
    2\\
    3\\
    0\\
    -1
  \end{pmatrix}$ y $u_3 = \begin{pmatrix}
    2\\
    -1\\
    2\\
    1
  \end{pmatrix}$.
  \item Determinar el valor de $x$ para que el vector $(1, x, 5) \in \mathbb{R}^3$ pertenezca al subespacio $\left\{(1, 2, 3), (1, 1, 1)\right\}$.
  \item Establecer si $(1,-2,-3,-3)$ es o no una combinación lineal de los vectores $(0, 1, 2, 3)$, $(-1, 1, 1, 0)$. Esos 3 vectores son dependientes o independientes? Justificar la respuesta.
  \item Determinar si los siguientes vectores de $\mathbb{R}^3$ son linealmente independientes o dependientes:
  \begin{enumerate}
    \item $(1, 2, 4),(3, 6, 2),(0, 0, 1)$
    \item $(1, 2, 0),(0, 6, 2),(4, 8, 0)$
  \end{enumerate}
  ¿En algún caso se puede afirmar que formen una base de $\mathbb{R}^3$? Justificar.
  \item Escribir la matriz $E = \begin{pmatrix}
3 & 1\\
1 & -1
  \end{pmatrix}$ como combinación lineal de las matrices $A = \begin{pmatrix}
1 & 1\\
1 & 0
  \end{pmatrix}$, $B = \begin{pmatrix}
0 & 0\\
1 & 1
  \end{pmatrix}$ y $C = \begin{pmatrix}
0 & 2\\
0 & -1
  \end{pmatrix}$.
  \item Escribir el polinomio $v = x^2 + 4x - 3$ como una combinación lineal de los polinomios $e_1 = x^2 - 2x + 5$, $e_2 = 2x^2 - 3x$ y $e_3 = x + 3$.
  \item ¿Para qué valores de $\alpha$ dejan de formar base de $\mathbb{R}^3$ los vectores $\left\{\begin{pmatrix}
  \alpha\\
  1-\alpha\\
  \alpha\end{pmatrix},\begin{pmatrix}
  0\\
  3\alpha-1\\
  2
  \end{pmatrix},\begin{pmatrix}
    -\alpha\\
    1\\
    0
  \end{pmatrix}\right\}$?
  \item Sea $V=E_{2\times2}(\mathbb{R})$ el espacio vectorial de todas las matrices cuadradas de orden 2 sobre el cuerpo $\mathbb{R}$. Hallar las coordenadas de la matriz $A = \begin{pmatrix}
    2 & 3\\
    4 & -7
  \end{pmatrix} \in V$ en la base $B=\left\{\begin{pmatrix}
  1 & 1\\
  1 & 1
  \end{pmatrix}, \begin{pmatrix}
  0 & -1\\
  1 & 0
  \end{pmatrix}, \begin{pmatrix}
  1 & -1\\
  0 & 0
  \end{pmatrix}, \begin{pmatrix}
  1 & 0\\
  0 & 0
  \end{pmatrix}\right\}$
  \item En el espacio vectorial $W$ de las matrices simétricas reales de orden 2, se considera la base $B=\left\{\begin{pmatrix}
  1 & -1\\
  -1 & 2
  \end{pmatrix}, \begin{pmatrix}
  1 & 0\\
  0 & 1
  \end{pmatrix}, \begin{pmatrix}
  4 & 1\\
  1 & 0
  \end{pmatrix}, \begin{pmatrix}
  3 & -2\\
  -2 & 1
  \end{pmatrix}\right\}$. Hallar las coordenadas de la matriz $A$ en la base $B$ en los siguientes casos:
  \begin{multicols}{2}
    \begin{enumerate}
      \item $A=\begin{pmatrix}
        1 & -5\\
        -5 & 5
      \end{pmatrix}$
      \item $A=\begin{pmatrix}
        1 & 2\\
        2 & 4
      \end{pmatrix}$
    \end{enumerate}    
  \end{multicols}
  \item En $\mathbb{R}^3$ se consideran las bases $B=\left\{e_1,e_2,e_3\right\}$ y $B'=\left\{v_1,v_2,v_3\right\}$, siendo $B$ la base canónica y:
  $$
  \begin{cases}
    v_1 = 2e_1\\
    v_2 = -e_2+2e_3\\
    v_3 = -3e_3
  \end{cases}
  $$
  Hallar las coordenadas del vector $4e_1+e_2-5e_3$ en la base $B'$.
  \item Dado el vector $u$, cuyas coordenadas en la base canónica $B=\left\{e_1,e_2,e_3,e_4\right\}$ son $\begin{pmatrix}
    1\\1\\0\\1
  \end{pmatrix}$, calcular sus coordenadas en la base $B'=\left\{e_1',e_2',e_3',e_4'\right\}$, relacionada con la anterior por las siguientes ecuaciones:
  $$
  \begin{cases}
    e_1'=e_1+e_2\\
    e_2'=e_3\\
    e_3'=e_2+e_4\\
    e_4'=e_2-e_3
  \end{cases}
  $$
  \item Encontrar una base y la dimensión del subespacio vectorial
  $$
  S =\left\{(1, 2, -1, 3), (2, 1, 0, -2), (0, 1, 2, 1), (3, 4, 1, 2)\right\}.
  $$
  \item Sea $V$ un espacio vectorial de dimensión 4 con base $\mathcal{B} = \{u_1 , u_2 , u_3 , u_4 \}$. Se definen los vectores
  $$
  v_1 = 2u_1 + u_2 - u3 \qquad
  v_2 = 2u_1 + u_3 + 2u_4 \qquad
  v_3 = u_1 + u_2 - u_3 \qquad
  v_4 = -u_1 + 2u_3 + 3u_4
  $$
  Probar que $\mathcal{B'} = \{v_1 , v_2 , v_3 , v_4 \}$ es una base de $V$ y calcular las coordenadas en la base $\mathcal{B'}$ de un vector  $v$ que tiene por coordenadas en $\mathcal{B}$ a $(1, 2, 0, 1)$.
  \item Calcular la dimensión y una base del siguiente subespacio vectorial de $\mathcal{M}_{2\times2}(\mathbb{R})$:
  $$
  U=\left\{\begin{pmatrix}
  a & b\\
  c & d
  \end{pmatrix} \in \mathcal{M}_{2\times2}(\mathbb{R}) / \begin{array}{l}
    a - b - c = 0\\
    a + 2b + d = 0\\
    3b + c + d = 0
  \end{array} \right\}
  $$
  \item Se considera la matriz
  $$
  B=\begin{pmatrix}
    2 & -1\\
    -1 & 2    
  \end{pmatrix}
  $$
  Hallar una base y la dimensión del subespacio vectorial
  $$
  U = \left\{ X \in \mathcal{M}_{2\times2}(\mathbb{R}) / BX = 3X\right\}
  $$
  \item Se considera la matriz
  $$
  A=\begin{pmatrix}
    1 & -1 & 1\\
    1 & -1 & 1    
  \end{pmatrix}
  $$
  Hallar una base y la dimensión del subespacio vectorial
  $$
  U = \left\{ X \in \mathcal{M}_{2\times2}(\mathbb{R}) / XA = 0\right\}
  $$
  \item En $\mathbb{R}^2$ se considera el conjunto $\mathcal{B} = \{(3/5; 4/5); (-4/5; 3/5)\}$
  \begin{enumerate}
    \item Probar que $\mathcal{B}$ es una base de $\mathbb{R}^2$.
    \item Calcular la matriz de cambio de base $P = P_{B\to C}$ de la base $\mathcal{B}$ a la base canónica $\mathcal{C} =\{(1; 0); (0; 1)\}$ y probar que P es una matriz ortogonal
    \item Usar que P es ortogonal para calcular la matriz $P_{C\to B}$ de cambio de base de $\mathcal{C}$ a $\mathcal{B}$ y calcular las coordenadas de $v = (2; 1)$ respecto de la base $\mathcal{B}$.
  \end{enumerate}
  \item Considere las siguientes bases del espacio vectorial $\mathbb{R}^3$, $S = \{(0;-2; 3); (0; 1; 1); (1; 1; 0)\}$ y $T = \{(0;-1; 1); (0; 3; 0); (1;-1; 1)\}$. Sean $[u]_T = (2; 1; 3)$ y $[v]_S = (-1; 4; 1)$ dos vectores escritos en términos de las bases $S$ y $T$ respectivamente.
  \begin{enumerate}
    \item Determine la matriz de transición de la base $T$ a la base $S$.
    \item Encuentre $[u]_S$. 
    \item Determine la matriz de transición de la base $S$ a la base $T$.
    \item Encuentre $[v]_T$.
  \end{enumerate}
  \item En $\mathbb{R}^3$ se considera el conjunto $\mathcal{B} = \{(1;-1; 1); (-2; \lambda; 0); (-1; 1; \lambda)\}$.
  \begin{enumerate}
    \item Hallar los valores de $\lambda$ para los que $\mathcal{B}$ no es una base de $\mathbb{R}^3$.
    Para $\lambda = 1$, calcular las coordenadas del vector $v = (2; 1; 2)$ respecto de $\mathcal{B}$.
    \item Para $\lambda = 0$, hallar la matriz de cambio de coordenadas de la base canónica de $\mathbb{R}^3$ a la base $\mathcal{B}$.
    \item Para $\lambda = 2$, hallar una base ortonormal del subespacio generado por $\mathcal{B}$.
  \end{enumerate}
  \item Se considera el plano de $\mathbb{R}^3$ dado por
  $$
  U=\{(x,y,z)\in\mathbb{R}^3 / x=y+2z\}
  $$
  \begin{enumerate}
    \item Hallar una base ortonormal de $U$.
    \item Calcular la matriz $P$ de proyección ortogonal sobre $U$.
    \item Hallar la distancia de $v = (1; 1; 1)$ a $U$.
    \item Hallar una base del subespacio $W$ formado por los vectores de $\mathbb{R}^3$ cuya proyección ortogonal sobre $U$ es $(0; 0; 0)$.
  \end{enumerate}
\end{enumerate}
\end{document}