\documentclass[12pt]{article}
\usepackage[utf8]{inputenc}
\usepackage{amsmath}
\usepackage{amsfonts}
\usepackage{amssymb}
\usepackage{graphicx}
\usepackage{geometry}
\geometry{letterpaper, margin=1.5cm}

\title{Quiz 3 - Cálculo III. Sección 2}
\author{Nombre: \rule{7cm}{0.4pt}  CI: \rule{5cm}{0.4pt}}
\date{\today}

\begin{document}

\maketitle

\begin{enumerate}
    \item Sea $T: \mathbb{R}^4\to\mathbb{R}^3$ una transformación lineal definida por.
  \begin{align*}
    T(1,1,1,1) = (7,2,3)\\
    T(1,1,1,0) =(6,1,7)\\
    T(1,1,0,0) = (4,1,5)\\
    T(1,0,0,0) = (1,0,1)
  \end{align*}
  Hallar $T( x,y,z,w)$.
  \item Sea $T: \mathbb{R}^2\to\mathbb{R}$, una transformación lineal definida por
  $$
  T (x,y,z) = 2x -3y + z
  $$
  \begin{enumerate}
    \item Encontrar $[T]_{\beta,\alpha}$ donde $\beta = \{(1,0,0),(1,1,0),(1,1,1)\}$ y $\alpha =\{2\}$
    \item  Encontrar kernel ($T$), Imagen ($T$), Nulidad($T$) y Rango ($T$).
  \end{enumerate}
  \item Sea $T: \mathbb{R}^3\to\mathbb{R}^3$, una transformación lineal , tal que: $T (1,1,1) = (1,0,2)$; $T( 1,0,1) = (0,1,1)$; $T ( 0,1,1) = ( 1,0,1)$. Encontrar $T(x,y,z)$
\end{enumerate}
\end{document}
