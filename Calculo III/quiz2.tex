\documentclass[12pt]{article}
\usepackage[utf8]{inputenc}
\usepackage{amsmath}
\usepackage{amsfonts}
\usepackage{amssymb}
\usepackage{graphicx}
\usepackage{geometry}
\geometry{letterpaper, margin=1.5cm}

\title{Quiz 2 - Cálculo III. Sección 2}
\author{Nombre: \rule{7cm}{0.4pt}  CI: \rule{5cm}{0.4pt}}
\date{\today}

\begin{document}

\maketitle

\begin{enumerate}
    \item ¿El conjunto de vectores $\{(1,2,3),(2,1,0),(3,3,3)\}$ es linealmente independiente en $\mathbb{R}^3$?
    \item ¿El polinomio $p(x)=2-3x+x^2$ pertenece al subespacio generado por $S=\{1-x,x-x^2\}$?
    \item hallar el vector de coordenadas de $v=(7,5)$ en $\mathbb{R}^2$ con respecto a la base $B=\{(1,1),(1,-1)\}$.
    \item Considere las siguientes bases del espacio vectorial $\mathbb{R}^3$, $S = \{(0;-2; 3); (0; 1; 1); (1; 1; 0)\}$ y $T = \{(0;-1; 1); (0; 3; 0); (1;-1; 1)\}$. Sean $[u]_T = (2; 1; 3)$ y $[v]_S = (-1; 4; 1)$ dos vectores escritos en términos de las bases $S$ y $T$ respectivamente.
    \begin{enumerate}
      \item Determine la matriz de transición de la base $T$ a la base $S$.
      \item Encuentre $[u]_S$. 
      \item Determine la matriz de transición de la base $S$ a la base $T$.
      \item Encuentre $[v]_T$.
  \end{enumerate}
  \item Se considera la matriz
  $$
  B=\begin{pmatrix}
    2 & -1\\
    -1 & 2    
  \end{pmatrix}
  $$
  Hallar una base y la dimensión del subespacio vectorial
  $$
  U = \left\{ X \in \mathcal{M}_{2\times2}(\mathbb{R}) / BX = 3X\right\}
  $$
\end{enumerate}
\end{document}
