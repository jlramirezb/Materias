\documentclass[12pt]{article}
\usepackage[utf8]{inputenc}
\usepackage{amsmath}
\usepackage{amsfonts}
\usepackage{amssymb}
\usepackage{graphicx}
\usepackage{geometry}
\geometry{letterpaper, margin=0.5cm}

\title{Quiz 1 - Cálculo III. Sección 2}
\author{Nombre: \rule{7cm}{0.4pt}  CI: \rule{5cm}{0.4pt}}
\date{\today}

\begin{document}

\maketitle

\begin{enumerate}
    \item Intel, en su continua búsqueda por optimizar la producción y satisfacer la demanda del mercado de laptops, está planificando la fabricación de tres nuevos modelos de procesadores: el Core i5 (para laptops de 13 pulgadas), el Core i7 (para laptops de 15 pulgadas) y el Xeon (para estaciones de trabajo portátiles de 17 pulgadas).
    \begin{itemize}
      \item {\bf Costos de fabricación}: El procesador Core i5 tiene un costo de fabricación de \$10 por unidad, el Core i7 de \$20 por unidad, y el Xeon de \$30 por unidad.
      \item {\bf Capacidad de producción}: La línea de ensamblaje tiene una capacidad total para producir 100.000 procesadores combinados de los tres tipos.
      \item {\bf Demanda del mercado}: Debido a las tendencias de venta, Intel estima que la cantidad de procesadores Core i7 fabricados debe ser el doble de la cantidad de procesadores Xeon.
      \item {\bf Presupuesto total}: Intel ha destinado un presupuesto total de \$1,800,000 para la fabricación de estos tres tipos de procesadores.
    \end{itemize}
    \begin{enumerate}
      \item Plantea un sistema de ecuaciones que describa el problema anterior y exprésalo en forma matricial.
      \item Reduciendo el sistema de ecuaciones a una forma triangular, encuentra la respuesta del problema: ¿Cuántos procesadores de cada tipo debe fabricar Intel?.
      \item Calculando el determinante de la matriz contesta si el sistema tiene solución única o no.
      \item Encuentra la matriz inversa.
      \item Con la matriz inversa encuentra la solución: ¿Cuántos procesadores de cada tipo debe fabricar Intel?.
    \end{enumerate}
    \item Encuentre el determinante de la siguiente matriz, usando las operaciones elementales por filas y propiedades de matrices.
    $$
    A=\begin{bmatrix}
    2 & 5 & -3 & -2\\
    -2 & -3 & 2 & -5\\
    1 & 3 & -2 & 2\\
    -1 & -6 & 4 &  3
    \end{bmatrix}
    $$
    \item Calcular el Rango $r$ de la siguiente matriz.
    $$
    A=\begin{bmatrix}
    1 & 1 & 0 & 0 & 0\\
    0 & 1 & 1 & 0 & 0\\
    0 & 0 & 1 & 1 & 0\\
    0 & 0 & 0 & 1 & 1\\
    \alpha & 0 & 0 & 0 & 1    
    \end{bmatrix}    
    $$
\end{enumerate}
\end{document}
