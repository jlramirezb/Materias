\documentclass{beamer}
\mode<presentation>
\usepackage{amssymb,textcomp}
%\usepackage{beamerthemesplit}
\usepackage{beamerthemeJuanLesPins}
\usepackage{verbatim}
\usefonttheme{serif}
\title{Resoluci\'on Num\'erica de Sistemas de Ecuacionesl Lineales.}
\author{Jos\'e Luis Ram\'irez B.}
\date{\today}

\begin{document}

\frame{\titlepage}

%\section[Introducci\'on]{}
\frame{\tableofcontents}

\section{Introducci\'on}
\begin{frame}[fragile]
  \frametitle{Motivaci\'on.} 
  \begin{itemize}
    \item En el planteamiento matem\'atico de muchos problemas realistas, los sistemas de ecuaciones algebraicas, y de una manera especial los lineales, aparecen de manera natural.
    \item<2-> La b\'usqueda de m\'etodos de resoluci\'on de sistemas de ecuaciones lineales es un tema de gran importancia en la ciencia.
    \item<3-> El objetivo de este tema es desarrollar estrategias num\'ericas que permitan resolver sistemas de ecuaciones relativamente grandes de una manera eficiente.
  \end{itemize}    
\end{frame}
% \begin{frame}[fragile]
%  %\frametitle{Motivaci\'on.}      
%  La cercha de la figura se carga con una fuerza uniforme repartida sobre el cord\'on superior

%  \begin{center}
%  %\includegraphics[scale=0.5]{cercha.png}
% \end{center}

%  El planteamiento del problema conduce a un sistema lineal de ecuaciones de dimensi\'on $n=50$ y en el que la matriz tiene la siguiente estructura

%  \begin{center}
%  %\includegraphics[scale=0.5]{matriz.png}
% \end{center}
 
%  Al resolver el sistema, se obtiene la deformaci\'on de la estructura
 
%  \begin{center}
%  %\includegraphics[scale=0.5]{cerchadef.png}
% \end{center}
 
% \end{frame}
%%%%
\frame
{
  \frametitle{Motivaci\'on.} 
  \begin{itemize}
   \item<1->La formulaci\'on de problemas de ingenier\'ia a menudo conduce a sistemas lineales de ecuaciones. Estos sistemas pueden llegar a tener cientos o miles de grados de libertad. 
   \item<2-> El objetivo de este tema es desarrollar estrategias num\'ericas que permitan resolver sistemas de ecuaciones relativamente grandes de una manera eficiente. 
   \item<3-> Adem\'as, se analizar\'an con detalle algunos m\'etodos directos.
  \end{itemize}
}
%%%%
\frame
{
   \frametitle{Motivaci\'on.}
   
Si bien existen m\'etodos exactos  como el m\'etodo de Cramer, estos son muy costosos de aplicar en situaciones donde los sistemas a resolver tienen muchas ecuaciones.

El n\'umero total de operaciones para resolver un sistema de dimensi\'on $n$ con este m\'etodo es
$$
T_C = (n+1)^{2}n!-1
$$

\begin{table}[!ht]
 \begin{tabular}{|c|c|}\hline
$n$ & $T_C$  \\\hline
$5$ & $4319$ \\\hline
$10$ & $4\times10^8$\\\hline
$100$ &  $10^{158}$\\\hline
 \end{tabular}
\caption{Operaciones elementales del m\'etodo de Cramer segun el tama\~no de la matriz($n$).}
\end{table}
}
%%%%
\frame{
\frametitle{Motivaci\'on.}
Desde el punto de vista num\'erico se buscan algoritmos eficientes en diferentes aspectos:
\begin{itemize}
 \item<2-> N\'umero de operaciones necesarias (tiempo CPU)
 \item<3-> Necesidades de almacenamiento (memoria)
 \item<4-> Rango de aplicabilidad (sobre que tipo de matrices se pueden aplicar)
\end{itemize}
}
%%%%%
\frame
{
\frametitle{Introducci\'on.}
Un sistema de $n$-ecuaciones (con coeficientes reales) en las $n$-inc\'ognitas $x_1,
x_2,\ldots,x_n$ es un conjunto de $n$ ecuaciones de la forma:

$$
\left\{
\begin{array}{c}
  f_1(x_1,x_2,\ldots,x_n)=0 \\
  f_2(x_1,x_2,\ldots,x_n)=0 \\
  \dotfill \\
  f_n(x_1,x_2,\ldots,x_n)=0 \\
\end{array}
\right.
$$

donde
$$f_i(x_1,x_2,\ldots,x_n) = a_{i,1}x_1 + a_{i,2}x_2 + \cdots + a_{i,n}x_n - b_i$$
}
%%%
\frame
{
con $a_{i,1},a_{i,2},\ldots,a_{i,n}$ y $b_i$ constantes reales, el sistema se dice lineal (con coeficientes reales);
en cualquier otro caso el sistema se dice no-lineal.

\vspace{0.5cm}
\uncover<2->{
  A los n\'umeros $a_{ij}$ se les denomina coeficientes del sistema y a los $b_i$ t\'erminos independientes.  
}

\vspace{0.5cm}
\uncover<3->{
Si $C = (c_1, c_2,\ldots,c_n) \in \mathbb{R}^n$ es tal que $f_i(c_1,c_2,\ldots,c_n) = 0$ para cada $i = 1,2,\ldots,n$,
entonces se dice que $C$ es una soluci\'on real del sistema planteado.
}}
%%%%
\frame{
Si se introducen las matrices
$$
  A = \left[\begin{array}{cccc}
            a_{11} & a_{12} & \cdots & a_{1n}\\
            a_{21} & a_{22} & \cdots & a_{2n}\\
            \vdots & \vdots & \ddots & \vdots\\
            a_{m1} & a_{m2} & \cdots & a_{mn}\\
        \end{array}\right], \qquad x=\left[\begin{array}{c}
          x_1\\
          x_2\\
          \vdots\\
          x_n\\
        \end{array}\right], \qquad b=\left[\begin{array}{c}
          b_1\\
          b_2\\
          \vdots\\
          b_m\\
        \end{array}\right]
$$

el sistema se puede representar de forma m\'as compacta por
$$
Ax=b
$$
}
%%%%
\frame{
  Podemos clasificar los sistemas de ecuaciones lineales atendiendo a:
  \begin{enumerate}
    \item Su tama\~no
      \begin{enumerate}
        \item Peque\~nos: $n \leq 300$ donde $n$ representa el n\'umero de ecuaciones.
        \item Grandes: $n>300$
      \end{enumerate}
    \item<2-> Su estructura
      \begin{enumerate}
        \item Si la matriz posee pocos elementos nulos diremos que se trata de un sistema lleno.
        \item Si, por el contrario, la matriz contiene muchos elementos nulos, diremos que la matriz, y por lo tanto, el sistema lineal es disperso o \textit{sparce}.
      \end{enumerate}
\end{enumerate}
}
\frame{
  \begin{enumerate}
  \item Matrices de este tipo son las denominadas
      \begin{itemize}
        \item Tridiagonales
          $$
          \left(\begin{array}{cccc}
            a_{1,1} & a_{1,2} & 0 & 0 \\
            a_{2,1} & a_{2,2} & a_{2,3} & 0 \\
            0 & a_{3,2} & a_{3,3} & a_{3,4} \\
            0 & 0 & a_{4,3} & a_{4,4} \\
            \end{array}\right)
          $$
      \item<2-> Triangulares Superiores
          $$
          \left(\begin{array}{cccc}
            a_{1,1} & a_{1,2} & a_{1,3} & a_{1,4} \\
            0 & a_{2,2} & a_{2,3} & a_{2,4} \\
            0 & 0 & a_{3,3} & a_{3,4} \\
            0 & 0 & 0 & a_{4,4} \\
            \end{array}\right)
          $$
      \item<3-> Triangulares Inferiores
          $$
          \left(\begin{array}{cccc}
            a_{1,1} & 0 & 0 & 0 \\
            a_{2,1} & a_{2,2} & 0 & 0 \\
            a_{3,1} & a_{3,2} & a_{3,3} & 0 \\
            a_{4,1} & a_{4,2} & a_{4,3} & a_{4,4} \\
            \end{array}\right)
          $$
    \end{itemize}
   \end{enumerate} 
}
%%%%
\frame{
\frametitle{Existencia y unicidad de soluciones}
\begin{block}{Teorema: Compatibilidad de un sistema de ecuaciones lineales}
  La ecuaci\'on $Ax=b$ admite soluci\'on si y s\'olo si
  $$
  rango(A|b) = rango(A)
  $$
\end{block}

\uncover<2->{
\begin{block}{Corolario}
 Si $A^{m\times n}$ tiene rango $m$, $Ax=b$ siempre tiene soluci\'on
\end{block}}

\uncover<3->{
\begin{block}{Teorema}
  Si $x_0$ es una soluci\'on de $Ax=b$, el conjunto de soluciones de la ecuaci\'on est\'a dado por $x_0+ker(A)$.
\end{block}}

\uncover<4->{
\begin{block}{Corolario}
 Una soluci\'on de $Ax=b$ es \'unica si y s\'olo si $ker(A)=\emptyset$.
\end{block}}
}
%%%%
\frame{
\frametitle{Existencia y unicidad de soluciones}
Consid\'erese una matriz cuadrada $A \in \mathbb{R}^{n\times n}$. Las siguientes condiciones son equivalentes:
\begin{itemize}
 \item<2-> Para cualquier $b \in \mathbb{R}^{n}$ el sistema $Ax=b$  tiene soluci\'on.
 \item<3-> Si $Ax=b$ tiene soluci\'on, \'esta es \'unica.
 \item<4-> Para cualquier $x \in \mathbb{R}^{n}$, $Ax=0 \Rightarrow x=0$.
 \item<5-> Las columnas (filas) de la matriz $A$ son linealmente independientes.
 \item<6-> Existe una matriz cuadrada $A^{-1}$ (matriz inversa) tal que
 $$
 AA^{-1} = A^{-1}A = I
 $$
 \item<7-> La matriz $A$ tiene determinante no nulo
 $$
 |A| = \det(A) \neq 0
 $$
\end{itemize}
}
%%%%%
\frame{
La primera opci\'on que se plantea es
$$
x = A^{-1}b
$$
\begin{itemize}
 \item<2-> No es eficiente (demasiadas operaciones).
 \item<3-> Si el determinante de $A$ es pr\'oximo a cero, el error de redondeo puede ser muy grande, y esto es dificil de estimar num\'ericamente
 $$
 \det(\gamma A) = \gamma^{n}\det(A)
 $$
\end{itemize}
}
%%%%
\frame{
Se requieren m\'etodos num\'ericos alternativos
\begin{itemize}
 \item<2-> m\'etodos directos, son exactos (no tienen asociado error de truncamiento), y son usados 
cuando la mayor\'ia de los coeficientes de $A$ son distintos de cero y las matrices no son demasiado 
grandes. Suelen ser algoritmos ``complicados de implementar''
 \item<3-> m\'etodos indirectos o iterativos, tienen asociado un error de truncamiento y se usan 
preferiblemente para matrices grandes ($n>>1000$) cuando los coeficientes de $A$ son la mayor\'ia 
nulos (matrices sparse). Algoritmos sencillos de implementar que requiere aproximaci\'on inicial y 
que en general no tiene porqu\'e converger (requieren an\'alisis de convergencia previo).
\end{itemize}

}
\section{M\'etodos Directos}
\begin{frame}
  \frametitle{M\'etodos Directos}
  \begin{itemize}
    \item \underline{\textbf{CASO 1}}: La matriz $A$ de coeficientes del sistema $Ax = b$ es triangular (superior o inferior) con
    todas sus componentes sobre la diagonal principal no nulas.
    
    $$
    \left\{\begin{array}{rcc}
            a_{1,1}x_1 + a_{1,2}x_2 + \cdots + a_{1,i}x_i + \cdots + a_{1.n}x_n & = & b_1\\
             a_{2,2}x_2 + \cdots + a_{2,i}x_i + \cdots + a_{2,n}x_n & = & b_2\\
              \vdots & & \vdots\\
                 a_{i,i}x_i + \cdots + a_{i,n}x_n & = & b_i\\
                        \vdots & & \vdots\\
                     a_{n,n}x_n & = & b_n\\
           \end{array}\right.
    $$
  \end{itemize}
\end{frame}
%%%%%
\frame{
  Como $a_{n,n} \neq 0$, se puede despejar $x_n$ de la \'ultima ecuaci\'on, y se obtiene:
  $$
  x_n = \frac{b_n}{a_{n,n}}
  $$
  conocido el valor de $x_n$, se puede emplear la pen\'ultima ecuaci\'on para conocer $x_{n-1}$
  $$
  x_{n-1} = \displaystyle\frac{b_{n-1}-a_{n-1,n}x_n}{a_{n-1,n-1}}
  $$
  conocidos $x_n$ y $x_{n-1}$, se obtiene de la antepen\'ultima ecuaci\'on
  $$
  x_{n-2} = \displaystyle\frac{b_{n-2}-\left(a_{n-2,n-1}x_{n-1}+a_{n-2,n}x_{n}\right)}{a_{n-2,n-2}}
  $$
}
%%%%%
\frame{
  En general, conocidos $x_n, x_{n-1},\ldots,x_{i+1}$, se obtiene:
  $$
  x_i =  \frac{b_i - \displaystyle\sum_{k=i+1}^na_{i,k}x_k}{a_{i,i}} \quad\,\,\,\, \forall i=n-1,n-2,\ldots,2,1
  $$
  El m\'etodo anterior para determinar la soluci\'on del sistema se denomina sustituci\'on regresiva o hacia atr\'as.  
}
%%%
\begin{frame}
  Si la matriz de coeficientes del sistema es triangular inferior, para resolver el sistema podemos proceder de manera similar al caso anterior, pero empezando por despejar $x_1$ de la primera ecuaci\'on. El procedimiento en este caso se denomina sustituci\'on progresiva o hacia adelante.
  $$
    \left\{\begin{array}{lcc}
            a_{1,1}x_1  & = & b_1\\
            a_{2,1}x_1 + a_{2,2}x_2 & = & b_2\\
              \vdots & & \vdots\\
            a_{i,1}x_1 + \cdots + a_{i,i-1}x_{i-1} + a_{i,i}x_i  & = & b_i\\
            \vdots & & \vdots\\
            a_{n,1}x_{1} + a_{n,2}x_2 + \cdots + a_{n,i-1}x_{i-1} + a_{n,i}x_i + \cdots + a_{n,n}x_n & = & b_{n}
           \end{array}\right.
    $$
\end{frame}
%%%%
\frame{
  \begin{itemize}
    \item \underline{\textbf{CASO 2}}: \underline{\textbf{CASO 2}}: La matriz $A$ de coeficientes, del sistema lineal $Ax = b$, es tal que no se requieren
    intercambios de
    filas para culminar con \'exito la eliminaci\'on Gaussiana.
    
    Digamos que el sistema $Ax = b$ tiene la forma
    $$
    \left\{\begin{array}{cccc}
      E_1: & a_{1,1}x_1 + a_{1,2}x_2 + \cdots + a_{1,j}x_j + \cdots + a_{1,n}x_n & = & b_1\\
      E_2: & a_{2,1}x_1 + a_{2,2}x_2 + \cdots + a_{2,j}x_j + \cdots + a_{2,n}x_n & = & b_2\\
      & \vdots &   &\\
      E_j: & a_{j,1}x_1 + a_{j,2}x_2 + \cdots + a_{j,j}x_j + \cdots + a_{j,n}x_n & = & b_j\\
      & \vdots &   &\\
      E_i: & a_{i,1}x_1 + a_{i,2}x_2 + \cdots + a_{i,j}x_j + \cdots + a_{i,n}x_n & = & b_i\\
      & \vdots &   &\\
      E_n: & a_{n,1}x_1 + a_{n,2}x_2 + \cdots + a_{n,j}x_j + \cdots + a_{n,n}x_n & = & b_n\\
      \end{array}\right.
    $$
  \end{itemize}
}
%%%%%
\frame{
  El proceso de eliminaci\'on Gaussiana sin Pivoteo consiste en lo siguiente:
\begin{enumerate}
 \item<2-> Se elimina el coeficiente de $x_1$ en cada una de las ecuaciones $E_2,E_3,\ldots,E_n$ para obtener un sistema
equivalente $A^{(1)}x=b^{(1)}$, realizando las operaciones elementales
$$
\left(E_i - \left(\frac{a_{i,1}}{a_{1,1}}\right)E_1\right)\rightarrow E_i^{(1)}, \quad \forall\,\,\, i=2,3,\ldots,n
$$
  \item<3-> Se elimina el coeficiente de $x_2$ en cada una de las ecuaciones $E_3^{(1)},E_4^{(1)},\ldots,E_n^{(1)}$, para
obtener un sistema equivalente $A^{(2)}x=b^{(2)}$, realizando las operaciones elementales
$$
\left(E_i^{(1)} - \left(\frac{a_{i,2}^{(1)}}{a_{2,2}^{(1)}}\right)E_2^{(1)}\right)\rightarrow E_i^{(2)}, \quad
\forall\,\,\, i=3,4,\ldots,n
$$  
\end{enumerate}
}
%%%%%
\frame{
\begin{enumerate}
  \item En general, eliminados los coeficientes de $x_1,x_2,\ldots,x_{j-1}$, se elimina el coeficiente de $x_j$ en cada
  una de las ecuaciones, para obtener un sistema equivalente $A^{(j)}x=b^{(j)}$, realizando las operaciones elementales
  $$
  \left(E_i^{(j-1)} - \left(\frac{a_{i,j}^{(j-1)}}{a_{j,j}^{(j-1)}}\right)E_j^{(j-1)}\right)\rightarrow E_i^{(j)}, \quad
  \forall\,\,\, i=j+1,\ldots,n
  $$
  debe ocurrir que $a_{j,j}\neq 0$.
\end{enumerate}
}
%%%%%
\frame{
  Los n\'umeros
$$
m_{i,j} = \frac{a_{i,j}^{(j-1)}}{a_{j,j}^{(j-1)}}\quad \forall\,\,\, j=1,\ldots,n-1,\,\, i=j+1,\ldots,n
$$
se llaman multiplicadores.

El sistema resultante tendr\'a entonces la forma triangular superior con elementos no nulos en la diagonal, por lo tanto, se puede resolver mediante sustituci\'on regresiva.
}














%%%%
\section{Condicionamiento}
\frame
{
   \frametitle{Condicionamiento de Sistemas.}
   \begin{itemize}
    \item<1-> La existencia de un sistema mal condicionado es fuente de posibles errores y dificultades a la hora de resolver un sistema lineal. Se plantea entonces, c\'omo definir y cuantificar el condicionamiento de un sistema.
    \item<2-> Supongase que se tiene el sistema lineal $Ax = b$, donde $A$ es la matriz de coeficientes, $b$ es el t\'ermino independiente y $x$ es la soluci\'on exacta del sistema, que llamaremos $u$.
   \end{itemize}
   \uncover<3->{\begin{block}{}
                 $$Ax=b \Rightarrow x=u$$
                \end{block}}
}
%%%%
\frame
{
\frametitle{Condicionamiento de Sistemas.}
 Modificando el t\'ermino independiente mediante una perturbaci\'on $\delta b$, entonces tenemos que resolver el sistema $Ax = b + \delta b$, que tendr\'a una nueva soluci\'on (distinta de la soluci\'on exacta) que llamaremos $u + \delta u$:

\uncover<2->{
\begin{block}{}
$$
Ax=b+\delta b \Rightarrow x=u+\delta u
$$
\end{block}
}
}
%%%%
\frame{
\frametitle{Ejemplo:}
Consideremos los dos sistemas de ecuaciones lineales
$$
Ax=b \to \left[\begin{array}{cc}
                8 & -5\\
                4 & 10
               \end{array}\right]\left[\begin{array}{c}
                                        x_1\\
                                        x_2
                                       \end{array}\right] = \left[\begin{array}{c}
                                        3\\
                                        14
                                       \end{array}\right]
$$
y
$$
Bx=c \to \left[\begin{array}{cc}
                0.66 & 3.34\\
                1.99 & 10.01
               \end{array}\right]\left[\begin{array}{c}
                                        x_1\\
                                        x_2
                                       \end{array}\right] = \left[\begin{array}{c}
                                        4\\
                                        12
                                       \end{array}\right]
$$
La soluci\'on de ambos es el vector $[1, 1]^{T}$

\uncover<2->{
Si se introduce una perturbaci\'on $[-0.04;-0.06]^{T}$ en el t\'ermino independiente del primer sistema, su soluci\'on pasar\'a a ser $[0.993;0.9968]^{T}$
}
}
%%%%%
\frame{
\frametitle{Ejemplo:}
\begin{itemize}
 \item<1-> El cambio relativo en la norma eucl\'idea del vector $b$ es
 $$
 \dfrac{\|\delta b\|_2}{\|b\|_2} = \dfrac{\sqrt{0.04^{2}+0.06^{2}}}{\sqrt{3^{2}+14^{2}}}\approx0.0050
 $$
 \item<2-> Por lo que respecta al vector soluci\'on, ese cambio relativo en la norma eucl\'idea es
 $$
 \dfrac{\|\delta x\|_2}{\|x\|_2} = \dfrac{\sqrt{(1-0.993)^{2}+(1-0.9968)^{2}}}{\sqrt{1^{2}+1^{2}}}\approx0.0054
 $$
 \item<3-> Como se puede ver, un peque\~no cambio en el vector $b$ induce un cambio peque\~no en el vector soluci\'on.
\end{itemize}
}
%%%%%
\frame{

\begin{itemize}
 \item<1-> Introduciendo el mismo cambio, $[-0.04;-0.06]^{T}$, en el t\'ermino independiente del segundo sistema, su soluci\'on pasa a ser $[6; 0]^{T}$
 \item<2-> Es decir, un cambio relativo en la norma eucl\'idea de $c$ igual a
 $$
 \dfrac{\|\delta c\|_2}{\|c\|_2} = \dfrac{\sqrt{0.04^{2}+0.06^{2}}}{\sqrt{4^{2}+12^{2}}}\approx0.0057
 $$
 produce un cambio en el vector soluci\'on igual a:
 $$
 \dfrac{\|\delta x\|_2}{\|x\|_2} = \dfrac{\sqrt{5^{2}+1^{2}}}{\sqrt{1^{2}+1^{2}}}\approx3.6055
 $$
 \item<3-> Evidentemente, el segundo sistema es mucho m\'as sensible a cambios en el t\'ermino independiente que el primero.
\end{itemize}

}
%%%%%
\frame{
\begin{center}
 \begin{tabular}{cc}
 %\includegraphics[scale=0.25]{sis1.jpg} & \includegraphics[scale=0.25]{sis2.jpg}
\end{tabular}
\end{center}

\begin{itemize}
 \item<1-> La figura representa a la izquierda el primer sistema; el segundo a la derecha.
 \item<2-> Como se puede apreciar, las dos rectas que representan las ecuaciones del primer sistema se cortan n\'itidamente en el punto $[1; 1]^{T}$.
 \item<3-> En el caso del segundo sistema, aun usando una resoluci\'on gr\'afica mayor, apenas se diferencian las dos rectas y mucho menos d\'onde se cortan.
\end{itemize}


}
%%%%%
\frame
{
El sistema est\'a bien condicionado si cuando $\delta b$ es peque\~na, $\delta u$ tambi\'en lo es. Observese que:

\uncover<2->{

$$
\left.\begin{array}{ccc}
 Au +A(\delta u) & = & b + \delta b\\
 Au & = & b
\end{array}\right\}\Rightarrow\left\{\begin{array}{ccc}
 A(\delta u) & = & \delta b\\
 \delta u & = & A^{-1}(\delta b)
\end{array}\right.
$$
}

\uncover<3->{
Usando la propiedad para normas matriciales:

$$
\|\delta u\| \leq \|A^{-1}\|\|\delta b\|
$$
}

\uncover<4->{
De la soluci\'on exacta, $\|b\| \leq \|A\|\|u\|$ , lo que implica que:

$$
\frac{1}{\|u\|}\leq\frac{\|A\|}{\|b\|}
$$
}
}
%%%%
\frame
{
De las dos relaciones anteriores se llega a que:

$$
\frac{\|\delta u\|}{\|u\|} \leq \|A\|\|A^{-1}\|\frac{\|\delta b\|}{\|b\|}
$$

donde $\frac{\|\delta u\|}{\|u\|}$ representa el error relativo en los resultados, y $\frac{\|\delta b\|}{\|b\|}$ el error relativo en los datos.

\uncover<2->{
De la relaci\'on, parece deducirse que el n\'umero $\|A\|\|A^{-1}\|$ es el factor determinante de la relaci\'on, ya que si es peque\~no se tendr\'a el efecto deseado, y si no, ocurre lo contrario.
}
}
%%%%%
\frame
{
\begin{block}{Definici\'on}
Sea $\| \cdot \|$ una norma matricial subordinada y $A$ una matriz invertible. Se denomina n\'umero de condici\'on de la matriz $A$ respecto de la norma $\| \cdot \|$ a la expresi\'on:
$$
\kappa(A) = \|A\|\|A^{-1}\|
$$
\end{block}
}
%%%%
\frame
{
\begin{block}{Ejemplo}
 Estudiar el condicionamiento del sistema $Ax = b$ con $A=\left(\begin{array}{cc}
                                                                 1 & 1+\varepsilon\\
								                                 1-\varepsilon & 1\\
                                                                \end{array}\right)$ siendo $\varepsilon>0$ en la norma $\|\cdot\|_{\infty}$
\end{block}

\uncover<2->{

$$
\|A\|_{\infty} = 2+\varepsilon \qquad \|A^{-1}\|_{\infty} = \dfrac{2+\varepsilon}{\varepsilon^2}
$$
}

\uncover<3->{
luego

$$
\kappa(A,\infty) = \dfrac{(2+\varepsilon)^2}{\varepsilon^2}>\dfrac{4}{\varepsilon^2}
$$
}

\uncover<4->{
Si $\varepsilon \leq 0.01$ entonces $\kappa(A,\infty)>40000$.  Esto indica que una perturbaci\'on de los datos de $0.01$ puede originar una perturbaci\'on de la soluci\'on del sistema de 40000.}

}
%%%%%
\frame{
\begin{itemize}
 \item Si los datos de un sistema $Ax=b$ son exactos con la precisi\'on de la m\'aquina, el error relativo de la soluci\'on cumple que
 $$
 \dfrac{\|x^{*}-x\|}{\|x\|} \leq \kappa(A)\epsilon
 $$
 \item<2-> El concepto de n\'umero de condici\'on de una matriz se generaliza a cualquier matriz $A$ (no necesariamente cuadrada) de rango completo mediante la expresi\'on
 $$
 \kappa(A) = \|A\|\|A^{\dagger}\|
 $$
 donde $A^{\dagger}$ es la matriz pseudoinversa de la matriz $A$.
 \item<3-> El n\'umero de condici\'on de una matriz $A$ es un indicador del error de amplificaci\'on que produce en un vector $x$ el someterlo a la transformaci\'on que define dicha matriz $A$.
\end{itemize}

}
%%%%%
\frame{
\begin{itemize}
 \item<1-> Estudiando la sensibilidad de un sistema de ecuaciones a peque\~nas perturbaciones en los coeficientes de la matriz.
 \item<2-> Comparando la soluci\'on de
 $$
 Ax=b \qquad \mbox{ y } \qquad (A+\delta A)(x+\delta x)=b
 $$
 \item<3-> De la segunda igualdad, como $Ax=b$, haciendo $\delta x = -A^{-1}\delta A(x+\delta x)$, despreciando el producto $\delta A\cdot\delta x$, resulta que
 $$
 \|\delta x\| \leq \|A^{-1}\|\|\delta A\|\|x\|
 $$
\end{itemize}
}
%%%%%
\frame{
\begin{itemize}
 \item<1-> Expresi\'on que tambi\'en se puede escribir como
 $$
 \dfrac{\|\delta x\|}{\|x\|} \leq \|A^{-1}\|\|A\|\dfrac{\|\delta A\|}{\|A\|}
 $$
 \item<2-> As\'i pues, el error relativo que resulta de perturbar ligeramente los coeficientesde la matriz del sistema $Ax=b$ est\'a tambi\'en acotado en t\'erminos del n\'umero de condici\'on de la matriz $A$.
\end{itemize}
}
%%%%%
\frame{
\begin{block}{Teorema: Para toda matriz $A$ de rango completo}
\begin{enumerate}
 \item<1-> Su n\'umero de condici\'on $\kappa(A) \geq 1$.
 \item<2-> $\kappa(A) = \kappa(A^{\dagger})$.
 \item<3-> $\kappa(\alpha A) = \kappa(A)$ para todo escalar $\alpha \neq 0$.
 \item<4-> $\kappa_2(A) = \dfrac{\sigma_n(A)}{\sigma_1(A)}$, donde $\sigma_n$ y $\sigma_1$ son, respectivamente, los valores singulares mayor y menor de $A$.
 \item<5-> $\kappa_2(A) = \dfrac{\displaystyle\max_{i}|\lambda_i(A)|}{\displaystyle\min_{i}|\lambda_i(A)|}$, si $A$ es sim\'etrica.
 \item<6-> $\kappa_2(A^{T}A)=\kappa_2^{2}(A)$
 \item<7-> Su n\'umero $\kappa(A)=1$ si la matriz es la identidad o se trata de una matriz ortogonal.
 \item<8-> Su n\'umero de condici\'on $\kappa_2(A)$ es invariante frente a transformaciones ortogonales.
\end{enumerate}
\end{block}
}
%%%%
\frame{
\begin{itemize}
 \item<1-> Las matrices con n\'umeros de condici\'on peque\~nos (pr\'oximos a la unidad), se dicen 
bien condicionadas; las que tienen n\'umeros de condici\'on alto, mal condicionadas.
 \item<2-> Los distintos n\'umeros de condici\'on de una matriz $A \in \mathbb{R}^{n\times n}$ 
asociados con las normas matriciales m\'as habituales cumplen que
 \begin{eqnarray}
\nonumber  \kappa_2(A)/n & \leq \kappa_1(A)  \leq &n\kappa_2(A)\\
\nonumber  \kappa_{\infty}(A)/n &\leq  \kappa_2(A) \leq  &n\kappa_{\infty}(A)\\
\nonumber  \kappa_1(A)/n^{2} & \leq \kappa_{\infty}(A)  \leq & n^{2}\kappa_1(A)
 \end{eqnarray}
\end{itemize}
}
%%%%%
\frame{
\frametitle{Ejemplo:}
Retomando el ejemplo planteado, La matriz

$$
A=\left[\begin{array}{cc}
                8 & -5\\
                4 & 10
               \end{array}\right]
$$
\uncover<2->{
cuya inversa es
$$
A^{-1}=\left[\begin{array}{cc}
                0.10 & 0.05\\
                -0.04 & 0.08
               \end{array}\right]
$$}
\uncover<3->{
tiene un n\'umero de condici\'on
$$
\kappa_1(A) = \|A\|_1\|A^{-1}\|_1 = 15 \cdot 0.14 = 2.1
$$}
}
%%%%%
\frame{
\frametitle{Ejemplo:}
\uncover<1->{
El de la matriz 
$$
B=\left[\begin{array}{cc}
                0.66 & 3.34\\
                1.99 & 10.01
               \end{array}\right]
$$}
\uncover<2->{
cuya inversa es
$$
B^{-1}=\left[\begin{array}{cc}
                -250.25 & 83.5\\
                49.75 & -16.5
               \end{array}\right]
$$}
\uncover<3->{
es 
$$\kappa_1(B) = \|B\|_1\|B^{-1}\|_1 = 13.35 \cdot 300 = 4005$$
tres \'ordenes de magnitud superior.
}
}
%%%%
\frame{
\begin{itemize}
 \item Un error que se comete con frecuencia es asimilar el concepto de n\'umero de condici\'on de una matriz con el de 
su determinante y que, en ese sentido, a mayor determinante, mayor n\'umero de condici\'on; nada m\'as lejos de la 
realidad.
\item<2-> Ejemplo 1: Sea $A$ una matriz diagonal de orden 100 definida por
$$
a_{11} = 1 ; \quad a_{ii}=0.1, \quad 2\leq i\leq n
$$
De esta matriz
$$
\|A\|_2 = 1 \qquad \mbox{ y } \qquad \|A^{-1}\|_2 = 10
$$
\item<3-> El n\'umero de condici\'on $\kappa_2(A)=10$. Por el contrario, su determinante es 
$\det(A)=1*(0.1)^{99}=10^{-99}$
\end{itemize}
}
%%%%%
\frame{
\begin{itemize}
 \item Ejemplo 2: Sea $A$ una matriz bidiagonal de la forma
 $$
 A=\left[\begin{array}{ccccc}
          1 & 2 & & & \\
           & 1 & 2 & & \\
           & & \ddots & \ddots & \\
           & & & 1 & 2 \\
            & & & & 1
         \end{array}
\right]
 $$
 su inversa es 
  $$
 A^{-1}=\left[\begin{array}{ccccc}
          1 & -2 & 4 &\cdots &(-2)^{n-1} \\
           & 1 & -2 & &(-2)^{n-2} \\
           & & 1 &  & \vdots\\
           & & & \ddots & \vdots \\
            & & & & 1
         \end{array}
\right]
 $$
\end{itemize}
}
%%%%%
\frame{
\begin{itemize}
 \item Sus diversas normas son $\|A\|_{\infty}=\|A\|_1=3$ y $\|A^{-1}\|_{\infty}=\|A^{-1}\|_1= 
1+2+4+\cdots+2^{n-1}=2^n-1$
\item<2-> Sus n\'umeros de condici\'on $\kappa_{\infty}(A)=\kappa_1(A)\approx3\cdot2^n$. 
\item<3-> Mientras qu su determinante es 1. 
\end{itemize}
}
%%%%%
\frame{
\frametitle{Ejercicios:}
Dado el sistema lineal
$$
 \left\{\begin{array}{rcl}
         1.01x+0.99y & = & 2\\
         0.99x+1.01y & = & 2
        \end{array}\right.
 $$
 Calcule
 \begin{enumerate}
  \item La soluci\'on exacta del problema.
  \item La soluci\'on usando solamente dos cifras decimales y redondeo.
  \item La inversa de la matriz de coeficientes $A^{-1}$ con dos cifras decimales y redondeo.
  \item El n\'umero de condici\'on de $A$ para el apartado anterior.
  \item El residuo obtenido en el segundo item.

 \end{enumerate}

}
%%%%%
\frame{
\frametitle{Ejercicios:}
Dado el sistema lineal perturbado
$$
 \left\{\begin{array}{rcl}
         1.02x+0.99y & = & 2.01\\
         0.99x+1.01y & = & 2.02
        \end{array}\right.
 $$
 Calcule
\begin{enumerate}
\item La soluci\'on exacta del problema.
  \item La soluci\'on usando solamente dos cifras decimales y redondeo.
  \item La inversa de la matriz de coeficientes $A^{-1}$ con dos cifras decimales y redondeo.
  \item El n\'umero de condici\'on de $A$ para el apartado anterior.
  \item El residuo obtenido en el segundo item y el error absoluto.
\end{enumerate}
}
%%%%
\frame{
\frametitle{Ejercicios:}
 La matriz de Hilbert
 $$
 H_n = \left(\dfrac{1}{i+j-1}\right)_{i,j=1}^{n} = \left(\begin{array}{cccc}
                                                          1 & 1/2 & \cdots & 1/(n-1)\\
                                                          1/2 & 1/3 & \cdots & 1/n\\
                                                          \vdots & \vdots & \ddots & \vdots\\
                                                          1/(n-1) & 1/n & \cdots & 1/(2n-1)
                                                         \end{array}
\right)
 $$
 es un ejemplo cl\'asico de una matriz mal condicionada. En MATLAB se puede construir f\'acilmente usando el comando \texttt{hilb(n)}.
\begin{enumerate}
 \item Escriba un script de MATLAB para calcular los valores de $\kappa_2(H_n)$ para $n = 1, 2, \ldots, 10$. Dibuje los resultados en escala logar\'itmica (comando \texttt{semilogy}).
 \item  Estime la dependencia del n\'umero de condici\'on de la matriz de Hilbert de sus dimensiones. ?`De qu\'e tipo de crecimiento se trata, lineal, exponencial, etc.?
\end{enumerate}
}
%%%%%
\frame{
\frametitle{Ejercicios:}
\begin{enumerate}
 \item Para la matriz de Hilbert $H_{10}$ aplique los criterios
 $$
 |\det(H_{10})\det(H_{10}^{-1})-1|, \quad \|(H_{10}^{-1})^{-1}-H_{10}\|_2, \quad \|H_{10}H_{10}^{-1}-I_{10}\|_2
 $$
para detectar la mala condici\'on de la misma.

\end{enumerate}

}
\end{document}