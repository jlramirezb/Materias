\documentclass[12pt,letterpaper]{article}
\usepackage[spanish]{babel}
\usepackage{amsfonts,amssymb,amsthm,amsmath,color,graphicx,bm}
\usepackage[left=1cm,right=1cm,bottom=2cm,top=2cm]{geometry}
\usepackage{newlfont}
\usepackage[latin1]{inputenc}
\usepackage{float}
\usepackage{multirow}
\thispagestyle{empty} \pretolerance=2000 \tolerance=3000
\linespread{1.3}
\newcommand{\sen}{\ensure{sen}}
\newcommand{\ds}{\displaystyle}
\usepackage[spanish,ruled,vlined]{algorithm2e}

\begin{document}

\begin{tabular}[h]{lr}
\includegraphics[width=1.5cm,height=1.5cm]{LogoUC.jpg} & \hspace{14cm}
\includegraphics[width=2cm,height=1.5cm]{LogoFaCyT.jpg}
\end{tabular}

\vspace{-2.5cm}
\begin{center}
{ Universidad de Carabobo \\
 \vspace{-3mm}
  Facultad Experimental de Ciencias y Tecnolog\'ia \\
 \vspace{-3mm}
Departamento de Matem\'aticas}
\\
 \vspace{1cm}

Quiz Pr\'actico II\\
M\'etodos Num\'ericos II.
\end{center}
\vspace{1cm}

\begin{enumerate}

\item Considere el sistema de ecuaciones lineales $Ax = b$, donde $A \in M^{n \times n}(\mathbb{R})$, $x,b
\in \mathbb{R}^n$. Adem\'as, $A$ y $b$ est\'an definidas por:

$$
A=\left(\begin{array}{rrrrrr}
         4      & -1     & 0      & 0      & \cdots & 0\\
         -1     &  4     & -1     & 0      & \cdots & 0\\
         0      & -1     & 4      & -1     & \cdots & 0\\
         \vdots & \vdots & \ddots & \ddots & \ddots & \vdots\\
         0      & \cdots & 0      & -1     & 4      & -1\\
         0      & 0      & \cdots & 0      & -1     & 4\\     
        \end{array}\right)
$$

y $b_j = \cos(2\pi j/n)$. Resuelva este sistema para $n=10$, $n=100$ y $n=1000$ usando:
\begin{enumerate}
 \item Factorizaci\'on de Cholesky.
 \item Iteraci\'on de Jacobi.
 \item Iteraci\'on de Gauss-Seidel.
\end{enumerate}
Elija el residuo relativo $\|x_{n+1}- x_n\|_2/\|x_n\|_2 \leq 10^{-3}$ como criterio de pararda en los m\'etodos iterativos.
Elabore un cuadro donde se muestre el tiempo invertido en cada caso. Comente los resultados observados.

\item El perfil de temperaturas $T(x)$ de una barra circular delgada, de radio despreciable y longitud unitaria obedece
a la ecuaci\'on:
$$
\frac{\partial^2T(x)}{\partial x^2} = f(x)
$$
donde $f(x) = 10x^2 + 5x + 10,\,\, 0 \leq x \leq 1$ representa la raz\'on de calor radiante a lo largo de la barra y
$T(0) = 300K$ y $T(1) = 400K$ son las conocidas temperaturas al final de la barra. Una soluci\'on aproximada de esta
ecuaci\'on puede ser encontrada usando el m\'etodo de diferencias finitas. Considere una partici\'on del intervalo en $n
+ 1$ subintervalos con $n + 2$ puntos denotados por $x_0 = 0, x_1 = h, \ldots, x_i = ih, \ldots, x_{n+1} = 1$ donde
$h(n + 1) = 1$ y la temperatura en cada punto es denotada por $T_i = T(x_i)$. El esquema de diferencias finitas es dado
por
$$
\frac{\partial^2T(x)}{\partial x^2} \approx \frac{T_{i-1}-2T_i+T_{i+1}}{h^2}
$$
\begin{enumerate}
 \item Establecer el sistema lineal $Ax = b$ que resulta de utilizar el m\'etodo de diferencias finitas.
 \item Pruebe que la matriz $A$ es diagonal dominante.
 \item Resuelva el sistema para obtener soluciones correspondientes a $h =\frac{1}{6},h =\frac{1}{11},h =
\frac{1}{21}, h =\frac{1}{41},h =\frac{1}{61}$, usando el m\'etodo $SOR$ con diferentes valores en el par\'ametro
$\omega$. Establezca, si es posible, una vinculaci\'on entre \'este y $h$.
\end{enumerate}

\item En numerosas ocasiones la matriz $A$ del sistema $Ax = b$ es mal condicionada. Una estrategia para resolver
estos sistemas es encontrar una matriz $P$ tal que $P^{-1}A$ sea bien condicionada y luego resolver el sistema
equivalente
$$
P^{-1}Ax=P^{-1}b
$$
La matriz $P$ se denomina precondicionador a la izquierda de $A$. Un requerimiento es que $P$ sea f\'acil de invertir.
Existen numerosos precondicionadores usados en la pr\'actica, de ellos tomar la diagonal de $A$, es decir:
$$
P = D = diag(a_1 ,\ldots, a_n),\qquad \mbox{(Precondicionador de Jacobi)}
$$
resulta generalmente efectivo cuando $A$ es sim\'etrica y definida positiva y con elementos diagonales no todos
iguales. El algoritmo correspondiente al m\'etodo del Gradiente Conjugado Precondicionado es el siguiente:

%\incmargin{1em}
\RestyleAlgo{boxed}
%\linesnumbered
\begin{algorithm}[ht]
 $x_0$: dato inicial\\
 $r_0=b-Ax_0$\\
 Resolver $Pz_0=r_0$\\
 $p_0=z_0$\\
 \For{$k\leftarrow 0$ \KwTo $N$}
 {
  $\alpha_k=\displaystyle\frac{r_k^tz_k}{p_k^tAp_k}$\\
  $x_{k+1}=x_k+\alpha_kp_k$\\
  $r_{k+1}=r_k-\alpha_kAp_k$\\
  Resolver $Pz_{k+1}=r_{k+1}$\\
  $\beta_{k+1}=\displaystyle\frac{z_{k+1}^tr_{k+1}}{z_k^tr_k}$\\
  $p_{k+1}=z_{k+1}+\beta_{k+1}p_k$\\
 }
\end{algorithm}
%\decmargin{1em}
\begin{enumerate}
 \item Haga un programa para resolver el problema $Ax = b$ por el m\'etodo del gradiente conjugado precondicionado con
Jacobi.
\item Pruebe su programa con la matriz $A$ definida, para $i,j \in \{1,\ldots, n\}$, por
$$
A(i,j) = \left\{\begin{array}{ll}
                 i+j & \mbox{ si } 0<|i-j|<3\\
                 (i+j)^2 &  \mbox{ si } i=j\\
                 0 & \mbox{ en cualquier otro caso}
                \end{array}
\right.
$$
y $b(i)=i$.
\item Realizar una comparaci\'on en n\'umero de iteraciones de los m\'etodos Gradiente Conjugado y Gradiente Conjugado
precondicionado para la matriz dada considerando distintos valores de $n$, digamos $n = 5, 10, 20, 30, 50$, etc.
\end{enumerate}

\item Actualmente google se ha establecido como la p\'agina de b\'usqueda en internet m\'as utlizada. Esto ha sido en gran parte gracias
al novedoso algoritmo que esta compa\~nia utiliza para, una vez encontradas las p\'aginas que cumplen con el criterio de b\'usqueda de
un usuario, escoger el orden en que \'estas ser\'an presentadas. Para escoger este orden google asocia a cada p\'agina un n\'umero, llamado
\textit{pagerank}, que calcula teniendo en cuenta el n\'umero de enlaces desde y hacia cada una de las p\'aginas web en el mundo. El orden
en que se presentan las p\'aginas depende de su \textit{pagerank} (las de mayor \textit{pagerank} se presentan primero). Para calcular el
\textit{pagerank} de cada p\'agina web, google crea una matriz $G$ que tiene una fila y una columna por cada p\'agina web, las entradas de
esta matriz son:
$$
G(i,j) = \left\{\begin{array}{ll}
                  1 & \mbox{si hay un enlace de p\'agina $j$ a p\'agina $i$}\\
		  0 & \mbox{en caso contrario}
                \end{array}
\right.
$$

El \textit{pagerank} de una p\'agina depende no s\'olo del n\'umero de p\'aginas que contengan enlaces a ella, sino tambi\'en del
\textit{pagerank} de esas p\'aginas. As\'i una p\'agina que sea referenciada en muchas p\'aginas de poca importancia podr\'a tener menos
importancia que una que sea referenciada en pocas p\'aginas de mucha importancia. Si $x$ es un vector con el \textit{pagerank} de cada una
de las p\'aginas web del mundo, google lo encuentra resolviendo el siguiente sistema de ecuaciones lineales

\begin{equation}\label{SEL}
  Ax=e, \qquad A=I-pGD
\end{equation}

d\'onde $I$ es la matriz de identidad, $p$ es la probabilidad de que una persona, en su b\'usqueda de alguna informaci\'on en internet,
siga uno de los enlaces en la p\'agina donde se encuentra y se toma igual a $0.85$, $G$ es la matriz mencionada antes, $D$ es una matriz
diagonal tal que $D(i,i)$ es el n\'umero de enlaces en la p\'agina $i$ y $e$ es el vector que contiene todas sus entradas iguales a
1. Encontremos experimentalmente, de entre todos los algoritmos que hemos estudiado para la soluci\'on de sistemas de ecuaciones lineales,
cu\'al es el m\'as conveniente para resolver este sistema de ecuaciones.
\begin{enumerate}
  \item Guarde en su directorio de trabajo el contenido del archivo \texttt{lab2\_m.zip}. En \'el encontrar\'a \texttt{data.mat},
\texttt{creatematrix\_pagerank.m} y \texttt{mostrarrangos.m}.
\item Cargue las variables almacenadas en \texttt{data.mat} en Matlab. Para ello debe escribir en los comandos de Matlab

\texttt{$>>$ load data.mat}

\texttt{$>>$ whos}

y observe que la matriz $G$ y el vector $U$ se han cargado en memoria (\'estas eran las variables guardadas en \texttt{data.mat}).
La matriz $G$ es como la matriz mencionada antes, pero se cre\'a suponiendo que la web est\'a formada s\'olo por 500 p\'aginas. La variable
$U$ contiene los nombres de estas p\'aginas. Llame ahora a $A$ \texttt{= creatematrix\_pagerank(}$G$\texttt{)} para construir la matriz $A$
en (\ref{SEL}).

\item Con \texttt{spy(A)} observe la estructura de $A$. ?`Es una matriz dispersa? ?`Es sim\'etrica? ?`Cu\'antas de sus entradas son
distintas de cero? ?`Ser\'a conveniente para google usar un m\'etodo directo para encontrar el \texttt{pagerank} de cada p\'agina en
internet? ?`Por qu\'e?

\item Busque una aproximaci\'on a la soluci\'on exacta a este sistema de ecuaciones con los m\'etodos iterativos que usted conoce.
?`Cu\'ales m\'etodos iterativos le permitieron obtener una aproximaci\'on a la soluci\'on exacta de (\ref{SEL})? ?`Cu\'al tolerancia us\'o?
?`En cu\'antas iteraciones alcanz\'o cada uno de ellos la precisi\'on requerida? ?`Cu\'al es la diferencia entre las soluciones obtenidas?
Si alguno de los m\'etodos no le permite obtener una aproximaci\'on con la precisi\'on requerida, ?`a qu\'e se debe esto?

\item Una vez encontrada una aproximaci\'on a la soluci\'on exacta de (\ref{SEL}) usted podr\'a, llamando a la funci\'on
\texttt{mostrarrangos}, ver un c\'odigo de barras con los rangos de las 10 p\'aginas que tienen mayor \textit{pagerank}, as\'i como una
lista con sus direcciones, n\'umero de enlaces en y hacia ellas y el \textit{pagerank} de cada una.
\end{enumerate}

\end{enumerate}

\textbf{NOTA}: Se deben tomar a consideraci\'on las siguientes observaciones.
\begin{itemize}
  \item Entregar los archivos .m necesarios para la ejecuci\'on correcta de cada pregunta en funci\'on de lo exigido.
\item Elaborar cada pregunta por separado y se debe indicar cual es el script principal de la pregunta.
\item Elaborar un peque\~no documento en latex que contenga los resultados te\'oricos exigidos asi como la definici\'on de las rutinas de cada pregunta, sea organizado.
\item Subir al classroom los archivos para su evaluaci\'on.
\end{itemize}
\end{document}