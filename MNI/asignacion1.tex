%------------------------------------------------------------------------
% Prueba Matematica I
%
%       Realizado por:
%       Jose Luis Ramirez
%       Lennys Bello
%                                                Actualizado: Noviembre 2007
% ------------------------------------------------------------------------
\documentclass[12pt,letterpaper]{article}
\usepackage[left=3.0cm,top=3.0cm,right=2.0cm,bottom=2.0cm]{geometry}
%\usepackage[activeacute,spanish]{babel}
\usepackage{amsmath}
\usepackage{graphicx, color}
\usepackage{amsfonts}
\usepackage{amssymb}
\usepackage{amsthm}
\usepackage{amsxtra}
\usepackage{amsgen}
\usepackage{amscd}
\usepackage{rotating}
\usepackage{multirow}
\usepackage{latexsym}
\usepackage{fancyhdr}
\pagestyle{empty}
\begin{document}
%%%%%%%%%%%%%%%%%%%%%%%%%%%%%%%%%%%%%%%%%%%%%%%%%%%%%%%%%%%%%%%%%%%%%%%%%%%%%%%%%%%%%%%%%%%%%%%%%%%%%%%%%
\begin{minipage}[ht]{3cm}
\includegraphics[width=2.5cm]{Logouc.jpg}
\end{minipage}
\begin{minipage}{10cm}
\centering\textbf{Universidad de Carabobo.\\
 Facultad de Ciencias y Tecnolog\'ia.\\
\'Departamento de Matem\'atica.\\
 M\'etodos Num\'ericos I.\\
}
\end{minipage}
\begin{minipage}[ht]{4cm}
\includegraphics[width=2.5cm]{facyt.jpg}
\end{minipage}

\begin{center}
\underline{\textbf{Asignaci\'on 1}}\\[7pt]
\end{center}

\vspace{0.5cm}

\begin{enumerate}
\item El valor de la intersecci\'on, de la recta que pasa por los puntos $(x_0,y_0)$ y $(x_1,y_1)$, con el
eje $x$ se puede encontrar mediante las expresiones

$$
x = \frac{x_0y_1-x_1y_0}{y_1-y_0} \qquad \textrm{y} \qquad x = x_0 - y_0\frac{x_1-x_0}{y_1-y_0}
$$

Muestre que ambas f\'ormulas son algebraicamente correctas.

Usando los puntos $(x_0,y_0)=(1.31;3.24)$ y $(x_1,y_1)=(1.93;4.76)$ y una aritm\'etica de redondeo correcto a
tres d\'igitos, calcule la intersecci\'on de la recta con el eje $x$ con ambas expresiones. Compare
los resultados obtenidos con el valor exacto. (Recuerde aplicar redondeo correcto en cada operaci\'on aritm\'etica)



\item Hallar el \'epsilon de la m\'aquina para un computador personal, usando el agoritmo visto en clase. Para ello use como lenguaje de programaci\'on Matlab. 

\item Determinar el n\'umero positivo $x$ m\'as peque\~no de la forma $x = 2^{-k}$ ($k$ entero no negativo) tal que $10^5 +x \neq 10^{5}$. Note que este enunciado es id\'entico al que define al \'epsilon de la m\'aquina, con la \'unica diferencia que se ha reemplazado la condici\'on $1 + x = 1$ por $10^5 + x = 10^5$. Compare con el resultado obtenido en el enunciado anterior. ?`Qu\'e se puede concluir? Justifique su respuesta.

\item Elabore un algoritmo recursivo en Matlab que permita obtener el valor $\displaystyle x_n=\frac{1}{3^n}$ mediante la f\'ormula definida por:

$$
x_{n+1} = Ax_n + \left(\frac{1-3A}{9}\right)x_{n-1}
$$

con los valores iniciales $x_0=1.0$ y $x_1=1/3$. Siendo $A$ un dato de entrada. Muestre como el algoritmo se hace inestable a medida que el valor $A$ es grande, mientras que si $0<A<1$ el algoritmo es estable. Realice diversas pruebas y muestre una tabla donde se compare el valor exacto y su aproximaci\'on, y as\'i mismo calcule y muestre para cada uno su error absoluto y relativo.

\item Suponga que $fl(y)$ es una aproximaci\'on de $y$ con un redondeo a $k$ cifras. Demuestre las cotas para el error absoluto $E_a$ y el error relativo $E_r$ si se usa una t\'ecnica de aproximaci\'on por truncamiento o por redondeo correcto.

\item Escriba un c\'odigo en Matlab que genere mil datos aleatorios de orden $10^{-5}$, y que calcule $a = 10^{12} + x_1 + \cdots
+ x_{1000}$ y $b = x_1 + \cdots + x_{1000} + 10^{12}$ ?`Son $a$ y $b$ iguales? De no ser as\'i, explique por qu\'e y diga cu\'al es el m\'as exacto.

\item Sea $\displaystyle f(x)=\frac{1-\cos(x)}{x^2}$
\begin{enumerate}
 \item Demuestre que $0 \leq f(x) \leq 0.5$ para todo $x \in \mathbb{R} - \{0\}$. Adem\'as, demuestre que es posible extender la definici\'on de $f(x)$ a todo $\mathbb{R}$. Luego, grafique la funci\'on en el intervalo $[-3, 3]$.
\item Eval\'ue la funci\'on $f(x)$ en el punto $x = 1.2 \times 10^{-8}$. ?`Qu\'e observa?, ?`es confiable este resultado?, ?`por qu\'e? Explique.
\item Usando el hecho de que $\cos(x) = 1-2\sin^2(x/2)$, la funci\'on dada se puede reescribir como $f(x) = \displaystyle\frac{1}{2}\left(\frac{\sin(x/2)}{x/2}\right)^2$ . Eval\'ue de nuevo en el punto $x = 1.2 \times 10^{-8}$, esta vez usando la nueva expresi\'on de $f(x)$. ?`Qu\'e obtuvo? Analice el resultado y explique lo que sucede.


\end{enumerate}

% \item Convertir $0.3_{10}$ a binario y hallar su representaci\'on en IEEE precisi\'on simple.

% \item ?`Qu\'e n\'umero decimal representa el siguiente patr\'on de bits en IEEE precisi\'on simple?
% $$
% 0\,\, 00001100\,\, 01000000000000000000000
% $$

% \item Se tiene procesador un NORM-32 que tiene una longitud de palabra de 32 bits (1bit = 1Binary digital), estos se distribuyen de la manera siguiente: donde los dos primeros espacios son reservado para los signos, asign\'andole cero si el signo es positivo y uno si es negativo, los siguientes siete espacios para el exponente y los restantes para la mantisa y dado que un n\'umero real distinto de cero $x = \pm q\times2^{m}$ siempre puede normalizarse de tal manera que $\frac{1}{2}\leq q<1$, podemos suponer que el primer bit en $q$ es 1, y por lo tanto no requiere almacenamiento. Representar y almacenar en punto flotante normalizado $117.125$.

\item Sea $\mathbb{F}$ el sistema de punto flotante caracterizado por $\beta = 2$,(base), $n = 4$(precisi\'on), $m = -1$, $M = 2$, cada n\'umero en el conjunto $\mathbb{F}$ est\'a representado por $\pm(.d_1d_2 \ldots d_n)_{\beta}\beta^{e}$ donde $m \leq e \leq M$
\begin{enumerate}
 \item ?`Cu\'al es el n\'umero m\'as peque\~no en valor absoluto del sistema $\mathbb{F}$?
 \item Demuestre que 3/4 y 5/16 pertenecen al sistema $\mathbb{F}$, pero la suma ``verdadera'' de estos no pertenece a 
$\mathbb{F}$.
 \item Suponga que el tipo de error introducido en la representaci\'on de un n\'umero real en el sistema $\mathbb{F}$ es por redondeo. Como queda representado el numero 3/4 + 5/16 en $\mathbb{F}$. esto es:
 $$
 \frac{3}{4}\oplus\frac{5}{16} = fl(\frac{3}{4}+\frac{5}{16})
 $$
 \item Encuentre el \'epsilon de la m\'aquina.
\end{enumerate}

\end{enumerate}
\end{document}
