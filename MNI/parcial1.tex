\documentclass[12pt]{article}
\usepackage[spanish]{babel}
\usepackage{amsfonts,amssymb,amsthm,amsmath,color,graphicx,bm,fancyhdr}
\usepackage[left=1cm,right=1cm,bottom=1cm,top=1cm]{geometry}
\pretolerance=2000 \tolerance=3000 \linespread{1.0}
\pagestyle{fancy}
\newcommand{\er}{\ensuremath{I\!\!R}}
\newcommand{\sen}{\ensure{sen}}


\begin{document}

    \thispagestyle{empty}

    %\textcolor{blue}

    \begin{center}
        \begin{figure}[!h]
        \hspace{2cm}
        \includegraphics[scale=1]{LogoUC.jpg}
        \label{fig:hoo}
        \hspace{11cm}\includegraphics[scale=0.3]{FaCyT.jpg}
        \end{figure}
        \vspace{-2,5cm}
        Universidad de Carabobo \\ Facultad Experimental de Ciencias y
        Tecnolog\'{\i}a \\ Departamento de Matem\'{a}ticas\\ Asignatura:
        M\'etodos Num\'ericos I\\
        \vspace{1cm} {\it\bf\large Parcial I}\\
    \end{center}

    \begin{enumerate}
        \item Demostrar que si el n\'umero conocido $\tilde x$ aproxima al desconocido $x$ con $|x - \tilde x|<\varepsilon$, 
        entonces $\displaystyle\frac{1}{\tilde x}$ aproxima a $\displaystyle\frac{1}{x}$ con
        $$
        |Error \, relativo| < \displaystyle\frac{\varepsilon}{|\tilde x|}
        $$
        \begin{flushright}
         {\small \it (valor: 3 pts)}
        \end{flushright}

        \item ?`Con cuantas cifras significativas se puede decir que $p^*$ aproxima a 3000 si se sabe que $2999 < p^*<3001$? 
        Razone su respuesta. 
        \begin{flushright}
         {\small \it (valor: 3 pts)}
        \end{flushright}

        \item Sea $x_A = -0.9989$. Suponga que dispone de un computador hipot\'etico que utiliza aritm\'etica decimal de 
redondeo correcto a 4 d\'igitos. Considere las expresiones equivalentes
$$
u_A = (1-x_A^2)(1+x_A^2) \mbox{ y } v_A = 1-x_A^4
$$
\begin{enumerate}
 \item ?`Qu\'e valor calcular\'ia dicho computador para $u_A$ y $v_A$?
 \item ?`Cu\'al de los valores obtenidos en ($a$) es m\'as exacto?
\end{enumerate}
\begin{flushright}
 {\small \it (valor: 3 pts)}
\end{flushright}

\item Sea $I_{n}={\displaystyle \int_{0}^{1}\frac{x^{n}}{x+4}dx}$

\begin{enumerate}
\item Probar que $I_{0}=\ln(5/4)$, que $0\leq I_{n}\leq I_{0}$ y que se
verifica $I_{n+1}+4I_{n}=\frac{1}{n+1}$. Es decir, $I_{n+1}=-4I_{n}+\frac{1}{n+1}$.
\begin{flushright}
 {\small \it (valor: 3 pts)}
\end{flushright}
\item Con la ecuaci\'on $I_{n+1}=-4I_{n}+1/(n+1)$, $I_{0}=\ln(5/4)$
se pretende calcular $I_{n}$, sin embargo al representar en la m\'aquina
$I_{0}$ se almacena $\tilde{I_{0}}=I_{0}-\varepsilon_{0}$ (con lo
que se generar\'a una sucesi\'on $\tilde{I_{n}}\diagup\tilde{I}_{n+1}=-4\tilde{I_{n}}+1/(n+1)$).
Suponiendo que ese es el \'unico error que se comete en el proceso,
calcule el error $\varepsilon_{n}$ en el paso $n$ ($\varepsilon_{n}=I_{n}-\tilde{I_{n}}$)
en funci\'on de $n$ y $\varepsilon_{0}$
\begin{flushright}
 {\small \it (valor: 3 pts)}
\end{flushright}
\item Si $\varepsilon_{0}\approx10^{-6}$ y $n=12$, estime $\varepsilon_{12}$.
?`Qu\'e opina de este m\'etodo num\'erico para calcular $I_{12}$?
\begin{flushright}
 {\small \it (valor: 2 pts)}
\end{flushright}
\end{enumerate}

\item Si se utiliza la estrategia de redondeo correcto, ?`cu\'al es el n\'umero de m\'aquina del 
sistema $F(4, -10, 10, 10)$ que se obtiene del n\'umero que aproxima al n\'umero $\pi$?. ?`Y cu\'al 
ser\'ia el n\'umero de m\'aquina que aproximar\'ia a $\pi^4$?. Esta 
m\'aquina solo realiza las operaciones elementales $+,-,\times,\div$
\begin{flushright}
 {\small \it (valor: 3 pts)}
\end{flushright}
    \end{enumerate}
\end{document}