%------------------------------------------------------------------------
% Prueba Matematica I
%
%       Realizado por:
%       Jose Luis Ramirez
%       Lennys Bello
%                                                Actualizado: Noviembre 2007
% ------------------------------------------------------------------------
\documentclass[12pt,letterpaper]{article}
\usepackage[left=1.0cm,top=1.0cm,right=1.0cm,bottom=1.0cm]{geometry}
%\usepackage[activeacute,spanish]{babel}
\usepackage{amsmath}
\usepackage{graphicx, color}
\usepackage{amsfonts}
\usepackage{amssymb}
\usepackage{amsthm}
\usepackage{amsxtra}
\usepackage{amsgen}
\usepackage{amscd}
\usepackage{rotating}
\usepackage{multirow}
\usepackage{latexsym}
\usepackage{fancyhdr}
\pagestyle{empty}
\begin{document}
%%%%%%%%%%%%%%%%%%%%%%%%%%%%%%%%%%%%%%%%%%%%%%%%%%%%%%%%%%%%%%%%%%%%%%%%%%%%%%%%%%%%%%%%%%%%%%%%%%%%%%%%%
\begin{minipage}[ht]{3cm}
\includegraphics[width=2.5cm]{Logouc.jpg}
\end{minipage}
\begin{minipage}{10cm}
\centering\textbf{Universidad de Carabobo.\\
 Facultad de Ciencias y Tecnolog\'ia.\\
\'Departamento de Matem\'atica.\\
 M\'etodos Num\'ericos I.\\
}
\end{minipage}
\begin{minipage}[ht]{4cm}
\includegraphics[width=2.5cm]{facyt.jpg}
\end{minipage}

\begin{center}
\underline{\textbf{Asignaci\'on 2}}\\[7pt]
\end{center}

\vspace{0.5cm}

\begin{enumerate}

\item Aplique el m\'etodo de bisecci\'on, encuentre una ra\'iz de:
$$
f(x) = x^8 - 36x^7 + 546x^6 - 4536x^5 + 22449x^4 - 67284x^3 + 118124x^
2 - 109584x + 40320
$$
en el intervalo $[5.5; 6.5]$. Cambie -36 por -36,001 y repita el ejercicio y comente los resultados.

\item Reescriba el algoritmo de bisecci\'on si se modifica as\'i: Se secciona en tres el intervalo de estudio en cada
iteraci\'on (donde se encuentra la ra\'iz), se elige de los tres subintervalos el que contiene la ra\'iz y se asigna
como aproximaci\'on el punto medio de este. Aplique su algoritmo al problema anterior y compare los
resultados obtenidos, comente acerca de ello.

\item Demuestre que al usar el M\'etodo de Newton-Raphson, para aproximar el reciproco de
un n\'umero $S$, $S > 0$ se obtiene la f\'ormula iterativa $x_{k+1} = x_k(2 - S\cdot x_k); k = 0; 1; \ldots$. Calcular $1/17$ usando el algoritmo.

\item La suma de dos n\'umeros $a$ y $b$ es 21.2. Si a cada n\'umero se le a\~nade la ra\'iz cuadrada de s\'i mismo (es
decir, $a + \sqrt{a}$), el producto de las dos sumas es 170.73. Usando el M\'etodo de Newton o el de la Secante, determine
los dos n\'umeros dentro de una tolerancia de $10^{-4}$. Considere $x_0 = 0$ y/o $x_1 = 1$ como valores
iniciales. Ayuda: Conforme un sistema de dos ecuaciones con dos inc\'ognitas y red\'uzcalo a una ecuaci\'on
con una inc\'ognita.

\item \begin{enumerate}
    \item Use el M\'etodo de Newton para hallar el m\'inimo de la siguiente funci\'on polin\'omica
    $$
    f(x) = 230x^4 + 18x^3 + 9x^2 - 221x - 9
    $$    
    Asigne el resultado a la variable $x_N$. Determine las funciones necesarias a mano.
    \item Use el M\'etodo de la Secante para aproximar las ra\'ices de $f(x)$ sobre los intervalos $[-2, x_N]$ y
    $[x_N, 2]$ (use los extremos de estos intervalos como aproximaciones iniciales). Indique la utilidad de
    hallar el m\'inimo de la funci\'on para dividir el int\'ervalo.
\end{enumerate}

\item Estudiemos la aplicaci\'on de la iteraci\'on de punto fijo al c\'alculo de las ra\'ices de la funci\'on
$$
f(x) = -x^2 + x + \sin(x + 0.15)
$$
Para ello:
 \begin{enumerate}
  \item Dibuje la gr\'afica de $f$ y localice la ra\'iz de menor valor absoluto.Aplique la iteraci\'on de punto fijo a $g(x) = x - f(x)$ para aproximar esa ra\'iz con tolerancia=$10^{-15}$. Dibuje la
 gr\'afica de los valores $x^{(k)}$. ?`Hay convergencia despu\'es de 100 iteraciones?

 \item Aplique el siguiente esquema de iteraci\'on para diversos valores de $\alpha \in (-1,1)$ 
 $$
 g(x) = x - \alpha f(x)
 $$
 para mejorar las propiedades de convergencia. Vuelva a aplicar la iteraci\'on de punto fijo y compare los resultados.

 \item Estudie si la misma iteraci\'on de punto fijo permite hallar la menor ra\'iz en m\'odulo de $f$.

 \item Aplique el algoritmo $\Delta^2$ de Aitken para acelerar la convergencia de la sucesi\'on de aproximaciones obtenida.
Compare la velocidad de convergencia en cada caso.
 \end{enumerate}
\item Considere la ecuaci\'on:
\begin{equation}\label{eq1}
 2x^3 = 3x+4
\end{equation}

\begin{enumerate}
 \item Demostrar que la funci\'on $f(x) = 2x^3 - 3x - 4$ tiene una \'unica ra\'iz real $\alpha$ en $(1,2)$.
 \item ?`Se puede usar bisecci\'on para hallar una aproximaci\'on a $\alpha$?
 \item Convertir el problema de calcular la ra\'iz $\alpha$ de la ecuaci\'on (\ref{eq1}) en un problema de P.F. en el intervalo $[1,2]$. Ensayar por lo menos cuatro funciones de iteraci\'on en $[1,2]$.
 \item Demostrar que la funci\'on  $g(x) = \sqrt[3]{\dfrac{3x+4}{2}}$  es una funci\'on de iteraci\'on para el problema. Demuestre que:
 \begin{enumerate}
  \item $1 < \sqrt[3]{\dfrac{7}{2}}\leq g(x) \leq \sqrt[3]{5} < 2 \quad \forall x \in [1,2]$
  \item $\dfrac{1}{\sqrt[3]{200}} \leq g'(x) \leq \dfrac{1}{\sqrt[3]{98}} \quad \forall x \in [1,2]$
  \item $g(x)$ cumple las hip\'otesis del T.P.F. en $[1,2]$. Concluir. Calcular 3 iteraciones de P.F.
 \end{enumerate}
\end{enumerate}
\item Diversos cables, como los de tel\'efono, l\'ineas el\'ectricas o TV por cable, suelen colgarse de postes de servicios p\'ublicos.
\begin{figure}[!h]
    \centering
    \includegraphics[width=0.5\textwidth]{Screenshot_219.png}
    \caption{Cable colgando de un poste.}
\end{figure}
El cable tiene un peso uniforme $w$ (en Newtons/metro) por unidad de longitud. A partir del equilibrio de fuerzas en las direcciones 
$x$ y $y$, se puede derivar la siguiente ecuaci\'on diferencial para la altura 
$y$ del cable en cada punto $x$:
$$
\dfrac{d^2y}{dx^2} = \dfrac{w}{T}\sqrt{1+\left(\dfrac{dy}{dx}\right)^2}
$$
Donde $T$ es la fuerza de tensi\'on a lo largo del cable (en Newtons).

La soluci\'on de esta ecuaci\'on diferencial es:
$$
y = \dfrac{T}{w}\cosh\left(\dfrac{wx}{T}\right) + y_{min}-\dfrac{T}{w}
$$
Donde $\cosh(x)$ es el coseno hiperb\'olico ($cosh(x) = \dfrac{e^x + e^{-x}}{2}$).
\begin{itemize}
    \item Debe escribir un script en Matlab para calcular la tensi\'on de tensi\'on 
    $T$ en un cable tendido entre dos postes de altura 
    $y$ en las posiciones $-x$ y $+x$ en la dirección horizontal. El programa debe garantizar que el cable est\'e a una altura m\'inima de 
    $y_{min}$ por encima del suelo.
    \item Despu\'es de calcular esta tensi\'on, usa la ecuaci\'on dada para dibujar un gr\'afico de los valores de la altura del cable entre los dos postes.
    \item No es necesario verificar la validez de las entradas.
    \item No necesita usar la ecuaci\'on diferencial dada en la introducci\'on para resolver la tarea.
    \item Pruebe su programa con diferentes entradas. Ejemplo: para $x=50$, $y=15$, $y_{min}=8$ y 
    $w=8$, la tensi\'on $T$ debe ser aproximadamente 1258.
\end{itemize}
\end{enumerate}
\end{document}
