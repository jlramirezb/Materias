\documentclass[12pt]{beamer}
\usepackage[utf8]{inputenc}
\usepackage[spanish]{babel}
\usepackage{graphicx}
\usepackage{amsmath}
\usepackage{amsfonts}
\usepackage{amssymb}
\usepackage{amsthm}
\usepackage{listings}
\usetheme{Madrid}
%\usetheme{JuanLesPins}
%\usetheme{Montpellier}
%\usetheme{Berkeley}
%\usetheme{PaloAlto}
\usefonttheme{serif}
\usepackage{verbatim}

\title{Curso de \LaTeX}
%\author{Nombre del Autor}
\date{\today}

\begin{document}
\begin{frame}
\titlepage
\end{frame}
%%%%%%
\begin{frame}
\tableofcontents
\end{frame}
%%%%%%
\section{Introducción}
\begin{frame}{¿Qué es \LaTeX?}
  \begin{itemize}
    \item \LaTeX{} es un sistema de preparación de documentos.
    \item<2-> Utilizado para la creación de documentos científicos y técnicos.
  \end{itemize}
\end{frame}
%%%%%%
\begin{frame}{¿Qué se necesita?}
  \begin{itemize}
    \item A diferencia de muchos programas informáticos, \LaTeX{} no es una única aplicación que lo ``contenga todo'' en un solo lugar.
    \item<2-> En cambio, consta de programas separados que trabajan en conjunto.
    \item<3-> Podemos dividirlos en dos elementos que realmente se necesitan:
    \begin{itemize}
      \item<4-> Un sistema TeX.
      \item<5-> Un editor de texto.
    \end{itemize}
  \end{itemize}  
\end{frame}
%%%%%%
\begin{frame}{Sistema \LaTeX}
  \begin{itemize}
    \item<1-> El núcleo del trabajo con \LaTeX{} es tener disponible un sistema TeX. 
    \item<2-> Un sistema TeX es un conjunto de programas y archivos necesarios para que \LaTeX{} funcione.
    \item<3-> Existen dos sistemas TeX principales: MiKTeX y TeX Live. Ambos disponibles para Windows, macOS y Linux.
    \item<4-> MiKTeX tiene un fuerte respaldo en Windows; en macOS, TeX Live está incluido en una colección más grande llamada MacTeX.
  \end{itemize}
\end{frame}
%%%%%%
\begin{frame}{Editor de Texto}
  \begin{itemize}
    \item Los archivos de \LaTeX{} son archivos de texto plano con extensión {\color{blue}\texttt{.tex}}, por lo que pueden editarse con cualquier editor de texto.
    \item<2-> Sin embargo, es conveniente utilizar un editor diseñado para trabajar con LaTeX, ya que ofrecen funciones como:
    \begin{itemize}
      \item<3-> Compilación de archivos con un solo clic.
      \item<4-> Visores de PDF integrados.
      \item<5-> Resaltado de sintaxis.
    \end{itemize}
    \item<6-> Existen muchos editores de \LaTeX{}, entre los que podemos enumerar.
    \begin{itemize}
      \item<7-> TeXworks, está incluido en TeX Live y MiKTeX para Windows y Linux
      \item<8-> TeXShop, incluido en MacTeX.
      \item<9-> Winedt, un editor comercial para Windows.
      \item<10-> Overleaf, un editor en línea.
    \end{itemize}    
  \end{itemize}
\end{frame}
%%%%%%
\begin{frame}{Editor de Texto}  
    \begin{figure}
      \centering
      \includegraphics[width=0.8\textwidth]{Editores.png}
      \caption{Distintos editores de \LaTeX.}
      \label{fig:ejemplo_imagen}
    \end{figure}
\end{frame}
%%%%%%
\section{Estructura Básica}
\subsection{Documento Mínimo}
\begin{frame}{Documento Mínimo}
  \begin{itemize}
    \item La estructura básica de un documento es la siguiente:
    \begin{center}
    \begin{minipage}{0.5\textwidth}          
    \begin{block}{}       
       \color{blue}
        \texttt{\textbackslash documentclass\{article\}\\
        \textbackslash begin\{document\}\\
         \hspace{1cm}Hello World!\\
        \textbackslash end\{document\}}
      \end{block}
      \end{minipage}
    \end{center}
    \item<2-> El comando {\color{blue}\texttt{\textbackslash documentclass}} indica el tipo de documento que se va a crear.
    \item<3-> El \emph{argumento} en llaves {\color{blue}\texttt{\{} \texttt{\}}} le dice a \LaTeX{} qué tipo de documento estamos creando: en este ejemplo, {\color{blue}\texttt{article}}.
    \item<4-> Un signo de porcentaje {\color{blue}\texttt{\%}} comienza un \emph{comentario} --- \LaTeX{} ignorará el resto de la línea.
  \end{itemize}
\end{frame}
%%%%%%
\begin{frame}{\texttt{\textbackslash documentclass}}
  \begin{itemize}
    \item {\color{blue}\texttt{\textbackslash documentclass}} es un comando que le dice a \LaTeX{} qué tipo de documento estamos creando.
    \item<2-> Algunos tipos de documentos comunes son:
    \begin{itemize}
      \item {\color{blue}\texttt{article}}: Artículos de revistas, presentaciones, informes cortos, documentación, invitaciones, etc.
      \item<3-> {\color{blue}\texttt{report}}: Informes más largos que contienen varios capítulos, libros pequeños, tesis, etc.
      \item<4-> {\color{blue}\texttt{book}}: Libros.
      \item<5-> {\color{blue}\texttt{letter}}: Cartas.
      \item<6-> {\color{blue}\texttt{beamer}}: Presentaciones.
    \end{itemize}
  \end{itemize}
\end{frame}
%%%%%%
\begin{frame}{\texttt{\textbackslash documentclass}}
  \begin{itemize}
    \item<1-> El comando {\color{blue}\texttt{\textbackslash documentclass}} posee conjuntos de opciones que van entre corchetes {\color{blue}\texttt{[ ]}}. Algunas de ellas son:
    \begin{itemize}
      \item {\color{blue}\texttt{10pt, 11pt, 12pt}}: Tamaño de la fuente.
      \item<2-> {\color{blue}\texttt{a4paper, letterpaper, legalpaper}}: Tamaño del papel.
      \item<3-> {\color{blue}\texttt{twocolumn}}: Dos columnas.
      \item<4-> {\color{blue}\texttt{twoside, oneside}}: Impresión a doble o una cara.
    \end{itemize}
    \item<5-> Por ejemplo, {\color{blue}\texttt{\textbackslash documentclass[12pt,a4paper]\{article\}}} indica que el documento será un artículo con fuente
    de 12 puntos y tamaño de papel A4.
  \end{itemize}
\end{frame}
%%%%%%
\subsection{Paquetes}
\begin{frame}{Paquetes}
  \begin{itemize}
    \item<1->Los paquetes son archivos que contienen comandos y entornos adicionales para \LaTeX.
    \item<2-> Se cargan en el preámbulo del documento después del comando {\color{blue}\texttt{\textbackslash documentclass}}.
    \item<3-> El comando {\color{blue}\texttt{\textbackslash usepackage[]\{\}}} permite cargar un complemento (plugin), que añade nuevas funcionalidades.
    \end{itemize}    
\end{frame}  
%%%%%%
\begin{frame}{Paquetes}
  \begin{itemize}
    \item<1-> Existen numerosos complementos (por ejemplo, para mostrar imágenes, crear tablas, dibujar fórmulas químicas, generar cuadrículas de sudoku, etc.).

    {\color{red} Ejemplos}:
    \begin{itemize}
      \item<2-> {\color{blue}\texttt{\textbackslash usepackage[utf8]\{inputenc\}}}: Carga el paquete inputenc con la opción utf8 (esto es para la codificación de caracteres).
      \item<3-> {\color{blue}\texttt{\textbackslash usepackage[T1]\{fontenc\}}}: Especifica que se está utilizando el paquete de fuentes T1.
      \item<4-> {\color{blue}\texttt{\textbackslash usepackage[spanish]\{babel\}}}: Carga el paquete babel, que se encarga de la tipografía con el idioma español.
      \item<5-> {\color{blue}\texttt{\textbackslash usepackage\{graphicx\}}}: Carga el paquete que permite incluir imágenes externas en el documento.
    \end{itemize}
  \end{itemize}
\end{frame}
%%%%%%
\subsection{Entornos}
\begin{frame}{Entornos}
  \begin{itemize}
    \item<1-> Los entornos definen un ``bloque'': todo el texto dentro de este bloque se transformará según la función del entorno.
    \item<2-> Un entorno siempre comienza con {\color{blue}\texttt{\textbackslash begin\{\}}} y termina con {\color{blue}\texttt{\textbackslash end\{\}}}. Dentro de las {\color{blue}\{ \}} se especifica el nombre del entorno.
    \item<3-> El entorno {\color{blue}\texttt{document}} es obligatorio: lo que está dentro constituye el contenido del documento. Fuera del bloque document, se encuentran comandos que modifican las características del documento o cómo se imprime (por ejemplo, paquetes o comandos globales).
    \item <4-> Todos los demás entornos son opcionales y se usan según sea necesario.
    
    {\color{red}Ejemplo}:
    El entorno {\color{blue}\texttt{itemize}} crea listas con viñetas (listas sin numerar). Por lo tanto, una lista con viñetas se crea cada vez que se llama al comando {\color{blue}\texttt{\textbackslash item}}.
  \end{itemize}
\end{frame}
%%%%%%
\subsection{Dando Formato al texto}
\begin{frame}{Dando Formato al texto}
  \LaTeX{} tiene comandos para dar formato al texto.
  \begin{itemize}
    \item<1-> {\color{blue}\texttt{\textbackslash textbf\{texto\}}}: \textbf{Texto en negrita}.
    \item <2-> {\color{blue}\texttt{\textbackslash textit\{texto\}}}: \textit{Texto en cursiva}.
    \item <3-> {\color{blue}\texttt{\textbackslash underline\{texto\}}}: \underline{Texto subrayado}.
    \item <4-> {\color{blue}\texttt{\textbackslash texttt\{texto\}}}: \texttt{Texto en fuente de máquina de escribir}.
    \item <5-> {\color{blue}\texttt{\textbackslash color\{nombre color\}}}: \color{green}{Texto en color}.
    \item<6-> Formato de tamaño de fuente: {\color{blue}\texttt{\textbackslash tiny}}, {\color{blue}\texttt{\textbackslash scriptsize}}, {\color{blue}\texttt{\textbackslash footnotesize}}, {\color{blue}\texttt{\textbackslash small}}, {\color{blue}\texttt{\textbackslash normalsize}}, {\color{blue}\texttt{\textbackslash large}}, {\color{blue}\texttt{\textbackslash Large}}, {\color{blue}\texttt{\textbackslash LARGE}}, {\color{blue}\texttt{\textbackslash huge}}, {\color{blue}\texttt{\textbackslash Huge}}.
    \item<7-> Forzar un salto de línea: {\color{blue}\texttt{\textbackslash\textbackslash}}. 
  \end{itemize}
\end{frame}
%%%%%%
\section{Listas}
\begin{frame}{Dando Formato al texto}
  El entorno {\color{blue}\texttt{itemize}} crea listas con viñetas (listas sin numerar).
  \begin{center}
    \begin{minipage}{0.5\textwidth} 
  \begin{block}{}
    \texttt{\textbackslash begin\{itemize\}\\
      \textbackslash item Elemento 1\\
      \textbackslash item Elemento 2\\
      \textbackslash item Elemento 3\\
    \textbackslash end\{itemize\}}
  \end{block}
  \end{minipage}
  \end{center}
  \begin{itemize}
    \item Elemento 1
    \item Elemento 2
    \item Elemento 3
  \end{itemize}
\end{frame}
%%%%%%
\begin{frame}{Dando Formato al texto}
  El entorno {\color{blue}\texttt{enumerate}} crea listas numeradas.
  \begin{center}
    \begin{minipage}{0.5\textwidth} 
  \begin{block}{}
    \texttt{\textbackslash begin\{enumerate\}\\
      \textbackslash item Elemento 1\\
      \textbackslash item Elemento 2\\
      \textbackslash item Elemento 3\\
    \textbackslash end\{enumerate\}}
  \end{block}
  \end{minipage}
  \end{center}
  \begin{enumerate}
    \item Elemento 1
    \item Elemento 2
    \item Elemento 3
  \end{enumerate}
\end{frame}
%%%%%%
\section{Secciones}
\begin{frame}{Secciones}
  LaTeX puede dividir/estructurar documentos en múltiples niveles jerárquicos, dependiendo del tipo de documento con el que se trabaje.
  \begin{itemize}
    \item<1-> {\color{blue}\texttt{\textbackslash section\{texto\}}}: Secci\'on.
    \item<2-> {\color{blue}\texttt{\textbackslash subsection\{texto\}}}: Subsecci\'on.
    \item<3-> {\color{blue}\texttt{\textbackslash subsubsection\{texto\}}}: Subsubsecci\'on.
    \item <4-> {\color{blue}\texttt{\textbackslash paragraph\{texto\}}}: Párrafo.
    \item <5-> {\color{blue}\texttt{\textbackslash subparagraph\{texto\}}}: Subpárrafo.
    \item <6-> {\color{blue}\texttt{\textbackslash chapter\{texto\}}}: Cap\'itulo.
  \end{itemize}
\end{frame}
%%%%%%
\begin{frame}{Secciones}
  \begin{center}
    \begin{minipage}{0.8\textwidth}
  \begin{block}{}
    \footnotesize
    \textbackslash documentclass\{article\}\\
    \textbackslash usepackage[T1]\{fontenc\}\\
    \textbackslash begin\{document\}\\
\hspace{1cm}Hola Mundo!\\[10pt]
\hspace{1cm}Primer Documento.\\[10pt]

\hspace{1cm}\textbackslash section\{Primera Sección\}\\
\hspace{1cm}Texto de la primera sección.\\[10pt]

\hspace{1cm}Segundo Párrafo.\\[10pt]

\hspace{1cm}\textbackslash subsection\{Subseccion de la primera sección\}\\

\hspace{1cm}Texto de la subsección.\\[10pt]

\hspace{1cm}\textbackslash subsubsection\{Subsubsección de la primera sección\}\\

\hspace{1cm}Texto de la subsubsección.\\

\textbackslash end\{document\}\\
  \end{block}
  \end{minipage}
  \end{center}
\end{frame}
%%%%%%
\begin{frame}{Secciones}
  \begin{center}
    \includegraphics[scale=0.9]{Secciones.png}  
  \end{center}
\end{frame}
%%%%%%
\begin{frame}{Aclaraciones}
  \begin{itemize}
    \item Las comillas son un poco complicadas: use el acento invertido \texttt{\`{}} sobre el lado izquierdo y el apóstrofe \texttt{'{}} sobre el lado derecho.
      
  \begin{tabular}{ll}
     Comillas simple: & `texto'.\\  
   Comillas dobles: & ``texto''.
  \end{tabular}
    \item<2-> Algunos caracteres comunes tienen significados especiales en \LaTeX:  
  \begin{center}
    \begin{tabular}{cl}
      \texttt{\%} & Signo de porcentaje \\
      \texttt{\#} & Signo numeral \\
      \texttt{\&} & Ampersand                 \\
      \texttt{\$} & Signo pesos               \\
    \end{tabular}
  \end{center}
    \item<3-> Si son usados, tendremos errores en la compilación. Para usar estos caracteres en la salida, se debe colocar barra invertida al caracter.
      \begin{center}
      \begin{minipage}{3.5cm}
        \begin{block}{}        
          \textbackslash\$\hspace{10pt}\textbackslash\%\hspace{10pt}\textbackslash\&\hspace{10pt}\textbackslash\#!              
      \end{block}
      \end{minipage}
    \end{center}      
  \end{itemize}  
\end{frame}
%%%%%%
\begin{frame}{Aclaraciones}
  \begin{itemize}
    \item<1-> Los espacios en blanco en el código fuente de \LaTeX{} no tienen ningún efecto en el documento final.
    \item<2-> \LaTeX{} trata los espacios en blanco como ``espacios en blanco''.
    \item<3-> Para obtener un espacio en blanco en el documento final, se deben usar comandos especiales.
    \item<4-> Para obtener un espacio en blanco en el documento final, se deben usar comandos especiales.    
  \end{itemize}
\end{frame}
%%%%%%
\begin{frame}{Tipografía Matemática}
  \begin{itemize}\small{
    \item<1-> ¿Por qué son especiales los signos pesos \texttt{\textbackslash \$}? Los
    usamos para marcar contenido  matemático en el texto.\\[1ex]    
    \begin{tabular}{|p{5cm}|p{5cm}|}\hline
\color{green}\% no tan bueno: & \\
Sean a y b tales que c = a - b + 1.&Sean a y b tales que c = a - b + 1.\\
\color{green}\% mucho mejor: & \\
Sean \$a\$ y \$b\$ tales que \$c = a - b + 1\$. &
Sean $a$ y $b$ tales que $c = a - b + 1$.\\\hline
    \end{tabular}
  \item Utilice siempre los signos de pesos en pares --- uno para
    comenzar el contenido matemático, y uno para terminarlo.
  \item \LaTeX{} maneja el espacio automáticamente; por lo que
    ignorará los que hayamos puesto.\\[1ex]
    \begin{center}    
    \begin{tabular}{|l|l|}\hline
Sea \$y=mx+b\$ \ldots & Sea $y=mx+b$ \ldots\\[5pt]      
Sea \$y = m x + b\$ \ldots & Sea $y = m x + b$ \ldots\\\hline
    \end{tabular}
    \end{center}}
  \end{itemize}
\end{frame}
%%%%%%
\begin{frame}{Tipografía Matemática:Notaci\'on}
  \begin{itemize}
    \item<1-> Use el signo \texttt{\^} para indicar superíndices y el
    guión bajo \texttt{\_} para marcar subíndices.

    \begin{tabular}{|l|l|}\hline
      \texttt{\$y = c\_2 x\texttt{\^}2 + c\_1 x + c\_0\$} & $y = c_2 x^2 + c_1 x + c_0$\\\hline
    \end{tabular}
    \vskip 2ex
    
  \item Utilice las llaves \texttt{\{} \texttt{\}} para
    agrupar superíndices y subíndices.
    \begin{tabular}{|l|l|}\hline      
\texttt{\$F\_n = F\_n-1 + F\_n-2\$ \% oops!} & $F_n = F_n-1 + F_n-2$\\ 
\texttt{\$F\_n = F\_\{n-1\} + F\_\{n-2\}\$ \% ok!} & $F_n = F_{n-1} + F_{n-2}$\\\hline
    \end{tabular}
    \vskip 2ex
    
  \item Hay comandos para letras Griegas y notación común.
    \begin{tabular}{|l|l|}\hline
      \texttt{\$\textbackslash mu = A e\^\{Q/RT\}\$} & $\mu = A e^{Q/RT}$\\
      \texttt{\$\textbackslash Omega = \textbackslash sum\_\{k=1\}\texttt{\^}n \textbackslash omega\_k\$} & $\Omega = \sum_{k=1}^{n} \omega_k$\\\hline
    \end{tabular}
  \end{itemize}
\end{frame}
%%%%%%
\begin{frame}[fragile]{Tipografía Matemática:Entornos}
  \begin{itemize}
  \item {\color{blue}\texttt{equation}} es un \emph{entorno}.
  \item Un comando puede producir diferentes salidas en diferentes contextos.
  \begin{columns}
    \begin{column}{0.5\textwidth}
      Podemos escribir
$ \Omega = \sum_{k=1}^{n} \omega_k $
en nuestro texto, o podemos escribir
\begin{equation}
  \Omega = \sum_{k=1}^{n} \omega_k
\end{equation}
para mostrarlo en un entorno diferente.
    \end{column}
    \begin{column}{0.5\textwidth}
      $ \Omega = \sum_{k=1}^{n} \omega_k $
\begin{equation}
  \Omega = \sum_{k=1}^{n} \omega_k
\end{equation}
    \end{column}
  \end{columns}    
    \vskip 2ex
  \item Note como el $\Sigma$ es más grande en el entorno
    \bftt{equation}, y como el subíndice y superíndice cambian de
    posición, a pesar de que utilizamos los mismos comandos.
    \vskip 1em
    {\scriptsize Incluso, podríamos haber escrito \bftt{\$...\$} como
      \cmdbegin{math}\bftt{...}\cmdend{math}.}
  \end{itemize}
\end{frame}
\end{document}