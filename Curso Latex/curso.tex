\documentclass[12pt]{beamer}
\usepackage[utf8]{inputenc}
\usepackage[spanish]{babel}
\usepackage{graphicx}
\usepackage{amsmath}
\usepackage{amsfonts}
\usepackage{amssymb}
\usepackage{amsthm}
\usetheme{Madrid}
%\usetheme{JuanLesPins}
%\usetheme{Montpellier}
%\usetheme{Berkeley}
%\usetheme{PaloAlto}
\usefonttheme{serif}
\usepackage{verbatim}

\title{Curso de \LaTeX}
%\author{Nombre del Autor}
\date{\today}

\begin{document}

\begin{frame}
\titlepage
\end{frame}

\begin{frame}
\tableofcontents
\end{frame}

\section{Introducción}
\begin{frame}{¿Qué es \LaTeX?}
  \begin{itemize}
    \item \LaTeX es un sistema de preparación de documentos.
    \item<2-> Utilizado para la creación de documentos científicos y técnicos.
  \end{itemize}
\end{frame}

\begin{frame}{¿Qué se necesita?}
  \begin{itemize}
    \item A diferencia de muchos programas informáticos, \LaTeX no es una única aplicación que lo "contenga todo" en un solo lugar.
    \item<2-> En cambio, consta de programas separados que trabajan en conjunto.
    \item<3-> Podemos dividirlos en dos elementos que realmente se necesitan:
    \begin{itemize}
      \item<4-> Un sistema TeX.
      \item<5-> Un editor de texto.
    \end{itemize}
  \end{itemize}  
\end{frame}

\begin{frame}{Sistema \LaTeX}
  \begin{itemize}
    \item El núcleo del trabajo con \LaTeX es tener disponible un sistema TeX. 
    \item Un sistema TeX es un conjunto de programas y archivos necesarios para que LaTeX funcione.
    \item Existen dos sistemas TeX principales: MiKTeX y TeX Live. Ambos disponibles para Windows, macOS y Linux.
    \item MiKTeX tiene un fuerte respaldo en Windows; en macOS, TeX Live está incluido en una colección más grande llamada MacTeX.
  \end{itemize}
\end{frame}

\begin{frame}{Editor de Texto}
  \begin{itemize}
    \item Los archivos de \LaTeX son archivos de texto plano con extensión \texttt{.tex}, por lo que pueden editarse con cualquier editor de texto.
    \item<2-> Sin embargo, es conveniente utilizar un editor diseñado para trabajar con LaTeX, ya que ofrecen funciones como:
    \begin{itemize}
      \item<3-> Compilación de archivos con un solo clic.
      \item<4-> Visores de PDF integrados.
      \item<5-> Resaltado de sintaxis.
    \end{itemize}
    \item<6-> Existen muchos editores de LaTeX, entre los que podemos enumerar.
    \begin{itemize}
      \item<7-> TeXworks, está incluido en TeX Live y MiKTeX para Windows y Linux
      \item<8-> TeXShop, incluido en MacTeX.
      \item<9-> Winedt, un editor comercial para Windows.
      \item<10-> Overleaf, un editor en línea.
    \end{itemize}    
  \end{itemize}
\end{frame}

\begin{frame}{Editor de Texto}  
    \begin{figure}
      \centering
      \includegraphics[width=0.8\textwidth]{Editores.png}
      \caption{Distintos editores de LaTeX.}
      \label{fig:ejemplo_imagen}
    \end{figure}
\end{frame}

\section{Estructura Básica}
\subsection{Documento Mínimo}
\begin{frame}{Documento Mínimo}
  \begin{itemize}
    \item La estructura básica de un documento es la siguiente:
    \begin{center}
    \begin{minipage}{0.5\textwidth}    
    \begin{block}{}
       % Cambia el ancho aquí
        \texttt{\textbackslash documentclass\{article\}\\
        \textbackslash begin\{document\}\\
         \hspace{1cm}Hello World!\\
        \textbackslash end\{document\}}
      \end{block}
      \end{minipage}
    \end{center}
    \item<2-> El comando \texttt{\textbackslash documentclass} indica el tipo de documento que se va a crear.
    \item<3-> El \emph{argumento} en llaves \texttt{\{} \texttt{\}} le dice a \LaTeX qué tipo de documento estamos creando: en este ejemplo, \texttt{article}.
    \item<4-> Un signo de porcentaje \texttt{\%} comienza un \emph{comentario} --- \LaTeX ignorará el resto de la línea.
  \end{itemize}
\end{frame}
%%%%%%
\begin{frame}{\texttt{\textbackslash documentclass}}
  \begin{itemize}
    \item \texttt{\textbackslash documentclass} es un comando que le dice a \LaTeX qué tipo de documento estamos creando.
    \item<2-> Algunos tipos de documentos comunes son:
    \begin{itemize}
      \item \texttt{article}: Artículos de revistas, presentaciones, informes cortos, documentación, invitaciones, etc.
      \item<3-> \texttt{report}: Informes más largos que contienen varios capítulos, libros pequeños, tesis, etc.
      \item<4-> \texttt{book}: Libros.
      \item<5-> \texttt{letter}: Cartas.
      \item<6-> \texttt{beamer}: Presentaciones.
    \end{itemize}
  \end{itemize}
\end{frame}
%%%%%%
\begin{frame}{\texttt{\textbackslash documentclass}}
  \begin{itemize}
    \item<1-> El comando \texttt{\textbackslash documentclass} posee conjuntos de opciones que van entre corchetes \texttt{[ ]}. Algunas de ellas son:
    \begin{itemize}
      \item \texttt{10pt, 11pt, 12pt}: Tamaño de la fuente.
      \item<2-> \texttt{a4paper, letterpaper, legalpaper}: Tamaño del papel.
      \item<3-> \texttt{twocolumn}: Dos columnas.
      \item<4-> \texttt{twoside, oneside}: Impresión a doble o una cara.
    \end{itemize}
    \item<5-> Por ejemplo, \texttt{\textbackslash documentclass[12pt,a4paper]\{article\}} indica que el documento será un artículo con fuente
    de 12 puntos y tamaño de papel A4.
  \end{itemize}
\end{frame}
%%%%%%
\begin{frame}{Paquetes}
  \begin{itemize}
    \item<1-> Los paquetes son archivos que contienen comandos y entornos adicionales para \LaTeX.
    \item<2-> Se cargan en el preámbulo del documento con el comando \texttt{\textbackslash usepackage}.
    \item<3-> Por ejemplo, el paquete \texttt{geometry} permite cambiar el tamaño de la fuente y el margen del documento.
  \end{itemize}    
\end{frame}
\end{document}