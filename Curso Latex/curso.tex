\documentclass[12]{beamer}
\usepackage[utf8]{inputenc}
\usepackage[spanish]{babel}
\usepackage{graphicx}
\usepackage{amsmath}
\usepackage{amsfonts}
\usepackage{amssymb}
\usepackage{amsthm}
%\usetheme{Madrid}
%\usetheme{JuanLesPins}
%\usetheme{Montpellier}
%\usetheme{Berkeley}
\usetheme{PaloAlto}
\usefonttheme{serif}

\title{Curso de \LaTeX}
%\author{Nombre del Autor}
\date{\today}

\begin{document}

\frame{\titlepage}

\frame{\tableofcontents}
\section{Introducción}
\begin{frame}{¿Qu\'e es \LaTeX?}
  \begin{itemize}
    \item \LaTeX es un sistema de preparaci\'on de documentos.
    \item<2-> Utilizado para la creaci\'on de documentos cient\'ificos y t\'ecnicos.
  \end{itemize}
\end{frame}
%%%%%%%
\begin{frame}{?`Qu\'e se necesita?}
  \begin{itemize}
    \item A diferencia de muchos programas inform\'aticos, \LaTeX no es una \'unica aplicaci\'on que lo ``contenga todo'' en un solo lugar.
    \item<2-> En cambio, consta de programas separados que trabajan en conjunto.
    \item<3-> Podemos dividirlos en dos elementos que realmente se necesitan:
    \begin{itemize}
      \item<4-> Un sistema TeX.
      \item<5-> Un editor de texto.
    \end{itemize}
  \end{itemize}  
\end{frame}
%%%%%%%
\begin{frame}{Sistema \LaTeX}
  \begin{itemize}
    \item El n\'ucleo del trabajo con \LaTeX es tener disponible un sistema TeX. 
    \item Un sistema TeX es un conjunto de programas y archivos necesarios para que LaTeX funcione.
    \item Existen dos sistemas TeX principales: MiKTeX y TeX Live. Ambos disponibles para Windows, macOS y Linux.
    \item MiKTeX tiene un fuerte respaldo en Windows; en macOS, TeX Live está incluido en una colecci\'on m\'as grande llamada MacTeX.
    \item Cada sistema tiene sus ventajas, y quizás quieras consultar más consejos del LaTeX Project.    
  \end{itemize}
\end{frame}
%%%%%%%
\begin{frame}{Editor de Texto}
  \begin{itemize}
    \item Los archivos de LaTeX son archivos de texto plano con extensión \texttt{.tex}, por lo que pueden editarse con cualquier editor de texto.
    \item<2-> Sin embargo, es conveniente utilizar un editor dise\~nado para trabajar con LaTeX, ya que ofrecen funciones como:
    \begin{itemize}
      \item<3-> Compilaci\'on de archivos con un solo clic.
      \item<4-> Visores de PDF integrados.
      \item<5-> Resaltado de sintaxis.
    \end{itemize}
    \item<6-> Existen muchos editores de LaTeX, entre los que podemos enumerar.
    \begin{itemize}
      \item<7-> TeXworks, est\'a incluido en TeX Live y MiKTeX para Windows y Linux
      \item<8-> TeXShop, incluido en MacTeX.
      \item<9-> Winedt, un editor comercial para Windows.
      \item<10-> Overleaf, un editor en l\'inea.
    \end{itemize}    
  \end{itemize}
\end{frame}

\section{Estructura Básica}
\subsection{Documento Mínimo}
\begin{frame}{Documento Mínimo}
  \begin{itemize}
    \item \textbackslash documentclass\{article\}
    \item \textbackslash begin\{document\}
    \item \textbackslash end\{document\}
  \end{itemize}
\end{frame}

\subsection{Paquetes}
\begin{frame}{Paquetes}
  \begin{itemize}
    \item \textbackslash usepackage\{nombre\_del\_paquete\}
    \item Añaden funcionalidad extra a LaTeX.
  \end{itemize}
\end{frame}

\section{Elementos Básicos}
\subsection{Texto}
\begin{frame}{Texto}
  \begin{itemize}
    \item Formato de texto: \textbf{negrita}, \textit{cursiva}, \underline{subrayado}.
    \item Listas: \begin{itemize}
      \item Item 1
      \item Item 2
    \end{itemize}
  \end{itemize}
\end{frame}

\subsection{Matemáticas}
\begin{frame}{Matemáticas}
  \begin{itemize}
    \item Fórmulas en línea: \( E = mc^2 \)
    \item Fórmulas en display: \[ E = mc^2 \]
  \end{itemize}
\end{frame}

\section{Conclusión}
\subsection{Resumen}
\begin{frame}{Resumen}
  \begin{itemize}
    \item LaTeX es una herramienta poderosa para la creación de documentos.
    \item Permite un control preciso sobre el formato y la presentación.
  \end{itemize}
\end{frame}

\end{document}
