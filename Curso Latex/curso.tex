\documentclass[12]{beamer}

\title{Curso de LaTeX}
%\author{Nombre del Autor}
\date{\today}

\begin{document}

\frame{\titlepage}

\section{Introducción}
\subsection{¿Qué es LaTeX?}
\begin{frame}{¿Qué es LaTeX?}
  \begin{itemize}
    \item LaTeX es un sistema de preparación de documentos.
    \item Utilizado para la creación de documentos científicos y técnicos.
  \end{itemize}
\end{frame}

\subsection{Historia de LaTeX}
\begin{frame}{Historia de LaTeX}
  \begin{itemize}
    \item Creado por Leslie Lamport en 1984.
    \item Basado en el sistema de tipografía TeX de Donald Knuth.
  \end{itemize}
\end{frame}

\section{Estructura Básica}
\subsection{Documento Mínimo}
\begin{frame}{Documento Mínimo}
  \begin{itemize}
    \item \textbackslash documentclass\{article\}
    \item \textbackslash begin\{document\}
    \item \textbackslash end\{document\}
  \end{itemize}
\end{frame}

\subsection{Paquetes}
\begin{frame}{Paquetes}
  \begin{itemize}
    \item \textbackslash usepackage\{nombre\_del\_paquete\}
    \item Añaden funcionalidad extra a LaTeX.
  \end{itemize}
\end{frame}

\section{Elementos Básicos}
\subsection{Texto}
\begin{frame}{Texto}
  \begin{itemize}
    \item Formato de texto: \textbf{negrita}, \textit{cursiva}, \underline{subrayado}.
    \item Listas: \begin{itemize}
      \item Item 1
      \item Item 2
    \end{itemize}
  \end{itemize}
\end{frame}

\subsection{Matemáticas}
\begin{frame}{Matemáticas}
  \begin{itemize}
    \item Fórmulas en línea: \( E = mc^2 \)
    \item Fórmulas en display: \[ E = mc^2 \]
  \end{itemize}
\end{frame}

\section{Conclusión}
\subsection{Resumen}
\begin{frame}{Resumen}
  \begin{itemize}
    \item LaTeX es una herramienta poderosa para la creación de documentos.
    \item Permite un control preciso sobre el formato y la presentación.
  \end{itemize}
\end{frame}

\end{document}
