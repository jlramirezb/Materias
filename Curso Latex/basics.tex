%%%%%%%%%%%%%%%%%%%%%%%%%%%%%%%%%%%%%%%%%%%%%%%%%%%%%%%%%%%%%%%
%
% Welcome to Overleaf --- just edit your LaTeX on the left,
% and we'll compile it for you on the right. If you open the
% 'Share' menu, you can invite other users to edit at the same
% time. See www.overleaf.com/learn for more info. Enjoy!
%
%%%%%%%%%%%%%%%%%%%%%%%%%%%%%%%%%%%%%%%%%%%%%%%%%%%%%%%%%%%%%%%

\documentclass{article}
\usepackage[T1]{fontenc}
\usepackage[style=authoryear]{biblatex}
\addbibresource{learnlatex.bib} % archivo con información de las referencias

\begin{document}
La demostración matemática es de \citeyear{Graham1995}.

Algunos ejemplos de citas más complejas: \cite{Graham1995} o
\cite{Thomas2008} o posiblemente \cite{Graham1995}.

\cite[56]{Thomas2008}

\cite[Ver][45-48]{Graham1995}

Together \cite{Thomas2008,Graham1995}
%\printbibliography


\bibliographystyle{plain}
\bibliography{learnlatex}

\end{document}
