\documentclass{article}
\usepackage{amsmath}
\usepackage{amsfonts}
\usepackage{amssymb}
\usepackage[spanish]{babel}
\usepackage[utf8]{inputenc}
\usepackage{geometry}
\geometry{tmargin=1.5cm, lmargin=1.5cm, rmargin=1.5cm, bmargin=1.5cm}
\usepackage{pgfplots}
\usepackage{tikz}
\pgfplotsset{compat=1.18}

\begin{document}

\textbf{Pregunta:} Halle la longitud del arco de la curva $y = 3\operatorname{sen}(2x)$ en $[0, 2\pi]$ usando la regla trapezoidal con $n = 6$.

\textbf{Respuesta:}

Para hallar la longitud del arco de la curva $y = 3\operatorname{sen}(2x)$ en el intervalo $[0, 2\pi]$ usando la regla trapezoidal con $n = 6$, seguiremos los siguientes pasos:

\begin{enumerate}
    \item \textbf{Fórmula de la longitud del arco:}
    La longitud del arco $L$ de una curva $y = f(x)$ en el intervalo $[a, b]$ está dada por la fórmula:
    $$L = \int_{a}^{b} \sqrt{1 + \left(\frac{dy}{dx}\right)^2} \, dx$$

    \item \textbf{Calcular la derivada $\frac{dy}{dx}$:}
    Nuestra función es $y = 3\operatorname{sen}(2x)$. Derivamos $y$ con respecto a $x$:
    $$\frac{dy}{dx} = \frac{d}{dx} (3\operatorname{sen}(2x)) = 3 \cdot \cos(2x) \cdot \frac{d}{dx}(2x) = 3 \cos(2x) \cdot 2 = 6\cos(2x)$$

    \item \textbf{Calcular $\left(\frac{dy}{dx}\right)^2$:}
    $$\left(\frac{dy}{dx}\right)^2 = (6\cos(2x))^2 = 36\cos^2(2x)$$

    \item \textbf{Expresión para la longitud del arco:}
    Sustituyendo en la fórmula de la longitud del arco, obtenemos:
    $$L = \int_{0}^{2\pi} \sqrt{1 + 36\cos^2(2x)} \, dx$$

    \item \textbf{Regla Trapezoidal:}
    La regla trapezoidal para aproximar la integral $\int_{a}^{b} f(x) \, dx$ con $n$ subintervalos es:
    $$T_n = \frac{\Delta x}{2} \left[ f(x_0) + 2f(x_1) + 2f(x_2) + \dots + 2f(x_{n-1}) + f(x_n) \right]$$
    donde $\Delta x = \frac{b - a}{n}$ y $x_i = a + i\Delta x$.

    \item \textbf{Aplicar la regla trapezoidal con $n = 6$:}
    En nuestro caso, $a = 0$, $b = 2\pi$, y $n = 6$.

    Calculamos $\Delta x$:
    $$\Delta x = \frac{2\pi - 0}{6} = \frac{2\pi}{6} = \frac{\pi}{3}$$

    Los puntos $x_i$ son:
    $x_0 = 0, x_1 = \frac{\pi}{3}, x_2 = \frac{2\pi}{3}, x_3 = \pi, x_4 = \frac{4\pi}{3}, x_5 = \frac{5\pi}{3}, x_6 = 2\pi$

    Nuestra función a evaluar es $f(x) = \sqrt{1 + 36\cos^2(2x)}$. Calculamos los valores de $f(x_i)$:
    \begin{align*}
        f(x_0) &= f(0) = \sqrt{37} \approx 6.0828 \\
        f(x_1) &= f(\frac{\pi}{3}) = \sqrt{10} \approx 3.1623 \\
        f(x_2) &= f(\frac{2\pi}{3}) = \sqrt{10} \approx 3.1623 \\
        f(x_3) &= f(\pi) = \sqrt{37} \approx 6.0828 \\
        f(x_4) &= f(\frac{4\pi}{3}) = \sqrt{10} \approx 3.1623 \\
        f(x_5) &= f(\frac{5\pi}{3}) = \sqrt{10} \approx 3.1623 \\
        f(x_6) &= f(2\pi) = \sqrt{37} \approx 6.0828
    \end{align*}

    Ahora aplicamos la regla trapezoidal:
    \begin{align*}
        T_6 &= \frac{\frac{\pi}{3}}{2} \left[ f(0) + 2f(\frac{\pi}{3}) + 2f(\frac{2\pi}{3}) + 2f(\pi) + 2f(\frac{4\pi}{3}) + 2f(\frac{5\pi}{3}) + f(2\pi) \right] \\
        &= \frac{\pi}{6} \left[ \sqrt{37} + 2\sqrt{10} + 2\sqrt{10} + 2\sqrt{37} + 2\sqrt{10} + 2\sqrt{10} + \sqrt{37} \right] \\
        &= \frac{\pi}{6} \left[ 4\sqrt{37} + 8\sqrt{10} \right] \\
        &= \frac{2\pi}{3} \left[ \sqrt{37} + 2\sqrt{10} \right] \\
        &\approx \frac{2 \cdot 3.1416}{3} [6.0828 + 2(3.1623)] \\
        &\approx 2.0944 [6.0828 + 6.3246] \\
        &\approx 2.0944 \cdot 12.4074 \\
        &\approx 25.987
    \end{align*}

    Por lo tanto, la longitud del arco de la curva $y = 3\operatorname{sen}(2x)$ en $[0, 2\pi]$ usando la regla trapezoidal con $n = 6$ es aproximadamente $25.987$.
\end{enumerate}

\textbf{Pregunta:} Integre la siguiente función entre los límites $a=-1$ y $b=1$, utilizando 6 intervalos, usando la regla de Simpson:
\[
\int_{-1}^{1} \frac{1}{\sqrt{2\pi}} e^{-x^2/2} dx
\]

\textbf{Respuesta:}

Para integrar la función $f(x) = \frac{1}{\sqrt{2\pi}} e^{-x^2/2}$ entre los límites $a = -1$ y $b = 1$ utilizando la regla de Simpson con $n = 6$ intervalos, seguimos los siguientes pasos:

\begin{enumerate}
    \item \textbf{Fórmula de la regla de Simpson:}
    $$S_n = \frac{\Delta x}{3} \left[ f(x_0) + 4f(x_1) + 2f(x_2) + 4f(x_3) + 2f(x_4) + \dots + 4f(x_{n-1}) + f(x_n) \right]$$
    donde $\Delta x = \frac{b - a}{n}$ y $x_i = a + i\Delta x$.

    \item \textbf{Calcular $\Delta x$ y los puntos $x_i$:}
    $\Delta x = \frac{1 - (-1)}{6} = \frac{2}{6} = \frac{1}{3}$
    Los puntos $x_i$ son: $x_0 = -1, x_1 = -\frac{2}{3}, x_2 = -\frac{1}{3}, x_3 = 0, x_4 = \frac{1}{3}, x_5 = \frac{2}{3}, x_6 = 1$.

    \item \textbf{Evaluar la función $f(x) = \frac{1}{\sqrt{2\pi}} e^{-x^2/2} \approx 0.39894 e^{-x^2/2}$ en los puntos $x_i$:}
    \begin{align*}
        f(-1) &\approx 0.2420 \\
        f(-\frac{2}{3}) &\approx 0.3138 \\
        f(-\frac{1}{3}) &\approx 0.3779 \\
        f(0) &\approx 0.3989 \\
        f(\frac{1}{3}) &\approx 0.3779 \\
        f(\frac{2}{3}) &\approx 0.3138 \\
        f(1) &\approx 0.2420
    \end{align*}

    \item \textbf{Aplicar la regla de Simpson:}
    \begin{align*}
        S_6 &= \frac{1/3}{3} \left[ f(-1) + 4f(-\frac{2}{3}) + 2f(-\frac{1}{3}) + 4f(0) + 2f(\frac{1}{3}) + 4f(\frac{2}{3}) + f(1) \right] \\
        &\approx \frac{1}{9} \left[ 0.2420 + 4(0.3138) + 2(0.3779) + 4(0.3989) + 2(0.3779) + 4(0.3138) + 0.2420 \right] \\
        &\approx \frac{1}{9} \left[ 0.2420 + 1.2552 + 0.7558 + 1.5956 + 0.7558 + 1.2552 + 0.2420 \right] \\
        &\approx \frac{1}{9} [6.1016] \\
        &\approx 0.6780
    \end{align*}
\end{enumerate}

Por lo tanto, la aproximación de la integral utilizando la regla de Simpson con $n = 6$ intervalos es aproximadamente $0.6780$.

\end{document}